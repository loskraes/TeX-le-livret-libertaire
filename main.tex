\usepackage[utf8]{inputenc}
%\usepackage{geometry}
\usepackage{shorttoc}
\usepackage{setspace}

%\usepackage{fontspec}
%\setmainfont{Open Dyslexic}

\usepackage{titlesec}
\titleclass{\part}{top}
\titleformat{\part}[display]
  {\normalfont\huge\bfseries}{\centering\partname\ \thepart}{20pt}{\Huge\centering}
\titlespacing*{\part}{0pt}{0pt}{0pt}


\usepackage[french]{minitoc}
\setcounter{parttocdepth}{1}
\mtcsetfeature{parttoc}{open}{\vspace{0cm}}
\doparttoc[e]
%\doparttoc
%\renewcommand\ptctitle{}
\renewcommand\beforeparttoc{}

%p. 52


\usepackage[french]{babel}
\usepackage[pdfpagelayout=TwoPageRight,colorlinks,hidelinks]{hyperref}

\newcommand{\text}[3][\DefaultOpt]{%
	\def\DefaultOpt{#2}%
	\part[#1]{#2\\\vspace{10pt}\normalsize\normalfont\begin{spacing}{1.2} texte original de #3\end{spacing}}%
	\renewcommand\ptctitle{\thepart.\ #1}
	\mtcaddpart
	\parttoc
	\cleardoublepage
}

\renewcommand{\partmark}[1]{\markboth{\parbox{0.90\textwidth}{\raggedleft\thepart.\ \MakeUppercase#1}}{\parbox{0.90\textwidth}{\thepart.\ \MakeUppercase#1}}}
\renewcommand{\chaptermark}[1]{\markboth{\parbox{0.90\textwidth}{\raggedleft\thechapter.\ #1}}{\parbox{0.90\textwidth}{\thechapter.\ #1}}}
\renewcommand{\sectionmark}[1]{\markright{\parbox{0.90\textwidth}{\thesection.\ #1}}{}}
\addto\captionsfrench{\renewcommand{\partname}{Texte}}


%\usepackage[placement=center,angle=0,color=blue!95,scale=11]{background}
%\backgroundsetup{contents={-\thepage-}}

\title{Le Livret Libertaire\\[2em]\large Traduction de texte}
\author{\normalsize traduit et relié par @la.corvina \and \normalsize mis en page par @LoskrAES}

\begin{document} 

\frontmatter
\maketitle

\chapter*{information page}
Qui a fait quoi ici ?
\chapter*{introduction}
ptit texte introductif à ce qui suis.

\shorttoc{Liste des textes}{-1}

\mainmatter

\text{L'État}{la Fédération des Communistes Anarchistes (Italie)}
\chapter{Avant-propos}

\section*{Crise de l'État Providence et la consociation}

N'est-il pas contradictoire que face à la crise de l'État Providence, nous, communistes anarchistes, nous retrouvions dans notre action syndicale et politique parmi les rares partisans de l'intervention de l'État ? N'est-il pas paradoxal que ce soit précisément nous qui finissions par défendre la nécessité d'étendre l'intervention de l'État, alors qu'une des caractéristiques fondamentales de notre idéologie politique est celle de l'extinction de l'État ?

\section*{Crise de l'État Providence et des économies planifiées}

Comme on le sait, l'État Providence est né du keynésianisme et, par les forces de sociales-démocrates, il est devenu l'une des pierres angulaires du développement des sociétés régies par des systèmes capitalistes avancés. Il permet d'absorber les conflits sociaux et de les orienter vers une croissance plus équilibrée de l'accumulation, de réinvestir les salaires de manière à assurer une croissance régulière de l'économie, tout en garantissant de meilleures conditions de vie aux citoyens. L'État Providence n'élimine pas la pauvreté et l'inégalité dans la répartition des ressources, mais il rend certainement moins dramatique le conflit entre la misère et la richesse, il aurait dû garantir que certains services dits essentiels tels que les soins de santé, l'éducation, le droit au logement, à un salaire minimum, donc que des conditions de vie globalement acceptables puissent être fournies à tous.

Cette conception du rôle de l'État a été opposée par les pays dits du « socialisme réel » avec  celle de l’État planificateur qui, par la planification des ressources et de la production, était censé mettre en œuvre une répartition équitable des biens. Ce schéma de fonctionnement de l'État n'est pas exempt, comme celui de l'État Providence, de défauts ; il souffre de bureaucratisme pour l'un et de malhonnêteté et d'affairisme pour l'autre, à tel point que les raisons de critiquer les deux systèmes s'entremêlent souvent.

La longue phase d'expansion qu'a connue l'économie mondiale a mis en crise les deux systèmes de gestion sociale. D'où la crise des systèmes de « démocratie populaire » et celle des systèmes d'État Providence sous le néolibéralisme. Dans la nouvelle situation, le rôle de l'État change à l'Est comme à l'Ouest, et les systèmes de gestion de l'accumulation caractérisés par la déréglementation et la maximisation du profit se poursuivent à travers l'abandon des États-nations aux multinationales et la concentration économique et financière progressive qui a désormais atteint des dimensions planétaires. Cette stratégie du capital a pour corollaire nécessaire l'appauvrissement progressif et inéluctable du quart-monde, la dégradation des conditions de vie et de travail des habitants des pays riches eux-mêmes, la disparition de la sécurité sociale, la barbarisation des relations intersubjectives avec la poussée toujours plus forte de l'individualisme et de la satisfaction compétitive de ses propres besoins par rapport à ceux des autres. Bref, c'est ce phénomène que l'on appelle communément la logique de la privatisation.

\section*{Choix consociatifs}

L'une des formes de défense partielle contre ce processus de transformation adoptée par les groupes les plus forts est la création d'agrégats sociaux par segments sociaux. Il s'agit de groupes d'individus unis par de forts motifs d'identification - position sociale, recensement, religion, race, etc. - qui se mobilisent pour défendre les intérêts collectifs communs à l'ensemble auquel ils appartiennent. Une telle société peut trouver des règles de coexistence et une répartition équilibrée des ressources, mais ce n'est certainement pas celle que nous voulons.

Cependant, la crise de la structure de l'État Providence incite certains d'entre nous à émettre l'hypothèse de la création de structures et de services autogérés qui reflètent notre orientation culturelle et répondent à nos besoins. Si dans le passé - rappelons l'expérience de la Colonie Cecilia ou de l'École moderne de Ferrer - de telles expériences étaient acceptables soit comme des expériences immatures (c'est le cas du premier exemple), soit comme des instruments de lutte, aujourd'hui une telle hypothèse est tout à fait réabsorbable, voire facilitée et nourrie parce qu'elle est parfaitement organique à la logique consociative.

\section*{Les luttes pour le renforcement des services publics}

En tant que communistes anarchistes, nous pouvons et devons lutter pour la liquidation de l'État, mais cela ne signifie pas qu'il faille lutter pour que le coût et la responsabilité de la fourniture de certains services n'incombent pas à la structure sociale. Bien sûr, pour nous, les organes de gestion politique et administrative changent – nous ne soutiendrons jamais le système de santé comme il existe – mais nous devrons certainement penser à un service de santé qui offre assistance et aide à tous les citoyens et dont les coûts incombent à la communauté. Il en va de même pour les écoles, le nettoyage des rues, la distribution de l'eau et les services de transport que l'on appelle communément les services publics. Le problème n'est donc pas celui de la nature publique ou privée de certains services, mais celui de l'organe ou des organes politiques qui gèrent et géreront l'entreprise, de la composition des organes administratifs, même s'ils sont techniquement nécessaires, et de la manière dont les contrôles politiques sont exercés sur eux par la collectivité.

\chapter{Introduction}

Un des piliers de l'anarchisme historique est sans aucun doute l'anti-étatisme. Sans vouloir aller jusqu'aux excès de ceux qui nient même l'État Providence à cause de la présence de ce terrible petit mot et tombent dans les bras du libéralisme le plus féroce, même chez nous, trop souvent, la conception de la nécessité d'une société sans État conduit à des distorsions, qui proviennent, à notre avis, d'une prise en charge hâtive du bagage historique de l'anarchisme. Ce bagage doit en fait être contextualisé et analysé en profondeur, à une époque où le capitalisme rampant prône la dissolution de l'État en tant qu'appareil administratif bureaucratique qui perçoit des impôts et fournit des services.

\mainmatter
\chapter{La naissance de l'État et ce qui l'a précédé}

Un peu d'histoire ne nuit à personne ! Le moloch d'État est né, dans sa configuration moderne, il y a plus de deux siècles. Cela coïncide étroitement avec l'émergence de la classe bourgeoise comme nouvelle classe dirigeante. Ce n'est pas un hasard si la plupart des fonctions typiques de l'État moderne prennent forme dans la France de la révolution de 1789. Il est logique de s'interroger sur les raisons profondes de cette transformation des structures de pouvoir dans la société, sur les relations sociales qui ont cessé d'exister pour faire place à d'autres, sur les changements que cela a entraînés dans les relations de classe et, surtout, sur la manière dont la domination de la classe bourgeoise émergente s'est articulée.

\section{Les relations sociales dans l'organisation féodale}

Lorsque certains anarchistes dénoncent à juste titre les échecs que l'État, en tant qu'organisation bourgeoise de la société, provoque dans les classes subalternes, ils s'abstraient trop superficiellement de la situation que ces classes ont connue avant la naissance de l'État libéral. L'absence totale de règles permettait aux détenteurs du pouvoir n'importe quel arbitraire au détriment des subalternes : la lecture des Fiancés est, nous le croyons, une expérience commune à tous. À la réflexion, il apparaît clairement qu'il s'agit là, après tout, de la véritable essence du pouvoir absolu.

Les pays pauvres étaient non seulement très pauvres (ils le sont toujours), mais ils fournissaient de la main-d'œuvre sous la forme extrême de l'esclavage.

Le concept des droits n'existait même pas. Dans l'Antiquité, celle-ci ne s'appliquait qu'aux citoyens libres de la cité-État, mais dans la dégénérescence féodale, elle était encore plus restreinte aux membres de l'aristocratie et du haut clergé. La grande majorité de la population vivait dans une situation de déni total de la dignité humaine.

\section{L'État libéral et le droit}

\emph{Liberté, fraternité, égalité,} telle est la devise fondatrice de l'État libéral moderne. Inutile de répéter, entre nous, l'hypocrisie qu'elle recèle. Ce qui est intéressant, c'est une autre considération. Le passage d'une organisation sociale dépourvue de règles (seule celle du plus fort) à une organisation qui se prétend fondée sur des règles de base au-dessus de tout individu est loin d'être sans intérêt. Le principe, même s'il tend toujours à être ignoré, est là et il produit ses effets, malgré l'arrogance du pouvoir.

Pour donner un exemple, l'organisation des travailleurs serait inconcevable dans la société féodale ; attention, il s'agit d'une organisation des travailleurs et non d'une révolte. En effet, des révolutions même sanglantes (et même réussies) étaient possibles avant la révolution bourgeoise, mais la conquête progressive de portions croissantes de richesses ne l'était pas. Il est évident que ces conquêtes sont partielles, souvent temporaires car (comme nous le constatons aujourd'hui) elles sont résorbables, et que la seule étape qui compte est l'étape révolutionnaire. Cela n'enlève rien à deux choses : d'une part, comme le disait Malatesta, la gymnastique de la lutte est une gymnastique pour la révolution, d'autant plus nécessaire pour ceux qui, comme nous, croient en une révolution consciente et conscientisée, non réabsorbable par les prétentions d'une nouvelle classe dominante en vertu de son savoir. Et d'autre part, le fait que tout ce qui améliore la vie des gens aujourd'hui n'est pas méprisable simplement parce que ce n'est pas le communisme libertaire en soi.

La société libérale, en se couvrant du voile des droits humains, voile nécessaire à sa lutte contre les anciennes classes dominantes, sanctionne un principe qui est progressif, en fait et en résultat, même pour les classes qui restent subordonnées.

\section{Participation progressive}

« \emph{Le solidarisme kropotkinien, développé sur le terrain naturaliste et ethnographique, confondait l'harmonie de la nécessité biologique des abeilles avec la discordia concors et la concordia discors propres à l'agrégat social, et avait trop (sic !) de formes primitives de sociétés-associations présentes pour comprendre l'ubi societas ibi jus, “ là où il y a une société, il y a un droit ”, inhérent à des formes politiques qui ne sont pas préhistoriques ».}

Cette citation nous fournit deux bases de réflexion utiles.

La première est qu'il n'y a pas de société possible sans règles : on peut et on doit discuter, et les anarchistes le font, de la manière dont elles devraient être formulées, de qui a le pouvoir de les établir, de la manière dont elles devraient être universellement partagées, etc. En l'absence de règles, il n'y a pas d'anarchie, mais une jungle qui pénalise toujours les plus faibles et profite aux plus forts.

La seconde est que les règles « partagées » auront une double valeur : lier la liberté individuelle d'une part, et garantir la justice sociale et la protection pour tous d'autre part.

\chapter{L'État du 19\ieme{} siècle et la naissance de la théorie anarchiste}

Le point de départ de la réflexion anarchiste sur le rôle de l'État avant, pendant et après la révolution sociale est sans aucun doute Bakounine. Il faut dire tout de suite que pour comprendre le rôle de l'État moderne et les moyens de le dépasser, l'approche bakouniniste n'est pas d'un grand secours, parce qu'elle est trop liée aux nécessités de sa lutte et son contexte. Malheureusement, certaines de ses affirmations, prises hors contexte et sans aucun effort d'interprétation, ont été rendues incontestables et pour des principes inflexibles et immuables de l'anarchisme. Pour sortir d'une supposition superficielle de mots d'ordre qui finissent par fausser toute entreprise politique, il est nécessaire de faire quelques rappels.

L'élaboration de Bakounine s'est faite dans la dernière décennie de sa vie, au milieu de ses actions au sein de l'Association internationale des travailleurs et de sa polémique avec la composante marxiste ; en outre, les principales références, étroitement liées au développement de l'action révolutionnaire du groupe anti-autoritaire, étaient l'Italie, l'Espagne, la Russie et l'Autriche, auxquelles il faut ajouter l'empire allemand, à la fois en raison de son rôle émergent en tant que première puissance continentale européenne et parce que le noyau fort des antagonistes sociaux-démocrates s'y trouvait.

Dans ce cadre, les préoccupations immédiates de Bakounine sont au nombre de trois :

\begin{itemize}
\item établir définitivement que la conquête de l'État (par les élections) ou sa transformation par les réformes ne sont pas des voies viables vers la société égalitaire et solidaire ;
\item démontrer que là où il y a une forme de pouvoir, il y a toujours une forme d'exploitation et qu'il n'y a donc pas d'organisation sociale meilleure qu'une autre, si ce n'est la société sans propriété, sans classes et sans hiérarchie ;
\item enfin, conséquence logique, cette organisation étatique ne peut et ne doit pas survivre à la révolution sociale.
\end{itemize}

Ces points restent incontestablement les traits distinctifs et fondateurs de toute conception anarchiste.

Dans l'urgence de fixer les coordonnées ci-dessus, Bakounine, convaincu de l'imminence du soulèvement révolutionnaire des masses grâce au développement irrésistible de l'Internationale, n'a pas trouvé le temps ni l'espace de réflexion nécessaires à une analyse approfondie du rôle que l'État assumait déjà depuis trois quarts de siècle, dans un processus lent, contradictoire, souvent difficile à cerner, mais certain et en quelque sorte irréversible. Pour lui, l'État est essentiellement l'État allemand ou le tsarisme autocratique russe. À tel point qu'il ne considère même pas l'État anglais comme un véritable État, puisqu'il ne correspond pas aux critères qu'il estime distinctifs de l'État moderne, à savoir : la centralisation militaire, policière et bureaucratique. On voit bien la distorsion qu'implique, d'un point de vue théorique, l'échange des organisations étatiques, ou mieux centralisées, résiduelles du passé, avec l'État moderne que l'on identifie précisément en Grande-Bretagne et dans l'État français en pleine transformation, même si c'est avec l'héritage historique d'une centralisation pluriséculaire.

En effet, le moloch étatique est entré dans la théorie anarchiste précisément à partir de cette conception de la centralisation militaire, policière et bureaucratique, terreau de toutes les déformations futures et de l'incapacité d'adapter l'analyse. Chaque évolution de l'État a reçu l'interprétation d'un approfondissement de ces centralisations, ce qui a empêché le discernement de nouvelles fonctions, pas toujours négatives, et conduit aujourd'hui beaucoup d'anarchistes à un démantèlement théorique face à des formes de décentralisation et de dissolution apparente même de l'appareil oppressif.

Bakounine avait également averti que le non-État anglais (décentralisé) n'était pas moins dangereux pour cela, bien que sa polémique, nécessaire pour que l'urgence de la révolution soit correctement finalisée et pour balayer les illusions pernicieuses, ait eu tendance à assimiler différentes formes de régime bourgeois, sans savourer les différences même pour les besoins des conditions de vie matérielles des masses ; en effet, à certains moments, l'illusion démocratique a été considérée comme encore plus négative pour le développement de la conscience révolutionnaire du peuple.

Cependant, Bakounine ne semble pas toujours indifférent aux règles de la société dans laquelle se déroule la lutte révolutionnaire, ce qui prouve que cet aspect n'est resté que peu développé dans sa pensée.

\chapter{L'évolution de l'État}

Bien qu'au milieu du siècle dernier l'évolution de l'organisme étatique ait déjà pris des proportions repérables (mais qui ont échappé non seulement à Bakounine pour les raisons mentionnées plus haut, mais aussi à Marx), les connotations qu'il allait prendre étaient en effet difficiles à prévoir. Il y a deux considérations qu'il est intéressant de développer : d'une part, l’amalgame des compétences qu'il en est venu à assumer et l'évaluation de leurs retombées dans l'organisation sociale dans son ensemble ; d'autre part, si l’existence de l'État n’a que eu des impacts négatifs sur le « progrès » du genre humain, il doit être considéré comme une parenthèse de la tendance humaine originelle à la solidarité mutuelle. Il est évident que la réponse à ces deux questions est loin d'être négligeable pour l'évaluation des luttes d'aujourd'hui, bien qu'elle puisse difficilement constituer, comme nous le verrons, un changement de perspective pour la réalisation d'une société sans classes et, d'ailleurs, sans État.

\section{L'État entrepreneur}

Lorsque nous parlons de l'État moderne, nous avons tendance à confondre trois fonctions que l'appareil d'État lui-même exerce, mais qui sont profondément différentes les unes des autres et ne sont pas du tout nécessaires l'une à l'autre : la régulation du cours du cycle économique, l'intervention directe dans l'économie des entreprises et l'aide sociale. Ces trois fonctions se sont ajoutées au cours de ce siècle, se superposant au rôle traditionnel de gendarme des intérêts bourgeois, bien connu des révolutionnaires du 19\ieme{} siècle.

Les théoriciens de l'avènement de la technobureaucratie ont vu dans cette multiplication des pouvoir privilégiées la confirmation de leurs prédictions d'une incorporation totale de la société dans ce monstre omnivore que serait l'État.

En parfaite continuité avec le déterminisme de Kropotkine, pour eux l'histoire est à sens unique et les voies de l'évolution sociale sont déjà tracées, de sorte que les tendances qui se manifestent entre les années 1930 et 1970 indiqueraient sans équivoque les débouchés futurs : leur vision téléologique n'est que l'envers de la vision marxiste, puisque toutes deux manquent de la connaissance de la fonctionnalité de l'organisation sociale aux intérêts contingents du capital et par conséquent de la réversibilité de choix qui leur paraissent au contraire définitifs. Ce n'est donc pas un hasard si la désintégration de l'appareil d'État, qui a commencé à se manifester au cours des deux dernières décennies, les trouve théoriquement désarticulés et bredouillants dans leurs propositions, sinon résolument et irrémédiablement cohérents avec les mouvements en cours dans les hautes sphères de l'économie mondiale.

\subsection{Contrôler le cycle}

L'impossibilité de prévenir des crises cycliques de plus en plus dévastatrices, après l'échec des théories marginalistes visant à interpréter scientifiquement les tendances du marché, a conduit le capital à une mutation drastique de ses caractéristiques. Au cours des années allant du début de la quatrième décennie à la fin de la septième, l'État, de simple gendarme des intérêts capitalistes (drainage fiscal, contrôle policier, politique douanière, etc.), est devenu le moteur de l'économie, se chargeant, par une augmentation substantielle de la pression fiscale et l’initiative de travaux publics grandioses, de relancer le cycle économique précipité vers l'abîme de la crise.

Cette nouvelle approche économique (keynésianisme) a eu pour conséquence nécessaire l'expansion du marché, condition indispensable à l'absorption d'une quantité toujours croissante de biens, selon un cycle perpétuellement progressif. Les salaires deviennent le volant de l'économie (fordisme) et augmentent mais en dessous de la productivité, sous l'effet de l'innovation technologique dans l'organisation du travail (taylorisme). L'objectif est de réduire la lutte des classes à un outil permanent de rationalisation du système.

Il est clair que le capitalisme invente une nouvelle ère pour sa propre prospérité, mais en même temps des masses croissantes du prolétariat métropolitain dans les pays industrialisés accèdent à la consommation de biens qui leur étaient auparavant inaccessibles. La saison des luttes de la fin des années 60 a clairement montré que cette circonstance ne se traduisait pas par une intégration définitive des classes subalternes dans la logique de l'entreprise ; au contraire, c'est précisément à partir des secteurs les plus identifiables comme représentants des soi-disant masses ouvrières que la contestation systémique a commencé et a continué à s'articuler autour d'eux.

\subsection{Gestion directe des capitaux}

Une nouvelle étape a été franchie dans les années 1930. L'évolution se fait presque naturellement, mais elle n'est pas nécessaire ; à tel point qu'elle ne se produit pas dans le système capitaliste central : les États-Unis. Une lecture superficielle pourrait assimiler ce qui se passe dans les deux mondes antagonistes de l'économie planifiée globale (zone soviétique) et de l'économie planifiée directionnelle (Europe capitaliste). Mais, comme nous le verrons, les deux cas d'école présentent des caractéristiques qui ne les rendent pas assimilables.

Le premier stimulus est apparu presque par hasard dans l'Italie fasciste : face à la crise de nombreux complexes industriels, le régime a créé (1933) l'Institut pour la reconstruction industrielle (IRI), qui a repris les entreprises dites en déclin et devait les remettre sur le marché une fois qu'elles se seraient rétablies. Au contraire, au bout d'un certain temps, l'Institut s'est retrouvé en possession d'une partie considérable des forces de production industrielle et a fini par les gérer lui-même, donnant naissance au secteur des holdings d'État. L'IRI a survécu au fascisme et est devenu, après la Seconde Guerre mondiale, un acteur majeur de la vie économique nationale. Son succès à aplanir les aspérités du cycle économique, grâce à l'énorme disponibilité de capitaux, y compris, mais pas seulement, de capitaux d'État, est tel que, dans les années 1950, les travaillistes britanniques en viennent à étudier son fonctionnement pour le reproduire en Grande-Bretagne, imités par les Français et les Allemands. C'est ainsi qu'est né l'État qui participe directement à la vie économique avec son propre capital, l'État entrepreneur.

Il en va tout autrement de l'économie soviétique, où la gestion étatique de l'économie est globale et ne s'inscrit pas dans un régime concurrentiel, en réponse à l'arrivée au pouvoir d'une classe autre que la bourgeoisie entrepreneuriale : la petite bourgeoisie éduquée, avec ses propres mécanismes d'extraction du produit excédentaire. Il en résulte deux types différents de planification économique, qui ne se ressemblent que par leur nom.

Il n'est pas possible d'éviter, à ce stade, un jugement rapide sur le nouveau rôle que l'État a assumé, en continuité mais non en conséquence avec le rôle déjà examiné de régulateur et de stimulateur du cycle économique. Ceux qui ont connu les luttes syndicales des années 1960 et 1970 se souviennent certainement que deux contrats distincts étaient alors signés pour les travailleurs employés par les entreprises privées et pour ceux employés par les entreprises publiques : les seconds anticipaient souvent sur les premiers, jouant le rôle de précurseurs et les obligeant, par analogie, à faire des concessions que les patrons privés n'acceptaient pas volontiers de faire. A l'ère du libéralisme galopant, les participations de l'État sont devenues synonymes de gaspillage clientéliste, et sur cette vague émotionnelle se sont démantelées, vendant leur patrimoine instrumental à des particuliers. C'est ainsi qu'une entreprise modèle comme le Nuovo Pignone de Florence, après avoir été rachetée par AGIP (de l'IRI), après avoir été reconvertie à de nouveaux types de production, après avoir développé une technologie de pointe, après avoir conquis des parts très importantes du marché mondial du secteur et être devenue une source de profits substantiels pour l'État, a été vendue au concurrent américain General Electric.

Il ne fait aucun doute qu'une classe de gestionnaires publics s'est enrichie grâce à la gestion des entreprises publiques, mais il ne fait aucun doute non plus que les salaires privilégiés et le cadre réglementaire des travailleurs de ces entreprises ont servi de point de référence pour les autres travailleurs en poussant à la hausse les exigences de tous. Il est donc légitime de douter que la ferveur pour la destruction du secteur de la participation publique découle davantage de la nécessité pour l'entreprise privée d'éliminer un concurrent gênant que d'un vague besoin de moralisation peu crédible.

En revanche, l'élimination physique d'Enrico Mattei, président de l'AGIP et partisan d'une politique autonome d'approvisionnement en pétrole brut qui couperait les ponts avec le cartel pétrolier international (les sept sœurs), par les compagnies pétrolières elles-mêmes, est plus qu'une piste de réflexion.

\subsection{Bien-être}

L'État, au cours du siècle dernier, a progressivement assumé le rôle de fournisseur de services sociaux (éducation, santé, bien-être, transports, etc.). L'avantage pour les patrons est évident : ils se déchargent sur la fiscalité générale (à laquelle ils contribuent relativement moins que les salariés) de la préparation, de la récupération, d'une timide forme de sécurité et de la mobilité de la main-d'œuvre, ce qui se traduit par une meilleure qualité des performances professionnelles et, on l'espère, par une diminution des conflits sociaux. Cela n'empêche pas que, même pour les travailleurs, tout cela ne se traduit pas par un avantage indéniable, notamment parce que l'alternative n'est pas une baisse de la charge fiscale, sur laquelle il conviendra de revenir, mais l'abandon des formes de protection de la vie associées à la jungle du profit, comme nous le constatons avec une clarté absolue.

A tel point qu'à une certaine époque, l'aide sociale portait le nom de salaire social et était considérée par les associations de travailleurs comme une forme de rémunération de leur travail. Il faut donc considérer que si l'enseignement public était contraint à l'acquisition d'un métier, d'un autre point de vue, il constituait un contact avec l'acquisition d'instruments culturels et critiques, auparavant totalement interdits aux classes subalternes ; si les soins de santé ne tendaient qu'à restaurer la force de travail endommagée, d'un autre point de vue, ils garantissaient la guérison des maladies qui fauchaient auparavant le prolétariat ; Si les régimes de retraite tendent souvent à répercuter sur la société les coûts d'une main-d'œuvre licenciée ou obsolète, sous un autre angle, ils offrent une alternative à l'enfermement dans des maisons de repos et à la dégradation totale de la vieillesse à laquelle étaient soumis les travailleurs subalternes ; si le système de transport public permet la marginalisation de la main-d'œuvre massivement urbanisée dans des banlieues aliénantes, sous un autre angle, il garantit également une meilleure jouissance du temps libre à des couches de la population qui en étaient autrefois exclues.

Refuser d'examiner la réalité complexe de l'État et de toutes ses formes, c'est tout simplement faire preuve de manque de lucidité.

Ainsi, si l'État est l'ennemi, tout ce qui vient de lui doit être rejeté, quel que soit l'autre ennemi, le capitalisme, qui vise aujourd'hui précisément à la destruction de l'État. Mais il en est une autre, plus insidieuse mais non moins erronée. Le prolétariat et le capital étant antagonistes dans leurs intérêts, tout ce qui profite au second ne peut être qu'un désavantage pour le premier. S'il en était ainsi, puisqu'il est indéniable que les salaires sont le moins que les patrons aient à donner pour obtenir la pleine exploitation de la force de travail et qu'ils sont en eux-mêmes un avantage pour les employeurs, ils devraient être rejetés par les salariés. En effet, de même que l'on lutte (ou plutôt il serait souhaitable que l'on lutte) pour améliorer la part des marchandises en faveur des salaires et contre celle du profit, de même on devrait s'efforcer de tourner les services de plus en plus dans le sens utile aux classes exploitées et de moins en moins en faveur des classes aisées. Sans que cela signifie, bien entendu, que l'on puisse renoncer au bouleversement révolutionnaire pour parvenir à une société juste, libre et égalitaire.

\section{De l'État primitif à l'État moderne}

Il résulte des remarques sommaires qui précèdent qu'en un siècle et demi (et comment pourrait-il en être autrement ?) l'État a substantiellement modifié son rôle, son fonctionnement, sa structure. Si, d'une part, le marxisme, en séparant le rôle du gouvernement (le comité d'entreprise de la bourgeoisie, selon l'aphorisme bien connu de Marx) de celui de l'État en tant qu'appareil, a fini par émettre l'hypothèse de l'utilisation à des fins révolutionnaires de la machine étatique, soumise à une nouvelle gestion, une partie de l'anarchisme, en identifiant les deux fonctions, a prétendu perdre, au fil du temps, la capacité de distinction et, par conséquent, celle de l'orientation politique.

Il est donc nécessaire de reconsidérer l'ensemble de la question si l'on veut échapper à l'emprise de l'acceptation de l'appareil d'État tel qu'il est ou à la négation a priori de tout ce qui en découle, ce qui nous conduirait tout autant dans les bras du néolibéralisme le plus agressif.

\chapter{L'ambiguïté du rôle de l'État}

Si l'on fait abstraction de l'État absolutiste ou théocratique, pure expression du pouvoir d'une caste privilégiée (contre laquelle s'exerçait la critique de Bakounine, comme nous l'avons vu), encore en vigueur dans de nombreux pays au milieu du XIXe siècle, mais en tant que phénomène résiduel, notre attention doit se porter sur l'État libéral, désormais solidement implanté dans tout le monde du haut développement capitaliste (et dont on sait qu'il représente un moindre mal dans les pays tiers encore opprimés par des dictatures féroces).

Les droits bourgeois sont, il est vrai, des fictions ; l'État n'est jamais impartial ; dans une société divisée en classes, différentes classes vivent et pratiquent même l'anarchie avec des conséquences tout à fait différentes en termes de vie et de punition. Pourtant, l'aphorisme bien connu de l'eau sale et du bébé doit être pris en compte, même si l'eau est grande et le bébé vraiment petit, et ce pour deux bonnes raisons. La première est qu'il serait de toute façon stupide de sacrifier l'enfant ; la seconde est que nous aiderions l'ennemi de classe qui vise précisément à conserver l'eau sale en éliminant l'enfant, qui serait le premier à disparaître.

\section{L'État dans la révolution}

Le point sur lequel les anarchistes se sont toujours opposés aux marxistes a été celui de la nécessité ou non de la survie de l'État dans la période de transition : centralisation des fonctions pour propager et défendre les résultats révolutionnaires pour les adeptes du socialisme dit scientifique ; décentralisation et prise en charge par le prolétariat de la gestion sociale pour que le prolétariat s'approprie immédiatement l'événement révolutionnaire comme solution aux problèmes générés par la société divisée en classes, pour les communistes anarchistes.

Les marxistes ont qualifié la position des anarchistes de corporatisme, arguant que suivre leur méthode créerait des conflits et des inégalités et que personne ne serait en mesure de contrer efficacement l'inévitable réaction de la bourgeoisie. Les anarchistes, quant à eux, soutenaient que la survie d'un pouvoir centralisé (l'État) régénérerait une classe expropriatrice et détournerait les masses de la révolution. L'expérience a donné raison à ces derniers sans équivoque, notamment parce que des exemples admirables de solidarité entre les dépossédés se sont toujours produits là où l'autogestion révolutionnaire du prolétariat disposait de quelques timides espaces de libre expression.

Cela dit, venons-en au fond. Tout d'abord, dans leur critique vertueuse, certains anarchistes se sont engagés sur une pente glissant qui pourrait s'avérer dangereux s'il n'était pas suffisamment étudié : la solidarité est un projet de civilisation auquel l'homme doit être éduqué, et ce n'est pas un hasard si les exemples cités plus haut se sont tous produits là où les militants révolutionnaires avaient exercé le plus longtemps et le plus efficacement leur influence, et donc là où les masses étaient les plus préparées à la révolution. En d'autres termes, il serait pernicieux de confondre l'anarchie, qui est la condition finale de l'évolution de l'homme (résultat d'une croissance de la civilisation, de la prise de conscience de son rôle social et de sa sensibilité), avec le comportement primordial de l'homme animal, qui est violent, grossier et agressif (féral).

Deuxièmement, il faut éviter le glissement de contenu : l'administration des affaires publiques ne doit pas être centralisée. Au contraire, c’est les services sociaux qui doivent conserver un rôle centralisé (sur la base d'un libre accord ascendant, bien entendu), afin de garantir les mêmes droits à tous, quelle que soit leur situation géographique.

Les anarchistes espagnols de 1936 n'en doutaient pas, et sachant que la révolution ne marche que si dès le premier jour (dans la mesure du possible) tout fonctionne, de l'approvisionnement aux services, ils ont organisé les travailleurs des services publics (par exemple les transports à Barcelone) pour les rendre utilisables. Il s'ensuit que s'il est juste de démolir et de ne pas changer l'appareil d'État bourgeois (comme on disait autrefois), cela ne doit pas concerner la fourniture de services sociaux : apprentissage des enfants, protection des personnes âgées, soins aux malades, transport des citoyens, etc. Il semble également évident de déduire que là où ces services fonctionnent déjà sur la base de normes valables pour tous et sont fournis au citoyen en tant que tel, la transition des travailleurs du secteur vers une gestion collectivisée et uniforme est plus facile et plus efficace que là où les mêmes services sont éclatés dans des mains privées et soumis à la logique du profit.

\section{Le premier ennemi}

Les marxistes ont toujours soutenu que toute l'évolution historique est déterminée par la structure (la structure de production, avec les relations sociales qui y sont associées), tandis que les autres aspects (politique, culture, guerre, etc.) n'en sont que des conséquences plus ou moins directes, mais néanmoins nécessairement déterminées (superstructure).

Les anarchistes, au contraire, pensaient que oui, la structure était la source première de l'ordre social (l'histoire est l'histoire de la lutte des classes), mais que la superstructure n'en était pas si strictement dépendante, c'est-à-dire qu'elle possédait ses propres marges de vitalité et pouvait même à son tour interagir, en contribuant à la déterminer, avec la structure elle-même. {[}Curieusement, notons-le au passage, les marxistes ont développé un très grand intérêt pour les médiations politiques et électorales (les formes naissantes de l'économie, comme les appelait Marx), tandis que les anarchistes ont cultivé un désintérêt fanatique à leur égard{]}.

S'agissant de l'État, les marxistes en ont tiré la conséquence qu'une fois les rapports de production (structures de propriété) modifiés par la révolution, la superstructure étatique devait en suivre les impératifs jusqu'à disparaître par consommation de sa fonction (les trotskistes, partant de cet axiome même, parlaient d'un État prolétarien dégénéré pour l'URSS, n'admettant pas le renversement complet des objectifs révolutionnaires par le nouvel appareil bureaucratique soviétique). Les anarchistes, convaincus que le pouvoir pouvait à son tour régénérer l'exploitation, initialement abolie (ce qui s'est évidemment produit), préconisaient l'abolition immédiate de l'appareil d'État, remplacé par des formes alternatives d'associationnisme coopératif.

Là encore, le principe était bon, mais au fil du temps et de la mauvaise propagande, il a été corrompu au point de devenir dangereux, voire très dangereux. Oubliant que l'ennemi principal est l'exploitation de l'homme par l'homme (comme le savait bien Bakounine) et que l'État est une des formes historiques de sa manifestation, ni unique ni nécessaire, ils ont confondu la théorie de la phase transitoire avec la théorie de l'histoire et ont proclamé l'État comme premier ennemi du prolétariat (quand ce n'est pas le seul). Ils ont opposé à la « statolâtrie » marxiste une « statophobie » non moins obtuse. En d'autres termes, ils ont centralisé leur critique sur l'instrument de domination du capital dans une phase spécifique, négligeant la domination elle-même et ses autres formes d'existence possibles, uniquement par crainte que dans la phase révolutionnaire, l'État survive et reproduise l'exploitation.

C'est pour cette raison que, dans de nombreux écrits anarchistes, on affirme que l'État est le premier ennemi et que ceux qui affirment que le premier ennemi est la classe bourgeoise sont accusés de cryptomarxisme ; il est dommage qu'à présent les patrons eux-mêmes visent à la dissolution de l'État, tel qu'il était connu au XXe siècle, et que dans certaines franges extrêmes du néolibéralisme états-unien (Friedman Jr.), ils envisagent de privatiser même les forces de police, revenant ainsi aux Bravi de la mémoire de Manzoni, ou à toutes ces formes de police privée (et/ou de travail criminel) utilisées pour la répression sous différentes formes et à différents stades par presque tous les États du monde.

Rappelons au passage que les mafias du monde entier naissent ou résistent précisément comme une forme de contrôle social et policier, là où - les rapports d'exploitation n'ayant pas été abolis - les formes étatiques, incapables de garantir même par la force à la bourgeoisie le plein contrôle du territoire, se voient contraintes de le partager avec les pouvoirs forts des mafias, en les absorbant ou en se laissant imprégner par eux à tous les niveaux institutionnels.

\section{Fonctions collectives et coercitives}

En conclusion, une approche généralisante ne nous fait pas avancer d'un pas (mais reculer de beaucoup). Il est donc nécessaire de distinguer les différentes fonctions que l'État moderne remplit (ou plutôt qu'il remplissait avant le récent assaut néolibéral) : les fonctions de maintien de l'ordre social existant, tant à l'intérieur d'une région qu'au niveau international (la \emph{guerre}, comme on l'a appelée), des fonctions de fourniture d'un niveau minimum de sécurité aux citoyens (l'\emph{aide sociale}, précisément). Les premières sont purement coercitives et n'ont pas de raison d'être dans une société égalitaire, les secondes visent à une intégration sociale douce et jouent un rôle que toute société qui veut s'appeler telle doit assumer, même si c'est sous une forme variée.

Les tendances actuelles indiquent une voie très différente de la voie souhaitable, une voie que le capitalisme a empruntée avec beaucoup de zèle. L'élimination du bien-être et le maintien ou plutôt le renforcement de la guerre. Les traités de l'Union européenne, le renforcement de l'OTAN, l'élargissement de l'armée professionnelle en Italie et dans d'autres pays vont dans ce sens, ce qui exclut, entre autres, une réduction conséquente de la charge fiscale, au moins pour les salariés.

On peut d'ailleurs ajouter que le développement de la protection sociale marque une voie dont le renversement ne fait que le jeu de l'adversaire de classe, et qui prépare, plus qu'elle n'éloigne l'homme, en tant qu'animal, social, à une gestion collective et solidaire des relations. Il semble cependant que pour certains anarchistes autoproclamés, le mal soit la santé publique, l'éducation publique, l'assistance publique, dans la mesure où elles sont assurées par des organismes d'État, et non l'exploitation de la maladie, du savoir et de la vieillesse à des fins de profit.

Et n'oublions pas que si l'État est un obstacle à toute réalisation révolutionnaire, et qu'il doit disparaître dès le premier instant de tout renversement des rapports de force entre la bourgeoisie et le prolétariat, même son apparition historique représente un progrès par rapport à l'arbitraire barbare qui l'a précédé, et que sa disparition, sans renversement des rapports de propriété actuels, éloigne plus qu'elle ne rapproche du but.

\chapter{À propos des règles}

L'anti-étatisme anarchiste a sans doute le mérite d'avoir historiquement porté l'attention sur des aspects que le marxisme a résolument négligés : le rôle du pouvoir politique, le rôle des institutions pendant et après l'événement révolutionnaire, le rôle des classes intellectuelles, la logique interne de l'administration et sa capacité d'auto-reproduction, l'autonomie évolutive de la superstructure dans certaines conditions et son influence sur l'évolution générale. Dans tous ces domaines, les acquisitions sont théoriquement irréversibles et prouvées par l'expérience des diverses tentatives de construction du socialisme autour des paramètres des formes les plus variées du marxisme.

Il est cependant nécessaire de nettoyer l'antiétatisme des débris qu'il traîne derrière lui en raison de l'accumulation d'interprétations trop souvent superficielles et fondées sur de simples assonances nominales. En particulier, la confusion pernicieuse entre État et public, entre bureaucratie et services, entre top-down et collectif. Il est bien vrai que les services publics souffrent de la bureaucratisation et d'une faible perméabilité aux besoins des individus qui sont censés les utiliser. Mais il est tout aussi vrai que les polémiques que les médias de pouvoir déposent quotidiennement sur ces inefficacités sur les tables des télé-utilisateurs au cerveau lavé ne servent qu'à ouvrir la voie au profit privé. Le chemin qui mène des services publics très critiqués d'aujourd'hui à la société égalitaire et sans classe ne traverse pas le territoire infranchissable du capitalisme débridé et de l'intérêt supposé du citoyen individuel ; le chemin est autre et va dans la direction opposée :

\begin{itemize}
\item leur reconnaissance en tant que salaire d'égalisation indirect ;
\item la demande de services plus étendus, plus efficaces et gratuits pour tous ;
\item un contrôle de plus en plus efficace par la communauté, non compris sous la forme de ses représentants politiques, sur la qualité de leur prestation.
\end{itemize}

C'est ainsi que l'on peut préparer la voie à une autogestion efficace de la société et à des services qui comblent les inégalités que la nature crée entre les êtres humains, ce qui est le sens véritable et le plus profond du service public.

\chapter{Pour la liquidation de l'État}

Avant d'aborder le problème de la période de transition, il est nécessaire que l'organisation politique des communistes anarchistes clarifie brièvement, non seulement sur le plan terminologique, mais fondamentalement sur le plan stratégique, les différentes conceptions qui préfigurent la fin de l'État bourgeois, suite à la rupture politico-institutionnelle provoquée par une révolution victorieuse du prolétariat.

Nous dépassons les concepts « d'abolition de l'État » ou de « destruction de l'État » car ils préfigurent deux aspects liés à la fin de l'État, centrés sur l'action violente d'un groupe de professionnels de la politique et sur l'instantanéité ou la rapidité de cette action.

A l'opposé de ces deux conceptions, nous en trouvons deux autres que nous rejetons également. Il s'agit de la conception de la « décadence de l'État » ou de « l'extinction de l'État ». Nous les dépassons toutes deux dans la mesure où elles préfigurent deux aspects relatifs à la fin de l'État, le premier se référant à un processus entièrement objectif et mécanique qui conduirait à la disparition de l'État, et le second à une sorte de gradualité de ce processus.

Si, dans les deux premiers cas, nous ne voyons pas l'intérêt d'une action minoritaire violente d'un groupe politique contre l'État s'il n'y a pas d'auto-organisation réelle du prolétariat, en même temps, dans les deux autres cas, nous considérons qu'un processus spontané et automatique d'extinction de l'État est impossible sans l'action révolutionnaire de la classe subordonnée qui travaille dans ce sens.

Le choix stratégique de base des communistes anarchistes s'oriente vers la conception de la « liquidation de l'État », comme action politique et économique d'organisation de l'autonomie prolétarienne visant à rendre impossible toute reconstruction de l'État et à lui ôter ainsi toute base sur le plan social.

La liquidation de l'État est donc l'acte final d'un processus qui est déjà né et s'est développé au sein de la société divisée en classes et en opposition absolue à celle-ci, et qui marque la rupture définitive et totale entre le système classiste et autoritaire et la nouvelle société communiste anarchiste.

La liquidation de l'État s'identifie donc à la destruction des structures d'exploitation et des appareils de domination, à la transition de la société divisée en classes à la société communiste anarchiste, réalisant l'objectif révolutionnaire de la destruction des institutions légales, militaires et administratives des rapports de classes sociales pour permettre la mise en oeuvre de méthodes communistes de production, de distribution et de régulation sociale sous le contrôle et l'autogestion des structures prolétariennes fédérées et autogérées de façon libertaire.



\text[Es-tu anarchiste ?]{Es-tu anarchiste ?\\~\\La réponse pourrait te surprendre !}{David Greaber}
Il est probable que tu aies déjà entendu parler des anarchistes et de ce qu'ils sont censés croire. Il y a aussi de fortes chances que tout ce que tu as entendu soit absurde. Beaucoup de gens semblent penser que les anarchistes sont des partisans de la violence, du chaos et de la destruction, qu'ils sont opposés à toute forme d'ordre et d'organisation, ou qu'ils sont des nihilistes fous qui veulent tout faire sauter. En réalité, rien n'est plus éloigné de la vérité. Les anarchistes sont simplement des personnes qui pensent que les êtres humains sont capables de se comporter de manière raisonnable sans y être contraints. C'est une notion très simple. Mais c'est une notion que les riches et les puissants ont toujours trouvée extrêmement dangereuse.

Dans leur plus simple expression, les convictions anarchistes reposent sur deux hypothèses élémentaires. La première est que les êtres humains sont, dans des circonstances ordinaires, à peu près aussi raisonnables et décents qu'ils sont autorisés à l'être, et qu'ils peuvent s'organiser eux-mêmes et organiser leurs communautés sans avoir besoin qu'on leur dise comment faire. La seconde est que le pouvoir corrompt. Par-dessus tout, l'anarchisme consiste simplement à avoir le courage de prendre les principes simples de la décence commune que nous vivons tous et de les suivre jusqu'à leurs conclusions logiques. Aussi étrange que cela puisse paraître, tu es probablement déjà anarchiste à bien des égards, mais tu ne t'en rends pas compte.

Prenons d'abord quelques exemples de la vie quotidienne.

\begin{quotation}
\textbf{S'il y a une file d'attente pour monter dans un bus bondé, attends-tu ton tour et évites-tu de jouer des coudes pour passer devant les autres, même en l'absence de la police ?}
\end{quotation}

Si tu as répondu « oui », tu as l'habitude d'agir comme un anarchiste ! Le principe anarchiste le plus fondamental est l'auto-organisation : l'hypothèse selon laquelle les êtres humains n'ont pas besoin d'être menacés de poursuites judiciaires pour parvenir à des accords raisonnables entre eux ou pour se traiter avec dignité et respect.

Tout le monde pense être capable de se comporter raisonnablement. S'ils pensent que les lois et la police sont nécessaires, c'est uniquement parce qu'ils ne croient pas que les autres le sont. Mais si tu y réfléchis bien, ces personnes ne pensent-elles pas exactement la même chose de vous ? Les anarchistes affirment que presque tous les comportements antisociaux qui nous font penser qu'il est nécessaire d'avoir des armées, des polices, des prisons et des gouvernements pour contrôler nos vies, sont en fait causés par les inégalités et les injustices systématiques que ces armées, ces polices, ces prisons et ces gouvernements rendent possibles. C'est un cercle vicieux. Si les gens sont habitués à être traités comme si leurs opinions ne comptaient pas, ils sont susceptibles de devenir furieux et cyniques, voire violents - ce qui, bien sûr, permet à ceux qui détiennent le pouvoir de dire que leurs opinions ne comptent pas. Une fois qu'ils ont compris que leurs opinions comptent autant que celles des autres, ils tendent à devenir remarquablement compréhensifs. Pour faire court : les anarchistes pensent que c'est en grande partie le pouvoir lui-même et ses effets qui rendent les gens stupides et irresponsables.

\begin{quotation}
\textbf{Es-tu membre d'un club, d'une équipe sportive ou de toute autre association volontaire où les décisions ne sont pas imposées par un dirigeant mais prises sur la base du consentement général ?}
\end{quotation}

Si tu as répondu « oui », alors tu appartiens à une organisation qui travaille selon les principes anarchistes ! Un autre principe anarchiste fondamental est l'association volontaire. Il s'agit simplement d'appliquer les principes démocratiques à la vie ordinaire. La seule différence est que les anarchistes pensent qu'il devrait être possible d'avoir une société dans laquelle tout pourrait être organisé selon ces principes, tous les groupes étant basés sur le libre consentement de leurs membres, et donc que tous les styles d'organisation militaires, comme les armées, les bureaucraties ou les grandes entreprises, basés sur des chaînes de commandement, n'auraient plus lieu d'être. Peut-être tu ne crois pas que cela soit possible. Peut-être que si. Mais chaque fois que tu parviens à un accord par consensus, plutôt que par la menace, chaque fois que tu conclus un arrangement volontaire avec une autre personne, que tu parviens à un accord ou à un compromis en tenant dûment compte de la situation ou des besoins particuliers de l'autre personne, tu es anarchiste - même si tu ne t’en rend pas compte.

L'anarchisme est simplement la façon dont les gens agissent lorsqu'ils sont libres de faire ce qu'ils veulent et lorsqu'ils traitent avec d'autres personnes également libres - et donc conscientes de la responsabilité envers les autres que cela implique. Cela nous amène à un autre point crucial : si les gens peuvent être raisonnables et prévenants lorsqu'ils traitent avec des égaux, la nature humaine est telle qu'on ne peut pas leur faire confiance lorsqu'on leur donne du pouvoir sur les autres. Si l'on donne un tel pouvoir à quelqu'un, il en abusera presque invariablement d'une manière ou d'une autre.

\begin{quotation}
\textbf{Crois-tu que la plupart des hommes politiques sont des porcs égoïstes qui ne se soucient pas vraiment de l'intérêt général ? Penses-tu que nous vivons dans un système économique stupide et injuste ?}
\end{quotation}

Si tu as répondu « oui », tu adhères à la critique anarchiste de la société actuelle, du moins dans ses grandes lignes. Les anarchistes pensent que le pouvoir corrompt et que ceux qui passent leur vie à chercher le pouvoir sont les dernières personnes qui devraient l'avoir. Les anarchistes pensent que notre système économique actuel est plus susceptible de récompenser les gens pour leur comportement égoïste et sans scrupules que pour leur qualité d'êtres humains décents et bienveillants. La plupart des gens sont de cet avis. La seule différence est que la plupart des gens ne pensent pas qu'il y ait quoi que ce soit à faire à ce sujet, ou en tout cas - et c'est ce que les fidèles serviteurs des puissants sont toujours les plus susceptibles d'insister - quoi que ce soit qui ne finisse pas par empirer les choses.

Et si ce n'était pas le cas ?

Et y a-t-il vraiment une raison de le croire ? Lorsqu'on peut les tester, la plupart des prédictions habituelles sur ce qui se passerait sans État ni capitalisme se révèlent totalement fausses. Pendant des milliers d'années, les gens ont vécu sans gouvernement. Aujourd'hui, dans de nombreuses régions du monde, les gens vivent en dehors du contrôle des gouvernements. Ils ne s'entretuent pas tous. La plupart du temps, ils mènent leur vie comme n'importe qui d'autre. Bien sûr, dans une société complexe, urbaine et technologique, tout cela serait plus compliqué : mais la technologie peut aussi rendre tous ces problèmes beaucoup plus faciles à résoudre. En fait, nous n'avons même pas commencé à réfléchir à ce que pourrait être notre vie si la technologie était réellement mise au service des besoins humains. Combien d'heures devrions-nous réellement travailler pour maintenir une société fonctionnelle - c'est-à-dire si nous nous débarrassions de toutes les professions inutiles ou destructrices comme les télévendeurs, les avocats, les gardiens de prison, les analystes financiers, les experts en relations publiques, les bureaucrates et les politiciens, et si nous détournions nos meilleurs esprits scientifiques de l'armement spatial ou des systèmes boursiers pour mécaniser les tâches dangereuses ou ennuyeuses comme l'extraction du charbon ou le nettoyage des toilettes, et si nous répartissions le travail restant entre tous de manière égale ? Cinq heures par jour ? Quatre heures ? Trois heures ? Deux heures ? Personne ne le sait, car personne ne se pose ce genre de question. Les anarchistes pensent que ce sont précisément ces questions que nous devrions poser.

\begin{quotation}
\textbf{Crois-tu vraiment aux choses que tu dis à tes enfants, ou que tes parents t’ont dit?}
\end{quotation}

« Peu importe qui a commencé. » « Une mauvaise action n'en excuse pas une autre » «Répare tes propres bêtises. » « Fais aux autres… » « Ne sois pas méchant avec les gens juste parce qu'ils sont différents. » Peut-être devrions-nous nous demander si nous mentons à nos enfants lorsque nous leur parlons de bien et de mal, ou si nous sommes prêts à prendre nos propres injonctions au sérieux. Car si l'on pousse ces principes moraux jusqu'à leur conclusion logique, on arrive à l'anarchisme.

Prenez le principe selon lequel une mauvaise action n'en excuse pas une autre. Si on le prenait vraiment au sérieux, cela suffirait à faire disparaître presque tout le fondement de la guerre et du système de justice pénale. Il en va de même pour le partage : nous répétons sans cesse aux enfants qu'ils doivent apprendre à partager, à tenir compte des besoins des autres, à s'entraider ; puis nous partons dans le monde réel où nous supposons que tout le monde est naturellement égoïste et compétitif. Mais un anarchiste ferait remarquer qu'en fait, ce que nous disons à nos enfants est juste. Pratiquement toutes les grandes réalisations de l'histoire de l'humanité, toutes les découvertes et tous les accomplissements qui ont amélioré nos vies, ont été fondés sur la coopération et l'entraide ; même aujourd'hui, la plupart d'entre nous dépensent plus d'argent pour leurs amis et leur famille que pour eux-mêmes ; même s'il est probable qu'il y aura toujours des gens compétitifs dans le monde, il n'y a aucune raison pour que la société soit fondée sur l'encouragement d'un tel comportement, et encore moins pour que les gens se fassent concurrence pour les besoins essentiels de la vie. Cela ne sert que les intérêts des gens au pouvoir, qui veulent que nous vivions dans la peur les uns des autres. C'est pourquoi les anarchistes appellent à une société fondée non seulement sur la libre association, mais aussi sur l'entraide. Le fait est que la plupart des enfants grandissent en croyant à la morale anarchiste, puis doivent progressivement se rendre compte que le monde des adultes ne fonctionne pas vraiment de cette manière. C'est pourquoi tant d'entre eux deviennent rebelles, aliénés, voire suicidaires à l'adolescence, et finalement résignés et amers à l'âge adulte ; leur seul réconfort étant souvent la possibilité d'élever leurs propres enfants et de leur faire croire que le monde est juste. Mais si nous pouvions vraiment commencer à construire un monde qui soit au moins fondé sur des principes de justice ? Ne serait-ce pas là le plus beau cadeau que l'on puisse faire à ses enfants ?

\begin{quotation}
\textbf{Crois-tu que les êtres humains sont fondamentalement corrompus et mauvais, ou que certaines catégories de personnes (les femmes, les personnes de couleur, les gens ordinaires qui ne sont pas riches ou très instruits) sont des spécimens inférieurs, destinés à être gouvernés par leurs supérieurs ?}
\end{quotation}

Si tu as répondu « oui », il semble que vous ne soyez pas anarchiste. Mais si tu as répondu « non », il y a de fortes chances que tu adhères déjà à 90 \% des principes anarchistes et que tu vives ta vie en grande partie en accord avec eux. Chaque fois que tu traites un autre être humain avec considération et respect, tu es anarchiste. Chaque fois que tu règles tes différends avec les autres en parvenant à un compromis raisonnable, en écoutant ce que chacun a à dire plutôt qu'en laissant une personne décider pour les autres, tu es anarchiste. Chaque fois que tu as la possibilité de forcer quelqu'un à faire quelque chose, mais que tu décides plutôt de faire appel à son sens de la raison ou de la justice, tu es anarchiste. Il en va de même chaque fois que tu partages quelque chose avec un ami, que tu décides qui va faire la vaisselle ou que tu fais quoi que ce soit dans un souci d'équité.

On pourrait objecter que tout cela est très bien pour permettre à de petits groupes de personnes de s'entendre, mais que la gestion d'une ville, ou d'un pays, est une tout autre affaire. Et bien sûr, il y a une raison à cela. Même si l'on décentralise la société et que l'on confie le plus de pouvoir possible aux petites communautés, il y aura toujours beaucoup de choses à coordonner, qu'il s'agisse de faire fonctionner les chemins de fer ou de décider des orientations de la recherche médicale. Mais ce n'est pas parce qu'une chose est compliquée qu'il n'y a pas moyen de la faire démocratiquement. Ce serait simplement compliqué. En fait, les anarchistes ont toutes sortes d'idées et de visions différentes sur la manière dont une société complexe pourrait se gérer elle-même. Les expliquer dépasserait largement le cadre d'un petit texte d'introduction comme celui-ci. Il suffit de dire, tout d'abord, que beaucoup de gens ont passé beaucoup de temps à élaborer des modèles de fonctionnement d'une société vraiment démocratique et saine ; mais ensuite, et c'est tout aussi important, aucun anarchiste ne prétend avoir un plan parfait. De toute façon, la dernière chose que nous voulons, c'est imposer des modèles préfabriqués à la société. La vérité est que nous ne pouvons probablement même pas imaginer la moitié des problèmes qui surgiront lorsque nous essaierons de créer une société démocratique ; néanmoins, nous sommes convaincus que, l'ingéniosité humaine étant ce qu'elle est, de tels problèmes peuvent toujours être résolus, tant que c'est dans l'esprit de nos principes de base - qui sont, en fin de compte, simplement les principes de la décence humaine fondamentale.



\text{Qu’est-ce que la démocratie?}{}
\chapter*{\textbf{Première Partie}}\hypertarget{premire-partie}{}\label{premire-partie}
\markboth{Première Partie}{Première Partie}

On nous dit que nous, États-Uniens, vivons dans le pays le plus riche et le plus démocratique du monde. Nos droits incluent la liberté d'expression et de religion, ainsi que la liberté de voter pour nos dirigeants. Notre pays possède plus de richesses que n'importe quel autre - plus de richesses, en fait, que la majeure partie du reste du monde réuni. À la télévision et dans la vie réelle, nous voyons des États-Uniens avec de grandes maisons, des voitures de luxe, de nombreux gadgets dernier cri et des abonnements à des terrains de golf ou à des stations de ski.

Mais nous savons très bien qu'il ne s'agit pas d'un tableau complet. Il s'agit plutôt d'une publicité. Bien que nos quartiers soient séparés, riches et pauvres, blancs et noirs, latinos et amérindiens, peu de gens ignorent que la plupart des États-Uniens ne vivent pas comme les personnages des sitcoms télévisés. Les habitants des banlieues riches sont souvent confrontés à la pauvreté dans les villes où ils travaillent pour diverses entreprises et administrations. Les habitants des zones défavorisées sont souvent contraints de se rendre dans les banlieues pour travailler et servir du café aux personnes riches et blanches.

Que l'économie aille « bien » ou « mal », des millions de personnes sont au chômage et incapables de se vendre aux employeurs pour acheter les choses dont elles ont besoin. Beaucoup de ceux qui ont un emploi travaillent 40, 60 ou 80 heures par semaine, dans des emplois éreintants, dangereux, malsains et dégradants, juste pour payer un logement, des vêtements, de la nourriture et des médicaments pour eux-mêmes et leur famille. Pendant ce temps, leurs patrons, dont le travail est plus facile et plus sûr, gagnent deux fois plus d'argent, et les personnes qui siègent au conseil d'administration des entreprises ne travaillent pas et gagnent des millions. Les gens sont refoulés des hôpitaux, même en cas d'urgence, et se voient refuser des soins médicaux parce qu'ils n'ont pas les moyens de s'assurer, alors même que les compagnies d'assurance gagnent des centaines de millions de dollars en surfacturant les gens et en essayant de se soustraire au paiement des procédures médicales qu'elles considèrent comme « non essentielles ».

Dans ce pays d'abondance, des personnes dorment dans la rue, mourant dans le froid de l'hiver ou la chaleur de l'été, tandis que des propriétaires conservent des logements vacants en attendant que les prix augmentent. Et la police n'a manifestement aucun problème à battre ou à emprisonner les personnes sans domicile fixe qui squattent des appartements vacants. Pourquoi une grande partie des États-Unis vit-elle dans la pauvreté, alors que d'autres ont plus d'argent qu'ils ne peuvent en utiliser ?

La pauvreté n'est pas notre seul problème. Chaque jour, des policiers racistes battent ou abattent des personnes de couleur, et des millions de personnes, en particulier des personnes noires et des latinos, croupissent en prison, soumises à des peines extrêmement longues et à des conditions horribles pour des délits mineurs souvent anodins. Les femmes font l'objet de discriminations et sont souvent victimes de violences et de viols. Les lesbiennes, les gays, les queers et les transsexuels sont également victimes d'exclusion, de harcèlement et de violence. Les enfants sont traités comme des sous-humains, sans aucun droit et forcés d'aller dans des usines éducatives (« écoles ») où ils sont endoctrinés avec de nombreux mythes néfastes de notre société et où on leur apprend à accepter les problèmes de notre monde comme étant « naturels ». Les entreprises abattent nos forêts, conduisent les plantes et les animaux à l'extinction, empoisonnent le sol, les rivières, l'air et les personnes, tout cela dans l'intérêt du profit. Notre gouvernement déclenche des guerres auxquelles de nombreuses personnes s'opposent et obtient l'obéissance de tous les autres en utilisant les médias pour raconter des mensonges qui entraînent des milliers de morts.

Mais plus que notre conscience de tous ces problèmes, nous savons que nous vivons dans une démocratie et que nous pouvons utiliser nos droits et nos pouvoirs en tant que citoyens pour redresser la situation.

\chapter*{\textbf{Deuxième Partie}}\hypertarget{deuxime-partie}{}\label{deuxime-partie}
\markboth{Deuxième Partie}{Deuxième Partie}

Mais que signifie vivre dans une démocratie ? On nous dit que la démocratie se distingue de la « dictature » par le fait que les citoyens d'une démocratie participent à la prise de décision, alors que dans une dictature, toutes les décisions sont prises par un dirigeant ou un petit groupe de dirigeants. Cependant, dans les sociétés démocratiques, la plupart des gens ne sont pas membres du gouvernement et n'ont pas de contrôle direct sur les décisions qui affectent leur vie, mais ils doivent néanmoins se conformer à ces décisions. La justification est que les sociétés humaines avancées ne peuvent fonctionner sans gouvernement, et que les citoyens concluent donc un « contrat de gouvernés ». Ils s'engagent à suivre les règles et à honorer les décisions du gouvernement, et le gouvernement, en retour, est tenu de protéger ses citoyens et de défendre le bien commun.

Par conséquent, dans une démocratie, les personnes qui ne peuvent pas devenir membres du gouvernement en raison du nombre limité de postes à pourvoir peuvent voter pour leurs dirigeants, qui sont appelés « représentants » parce qu'ils doivent représenter les intérêts de leurs électeurs sous peine de ne pas être réélus. Le vote est donc le droit fondamental dans un État démocratique, et l'État ne peut être considéré comme démocratique que si la majorité de ses citoyens bénéficient de ce droit. Le deuxième droit le plus important est que chacun doit avoir la possibilité d'être élu à un poste gouvernemental, afin d'empêcher l'existence d'une élite permanente ou héréditaire. L'impossibilité apparente de permettre à chacun de participer de manière égale aux fonctions gouvernementales est surmontée par le mécanisme du vote, qui permet aux citoyens d'\emph{exercer leur contrôle sur le gouvernement tout en minimisant leur participation}, en choisissant des dirigeants qui, dépendant de l'élection, doivent « servir » ceux qu'ils « dirigent ».

Les représentants élus votent également sur les décisions proposées, un vote à la majorité décidant de la question en jeu. L'objectif de la prise de décision majoritaire, du moins selon les mythologies des sociétés démocratiques, est que la règle de la majorité résout les injustices antérieures de la règle de l'élite. D'un autre côté, la règle de la majorité menace les droits des populations minoritaires, en particulier dans les sociétés pluralistes. Pour éviter la loi du plus grand nombre, les sociétés démocratiques offrent également des garanties juridiques, ou « droits », à la plus petite des minorités, l'individu. Ainsi, un groupe minoritaire peut fréquemment devoir accepter des décisions qu'il ne soutient pas, mais au moins les membres de ce groupe jouiront toujours d'un ensemble de droits garantis, tels que la liberté d'expression, de religion et de propriété, afin de préserver leur dignité et leur bien-être fondamentaux. Si les droits d'une personne sont violés, elle a en outre le droit d'intenter une action en justice et d'exiger que ses droits soient respectés.

Pour éviter que le gouvernement ne devienne dictatorial, les différentes fonctions du gouvernement sont séparées et des équilibres structurels sont créés pour garantir qu'aucune branche du gouvernement n'accumule une part disproportionnée de pouvoir. Dans une démocratie, une force de police est nécessaire pour protéger les droits individuels, en particulier les droits à la vie et à la propriété, et (en conjonction avec le pouvoir judiciaire) pour punir ceux qui ne respectent pas les décisions de la majorité (lois) telles qu'elles sont exprimées par le pouvoir législatif. Pour protéger la souveraineté de la population et défendre ses droits de propriété dans les pays étrangers, une armée est nécessaire, bien que, pour éviter une dictature militaire, elle soit exclue du processus décisionnel et de l'application des lois (en articulant la mythologie libérale, nous devons proférer quelques faussetés, en ignorant les nombreuses violations nationales du \emph{posse comitatus} tout au long de l'histoire des États-Unis, et l'utilisation constante de l'armée pour appliquer la politique gouvernementale en dehors de nos frontières).

La dernière question est d'ordre économique. De nombreuses questions importantes ne relèvent pas de la sphère politique, mais de la sphère économique. Par conséquent, les États démocratiques vont de pair avec des économies de marché. Dans une économie de marché, chacun a le droit (légalement garanti par le gouvernement) de posséder la propriété privée, de vendre son travail, d'acheter et de vendre des marchandises et de profiter des bénéfices de son travail et de ses entreprises. D'un point de vue \emph{juridique}, tout le monde a \emph{les mêmes chances} de réussir, et la richesse sera donc distribuée à ceux qui la gagnent, plutôt que thésaurisée dans les mains d'une élite.

C'est du moins ainsi que l'on dit que la démocratie fonctionne, presque exactement comme dans les manuels scolaires produits en masse que les enfants sont obligés de lire, et dans des tautologies et des clichés parfois plus éloquents lorsqu'ils sont régurgités par les commentateurs érudits des médias d'information et des universités. Tout ce qui dépasse l'analyse symbolique du fonctionnement réel de notre système démocratique contredit les explications de la mythologie libérale.

\chapter*{\textbf{Troisième Partie}}\hypertarget{troisime-partie}{}\label{troisime-partie}
\markboth{Troisième Partie}{Troisième Partie}

Bien sûr, pour beaucoup de gens, l'idéal démocratique n'a pas de sens. La démocratie états-unienne, en particulier, va de pair avec le « libre marché », ce qui signifie que les riches politiciens blancs du Congrès et de la Maison Blanche ne votent aucune loi ni ne prennent aucune mesure susceptible de restreindre la liberté des riches hommes blancs qui siègent dans les conseils d'administration des entreprises et à Wall Street (les politiciens d'hier et de demain) de gagner des milliards de dollars en exploitant leurs travailleurs. Et ce sont ces travailleurs qui constituent la majorité de la population. Ils n'ont pas la possibilité de voter pour leurs patrons ou de décider collectivement de la politique de l'entreprise qu'ils enrichissent grâce à leur travail. Si c'était le cas, ils pourraient voter pour s'octroyer un salaire décent au lieu d'accorder au PDG une nouvelle augmentation de 100 millions de dollars.

À moins d'appartenir au 1 \% le plus riche de la population et d'avoir suffisamment d'argent pour acheter des terres, une usine ou d'autres moyens de production, et d'embaucher des personnes moins fortunées pour travailler pour nous et nous enrichir, notre seule véritable option est de vendre une partie importante de notre vie pour travailler à l'enrichissement de quelqu'un d'autre. Nous sommes certes libres de choisir, parmi une gamme limitée d'options liées à notre classe économique et à notre éducation, pour quelle entreprise travailler, mais elles sont toutes très similaires, car en fin de compte, le patron détient le pouvoir sur le travailleur, et les entreprises peuvent exploiter les travailleurs à des fins lucratives, mais tous les moyens pratiques dont disposent les travailleurs pour obtenir un peu d'équité de la part des entreprises ont été érigés en infractions. Dans ce pays, tout a un propriétaire, et partout où nous allons, pour tout ce que nous utilisons, nous devons payer un loyer. Toutes les activités nécessaires au maintien de la vie sont taxées, de sorte que notre survie dépend de notre capacité à servir les riches qui ont l'argent pour nous payer. C'est ce qu'on appelle l'esclavage salarié. N’est-il pas absurde de parler de liberté et de démocratie à quelqu'un qui est né dans un ghetto, ou à quelqu'un qui vient d'immigrer pour échapper à la pauvreté ou à la persécution, à quelqu'un qui n'a jamais eu la possibilité de recevoir une bonne éducation et qui travaille 80 heures par semaine dans un emploi dangereux et exténuant, sans dignité ni respect, juste pour pouvoir payer le loyer d'un taudis bon marché et un maigre régime alimentaire ?

Et que signifie la démocratie pour les personnes de couleur, qui sont confrontées au profilage, au harcèlement et à la violence de la police, à des taux de pauvreté plus élevés et à des possibilités d'éducation et d'emploi plus limitées ? Sont-elles vraiment censées croire que les riches politiciens blancs se soucient de représenter leurs intérêts ? La société est tellement habituée à considérer les femmes comme des êtres humains de seconde zone que des problèmes tels que le viol, le harcèlement, la discrimination au travail et les idéaux de beauté imposés par la société, qui entraînent de graves problèmes de santé, ne sont pas considérés comme des injustices pertinentes pour notre démocratie, mais plutôt comme des aspects naturels de l'existence humaine. En réalité, les patrons et les travailleurs ne sont pas égaux, les riches et les pauvres ne sont pas égaux, les personnes blanches et les personnes de couleur ne sont pas égales, les hommes, les femmes, et les autres ne sont pas égaux, mais nos attentes à l'égard de la démocratie sont si faibles que peu de gens considèrent ces « problèmes d’ordre sociétal » comme pertinents pour les affaires de notre gouvernement. Tout ce que nous attendons de notre démocratie, c'est le droit de vote et le droit pour les personnes blanches de la classe moyenne de pouvoir se plaindre sans être persécutés. Attendre davantage est un idéal irréaliste, précisément parce que notre gouvernement a rarement accordé plus que ces quelques droits symboliques.

Ainsi, notre expérience ultime de la démocratie est la suivante : une fois toutes les quelques années, nous avons la possibilité de voter pour l'un des deux hommes riches, blancs et chrétiens, tous deux redevables aux intérêts des entreprises, et nous savons que notre vote n'a pas vraiment d'importance, mais si nous participons, c'est généralement parce que nous pensons que l'un des candidats ne nous trahira pas aussi rapidement que l'autre. Le reste du temps, le fait que nous vivions dans une démocratie ne signifie pas grand-chose. Nous avons le droit de critiquer les politiciens, mais le fait de se plaindre ne semble pas changer le fait que le même groupe est au pouvoir. Nous sommes également libres de nous plaindre de l'aspect le plus important de notre vie, notre travail, mais bien sûr, si les patrons nous entendent, ils sont libres de nous licencier. Tout le monde sait que nous vivons dans une démocratie, mais face au racisme et à l'inégalité économique, peu de gens peuvent dire en quoi ce système de gouvernement nous donne réellement du pouvoir.

\chapter*{\textbf{Quatrième Partie}}\hypertarget{quatrime-partie}{}\label{quatrime-partie}
\markboth{Quatrième Partie}{Quatrième Partie}

Il est cependant facile de rejeter ces affirmations d'impuissance et d'injustice récurrente en reprochant simplement aux victimes d'être trop paresseuses pour sortir de la pauvreté ou pour faire fonctionner le processus démocratique en leur faveur, par le biais de pétitions, de votes, d'envois de lettres et de toutes les autres méthodes facilement disponibles pour remédier à l'injustice alléguée. Bien entendu, il serait plus que ridicule pour les experts blancs privilégiés qui guident les opinions de la nation depuis leurs talk-shows et leurs colonnes d'opinion de reprocher aux personnes nées dans les ghettos de ne pas surmonter le racisme et la pauvreté s'ils n'avaient pas au moins quelques exemples historiques de la manière dont la démocratie peut réellement fonctionner pour aider les personnes dans le besoin. Mais nos livres d'histoire regorgent d'exemples de groupes de personnes opprimées qui ont gagné leur égalité grâce au processus démocratique. Tout le monde connaît l'histoire de Martin Luther King et du mouvement des droits civiques et, comme n'importe quel écolier peut vous le dire, cette histoire se termine bien, car les personnes noires ont obtenu leurs droits. Face à des préjugés séculaires, le processus démocratique a prévalu. Ou bien, a-t-il vraiment prévalu ?

En fait, le processus démocratique avait déjà réussi à vaincre officiellement le racisme au 19éme siècle, lorsque notre gouvernement a accordé tous les droits légaux sans distinction de race, du moins sur le papier. Et en 1954, une décennie entière avant que le mouvement des droits civiques n'atteigne son apogée, la Cour suprême a ordonné la reconnaissance de ces droits légaux, en réponse au travail inlassable, au sein des voies démocratiques légales, de la NAACP et d'autres organisations. Cependant, il n'y a pas eu de véritable changement dans les relations raciales en Amérique. Toutes les réformes obtenues par le biais du processus démocratique étaient symboliques. Ce n'est que lorsque les personnes noires sont descendues dans la rue, souvent illégalement, en dehors du processus démocratique, que ce que nous appelons aujourd'hui le mouvement des droits civiques a pris toute son ampleur. Le mouvement des droits civiques a utilisé l'activisme illégal (la « désobéissance civile ») en tandem avec la pression légale sur le processus démocratique pour apporter des changements, et même dans ce cas, ce n'est que lorsque des émeutes raciales se sont produites dans presque toutes les grandes villes et que des organisations noires plus militantes se sont formées que l'appareil politique blanc a commencé à coopérer avec les éléments pacifistes et de classe moyenne du mouvement, comme la \emph{Southern Christian Leadership Conference} de Martin Luther King, Jr.

Et quel a été le résultat de ce compromis politique ? Les personnes de couleur aux États-Unis sont toujours confrontées à un chômage plus élevé, à des salaires plus bas, à un accès moindre à un logement et à des soins de santé de qualité, à une mortalité infantile plus élevée, à une espérance de vie plus faible, à des taux d'incarcération et de brutalité policière plus élevés, à une représentation disproportionnellement plus faible au sein du gouvernement, de la direction des entreprises et des médias (sauf en tant que méchants à Hollywood ou coupables dans la série télévisée COPS). En fait, le Dr Kenneth Clark, dont les travaux sur les effets psychologiques de la ségrégation sur les élèves noirs ont contribué à la victoire dans l'\emph{affaire Brown v. Board of Education} en 1954, a déclaré en 1994 que les écoles états-uniennes étaient plus ségréguées qu'elles ne l'étaient quarante ans plus tôt. La suprématie blanche existe toujours dans tous les domaines de la vie états-unienne.

Qu'est-ce que le mouvement des droits civiques a accompli exactement ? L'accès aux institutions dominées par les blancs a été ouvert à un très petit nombre de personnes noires, latines, et asiatiques, en particulier ceux qui adhèrent à l'idéologie conservatrice du statu quo suprématiste blanc, comme le juge de la Cour suprême Clarence Thomas, qui s'oppose aux quotas ou à d'autres mesures juridiques visant à atténuer les inégalités raciales, ou le général Colin Powell, qui est prêt à bombarder des personnes de couleur dans des pays étrangers, au mépris total de leur vie. Martin Luther King est donc mort, mais son rêve se perpétue à travers la poignée disproportionnée de membres noirs et latinos du Congrès, les un ou deux PDG de couleur parmi les 500 plus riches et les émissions de télévision occasionnelles qui dépeignent des familles noires aisées de la classe moyenne, comme les Cosby, à l'abri des brutalités policières ou de l'exploitation économique.

Le gouvernement a conservé son caractère suprémaciste blanc et, plus important encore, il est \emph{plus puissant aujourd'hui} qu'il ne l'était avant le mouvement des droits civiques, car il a largement éliminé la menace des conflits raciaux et des soulèvements motivés par l'oppression ; quelques personnes de couleur tokenisées accèdent à des postes de pouvoir, donnant l'illusion de l'égalité, mais les populations de couleur restent dans l'ensemble un réservoir de main-d'œuvre excédentaire bon marché dont le système peut user et abuser en fonction des besoins. Quand l'on considère la manière dont le gouvernement a réellement réagi au mouvement des droits civiques et les changements qui en ont résulté dans notre société, il apparaît clairement que le processus démocratique a été plus efficace pour sauver les dirigeants d'une situation d'urgence potentielle que pour apporter un réel soulagement ou une véritable libération à un groupe de personnes opprimées.

Et ce ne sont pas seulement les groupes minoritaires qui sont ignorés par le gouvernement. Même dans les situations historiques où la majorité de la population souhaite un changement, ce sont les intérêts des riches et des puissants qui dirigent la décision. Avant l'ère Reagan, une majorité de citoyens était en faveur d'une aide sociale fournie par le gouvernement \emph{afin de garantir à chacun l'accès à un minimum de nourriture, de logement et de soins médicaux}. Puis, pendant plusieurs années, une campagne a été menée par les politiciens et les médias (appartenant aux mêmes entreprises qui faisaient élire les politiciens grâce à des dons massifs), à l'aide de slogans, de publicités, de statistiques manipulées et d'une couverture sélective, pour dépeindre les bénéficiaires de l'aide sociale comme des drogués paresseux profitant d'une situation privilégiée.

Après cette vaste campagne de propagande, une majorité d'États-Uniens interrogés se sont déclarés opposés à l' « aide sociale », mais, curieusement, ils se sont tout de même déclarés en faveur d'un filet de sécurité fourni par le gouvernement \emph{pour garantir à chacun un accès minimal à la nourriture, au logement et aux soins médicaux}. Les médias les avaient programmés pour associer le mot « aide sociale » à un certain nombre de mauvaises choses, même s'ils soutenaient l'idée de l'aide sociale. Les politiciens pouvaient prétendre qu'ils agissaient dans l'intérêt du peuple lorsqu'ils démantelaient l'aide sociale au profit des bénéfices des entreprises, mais en réalité, l'élite travaillait très dur pour s'assurer que le peuple croyait ce qu'elle voulait qu'il croie. Le consentement démocratique a été fabriqué d'en haut.

\chapter*{\textbf{Cinquième Partie}}\hypertarget{cinquime-partie}{}\label{cinquime-partie}
\markboth{Cinquième Partie}{Cinquième Partie}

Ce sont les détenteurs du pouvoir et de l'argent qui décident des politiciens à élire. Une personne ne peut être désignée comme candidate à l'un des deux grands partis sans avoir de solides alliances au sein du parti. Par conséquent, avant même que quelqu’un puisse être considéré comme un candidat possible à l'élection, il (ou parfois elle !) doit faire appel à ceux qui sont déjà au pouvoir. Et après avoir reçu la nomination du parti, il est impossible d'être élu au Congrès ou à la Maison Blanche sans une énorme campagne de publicité, qui coûte des millions de dollars. Les entreprises et les particuliers fortunés fournissent la majorité de ces dons, et ils ne contribuent qu'aux campagnes des candidats qui promettent de servir les intérêts des riches. Un politicien qui trahit les entreprises qui le soutiennent, par exemple en soutenant une loi qui obligerait les employeurs à verser un salaire décent à leurs employés, ne sera pas réélu.

Mais le fait que les entreprises de médias, qui informent les opinions et les décisions de chacun, ne sont pas des institutions publiques, mais d'énormes sociétés de divertissement privées, conglomérées et à but lucratif, qui possèdent ou sont possédées par des sociétés d'autres secteurs, est encore plus important. Les entreprises qui fabriquent les produits que vous achetez dans les magasins, qui fabriquent les armes utilisées dans les guerres, les voitures que vous conduisez, l'essence que vous utilisez ; les entreprises qui sous-payent leurs travailleurs, détruisent l'environnement, polluent votre air, achètent vos « représentants » politiques. L'entreprise pour laquelle vous travaillez peut-être.

En outre, ces entreprises reçoivent leur argent d'autres sociétés qui achètent des publicités, et elles représentent les intérêts de ces sociétés, et de leurs PDG riches et blancs, avant de représenter vos intérêts. Que vendent-elles lorsqu'elles vendent de l'espace publicitaire ? Elles vous vendent vous. Vous achetez donc ce qu'on vous dit, vous votez comme on vous le dit et vous n'exercez que les choix limités qu'elles jugent acceptables. Parce qu'il n'est pas directement lié au gouvernement, ce réseau d'entreprises (qui vous fournit la quasi-totalité de vos informations sur le monde) constitue la machine de propagande la plus efficace et la plus crédible de l'histoire du monde.

Un dernier fait important est que les personnes qui contrôlent le gouvernement, les médias et les entreprises sont le même groupe de personnes. Les politiciens de haut niveau arrivent souvent au pouvoir après avoir fait carrière dans de puissantes entreprises et, après avoir mené une carrière fructueuse en tant qu'élus, « au service de leur pays », ils retournent généralement à la vie d'entreprise, gagnant encore plus d'argent en tant que consultants, lobbyistes et dirigeants d'entreprise. Le gouvernement n'a pas besoin de contrôler directement les médias, et les entreprises n'ont pas besoin de contrôler directement le gouvernement, parce qu'ils sont tous dans le même bateau, et qu'ils servent tous les mêmes intérêts : à savoir, les leurs. Après tout, les politiciens travaillent pour les mêmes personnes que les présentateurs de journaux télévisés. Ils ont fréquenté les mêmes écoles, ils vivent dans les mêmes banlieues riches et les mêmes communautés protégées, et entre les sessions du Congrès ou avant le tournage du journal télévisé du soir, ils jouent au golf ensemble.

\chapter*{\textbf{Sixième Partie}}\hypertarget{sixime-partie}{}\label{sixime-partie}
\markboth{Sixième Partie}{Sixième Partie}

Comment se fait-il que les riches et les puissants soient pris en charge, alors que tous les autres bénéficient de réformes symboliques qui ne résolvent pas leurs problèmes fondamentaux ? Quand est-ce que notre gouvernement démocratique est devenu si corrompu ? La réponse est en fait très simple. Il n'a jamais été corrompu, parce que le gouvernement démocratique a toujours existé pour protéger les intérêts des riches et des puissants. Si l'on va au-delà de ce qui est prôné dans les manuels scolaires et que l'on examine l'évolution réelle de la démocratie, on s'aperçoit qu'il s'agit simplement d'une autre forme de gouvernement dans le continuum historique des royaumes féodaux et des monarchies constitutionnelles. La démocratie n'est pas un nouveau produit de la lutte populaire et de la demande d'égalité face à la tyrannie. Il s'agit d'une \emph{évolution directe d'institutions élitaires antérieures}, créées pour, par et par les élites libérales d'Europe et d'Amérique.

Tout au long de l'histoire de l'Europe post-romaine, l'évolution vers des formes constitutionnelles et électorales n'a pas été le résultat d'une lutte populaire pour la libération. Au contraire, le gouvernement démocratique a été formulé pour apaiser l'aristocratie et la bourgeoisie, qui souhaitaient une \emph{coalition incluant l'ensemble de l'élite économique} dans la direction politique, et pas seulement le monarque et la bureaucratie à lui subordonnée. La démocratie, après tout, n'est pas un concept des Lumières. Le terme même que ces hommes d'État européens et États-Uniens éclairés ont choisi pour décrire le système politique qu'ils souhaitaient est emprunté aux cités-États de la Grèce antique, dans lesquelles tous les citoyens masculins possédant des biens avaient la possibilité d'influencer les dirigeants. Bien entendu, les classes inférieures étaient des esclaves et non des citoyens, de sorte que seuls 10 \% environ de la population pouvaient voter. Dans les premières cités-États, il n'y avait que peu ou pas de distinction entre le pouvoir politique et le pouvoir économique, car l'élite économique était, bien entendu, la bénéficiaire du pouvoir consolidé par les nouvelles structures politiques qu'elle avait créées. Au fur et à mesure que les empires se développaient, une grande partie de l'élite économique - les propriétaires terriens aristocratiques - était souvent exclue, dans une certaine mesure, du groupe d'élite détenant le pouvoir sur et à partir de l'appareil politique centralisé. C'est la lutte de l'aristocratie, et plus tard des marchands bourgeois, des banquiers et des propriétaires d'usines, pour se réincorporer dans l'élite politique, qui est à l'origine de l'évolution de ce processus politique que nous appelons la démocratie.

Aujourd'hui, nous voyons plus clairement l'évolution de la démocratie dans les États-nations européens, dont on dit souvent qu'elle a commencé avec la Magna Carta. Ce célèbre document, ainsi que les droits et garanties juridiques qu'il établit, a été créé lorsque le roi Jean d'Angleterre, confronté à la perspective d'être déposé militairement par l'aristocratie, a jugé bon d'étendre le pouvoir politique à une partie plus large de l'élite économique qu'auparavant, en garantissant des \emph{droits} aux principaux propriétaires terriens et en créant le précédent d'un conseil de barons, ou de leurs représentants, chargé de conseiller le roi et de négocier avec lui.

Le chancelier Bismarck, qui a unifié l'Allemagne en une démocratie constitutionnelle, n'était pas un populiste. Au contraire, son règne s'est caractérisé par une répression sévère des éléments progressistes et radicaux et par un désir machiavélique de renforcer l'État allemand. Il a tellement bien réussi que dans l'espace de quelques décennies, l'Allemagne est passée d'un ensemble de provinces arriérées et désunies à un État-nation uniforme, capable de menacer à lui seul le reste du continent. \emph{Bismarck savait que l’établissement d'élections et de droits constitutionnels ne ferait que consolider le pouvoir de l'élite dirigeante allemande}, en gagnant la loyauté de la bourgeoisie et de l'aristocratie, en épuisant ou en cooptant les efforts des progressistes qui cherchaient à changer la société par le biais du processus électoral et en marginalisant les radicaux qui rejettaient le « processus démocratique », éliminant ainsi le spectre de la résistance ou de la non-coopération qui nuisaient à l'efficacité de nombreux autres États européens qui essayaient constamment de gagner l'obéissance de leurs sujets opprimés. En outre, le \emph{pouvoir} politique et économique, jamais redistribué, \emph{était déjà consolidé entre les mains de l'élite}, qui pouvait s'assurer que seuls ses candidats étaient élus et que seules des lois favorables étaient adoptées, par divers moyens légaux ou illégaux (la légalité étant ici une question farfelue, car la police, historiquement partie intégrante de l'appareil monarchique, n'était pas prête à arrêter ses propres maîtres).

L'évolution sporadique de la démocratie en Russie a suivi une voie similaire à celle de l'Angleterre et de l'Allemagne, à la différence près que la plupart des réformes libérales ont été abrogées par un tsar jaloux, peu enclin à partager son pouvoir. L'existence d'un parlement russe a temporairement atténué l'agitation populaire, mais lors de sa dissolution, les courants subversifs qui ont finalement conduit à la révolution bolchevique ont repris de plus belle. Le parlement russe, actuellement appelé Douma, s'appelait au 19th siècle, avec un peu plus de candeur, la « Douma de Boyarskoe » (« Douma » voulant dire pensée, et les « boyards » étant l'aristocratie russe). Avant cela, les serfs russes ont été « libérés » en tant qu'étape nécessaire de l'évolution démocratique. Bien entendu, ils n'ont pas reçu la \emph{terre} qu'ils avaient travaillée et sur laquelle ils vivaient (et dont ils dépendaient pour leur survie) ; cette terre est restée entre les mains de l'aristocratie, bien que les serfs aient été autorisés à en acheter environ un tiers. Étant donné qu'ils étaient jusque là des travailleurs non rémunérés et qu'ils n'avaient pas d'argent pour acheter la terre, certains des serfs « libérés » ont dû se rendre dans les villes et travailler comme salariés dans les nouvelles usines (un arrangement qui, par coïncidence, convenait parfaitement aux propriétaires des usines et à l'élite politique russe, qui avait besoin de l'industrialisation pour rester une puissance européenne compétitive), tandis que les autres anciens serfs sont restés à la campagne et ont travaillé comme métayers pour leurs anciens maîtres.

Les premiers organes représentatifs du gouvernement, les précurseurs du Congrès ou du Parlement moderne, étaient dès le départ censés représenter l'aristocratie, les propriétaires fonciers, les banquiers et toutes les autres personnes riches qui contrôlaient la vie économique de la nation. La représentation de l'élite économique garantissait que les dirigeants politiques (anciennement le monarque) qui contrôlaient l'armée, la police, le système fiscal et d'autres bureaucraties, protégeraient et serviraient les intérêts des riches. La singularité du monarque a été remplacée par une coalition de l'élite, divisée en partis politiques et rivalisant pour l'influence, mais surtout collaborant au niveau fondamental \emph{pour maintenir le contrôle}. Le vote a permis de garantir que le parti ayant la stratégie de contrôle la plus populaire puisse la mettre en œuvre, alors qu'auparavant, le conservatisme et l'obstination d'un souverain unique et incontesté risquaient d'être moins flexibles pour s'adapter aux changements de circonstances.

Au fur et à mesure que le droit de vote s'est étendu à tous les citoyens adultes (en phase, ce qui n'est pas une coïncidence, avec l'essor des médias de masse contrôlés par les entreprises), le vote a également servi à donner l'illusion de l'égalité, à créer une soupape de décompression pour le mécontentement populaire et, surtout, à \emph{maintenir l'efficacité du contrôle gouvernemental} en favorisant les partis politiques qui réussissaient le mieux à duper la population et à gagner son obéissance. L'absence de participation réelle de la population est d'autant plus évidente que les choix des électeurs sont principalement guidés par la reconnaissance du nom, l'affiliation à un parti et le bombardement de slogans superficiels par les médias publicitaires, et que \emph{peu d'électeurs sont capables de formuler une différence factuelle entre les programmes des candidats opposés,} - et encore moins une analyse critique de leurs politiques.

Le système bicaméral, caractéristique des États-Unis et d'autres démocraties, s'est re-développé en Angleterre, où les deux chambres parlementaires ont été nommées avec plus d'honnêteté qu'il ne serait permis de le faire à l'époque moderne. La Chambre des Lords a été créée pour les représentants de l'aristocratie, et la Chambre des Communes pour ceux qui n'avaient pas de titre de noblesse - plus précisément, pour les représentants de la bourgeoisie ou de la classe moyenne supérieure. L'exclusion de la majorité de la population, même de la chambre la plus basse du parlement, devient évidente lorsqu'on essaie de trouver des roturiers pauvres et issus de la classe ouvrière parmi les membres du parlement, tout au long de l'histoire de la Chambre des communes jusqu'à aujourd'hui. Il en va de même pour les membres du Congrès américain, dont le revenu moyen avant leur élection ne s'est jamais approché de la faible moyenne de l'ensemble de la population états-unienne. Les quelques représentants issus de la classe moyenne inférieure occupent généralement des postes bien rémunérés de consultants d'entreprise après un mandat réussi au Congrès, et aucun homme politique au niveau national n'est issu de la classe inférieure, qui constitue la grande majorité de la population totale.

Il ne s'agit en aucun cas d'une évolution récente du gouvernement américain. Certains des pères fondateurs envisageaient le rôle du président comme celui d'un roi et ont suggérés divers titres majestueux. Comme à l'époque la majorité des gens étaient analphabètes, l'élite pouvait être beaucoup plus directe, et ses commentaires sont très éclairants. Le père de la Constitution, James Madison, a écrit que : « La minorité des opulents {[}les classes riches{]} doit être protégée de la majorité ». Son ami et collègue fédéraliste influent, John Jay, a dit plus clairement que « les gens qui possèdent le pays doivent le gouverner ». La révolution démocratique en Amérique a été la tentative réussie de l'élite économique états-unienne de s'emparer du pouvoir politique des mains des Britanniques. Les plaintes concernant la fiscalité britannique injuste étaient les plaintes d'hommes d'affaires. Lorsque les paysans États-Uniens, déçus que leur situation économique difficile ne s'améliore pas après la révolution, se sont révoltés contre la nouvelle élite états-unienne dans les capitales des États, les Pères fondateurs (qui étaient des marchands, des banquiers et des avocats du Nord et des propriétaires terriens esclavagistes du Sud) se sont réunis pour créer un gouvernement plus fort et centralisé qui protégerait les intérêts des minorités, c'est-à-dire les intérêts de l'élite dirigeante.

La nouvelle Constitution a créé un certain nombre de structures et de droits, les droits étant les privilèges codifiés de l'élite. Un système électoral permettait à ceux qui possédaient la terre, les banques et les usines de décider quels hommes politiques représenteraient le mieux leurs intérêts. Avec l'élargissement du droit de vote, les élections ont également eu pour fonction de tester quel candidat avait la meilleure rhétorique populiste, la meilleure stratégie pour conserver la soumission et la loyauté de l'ensemble de la population. Le fameux équilibre des pouvoirs aux États-Unis, entre les juges, les sénateurs, les présidents et les généraux, est une coalition dirigeante au sein de l'élite. La liberté d'expression était et reste la liberté pour les membres de l'élite de critiquer la politique gouvernementale afin de formuler des stratégies de domination plus efficaces. Incidemment, la liberté d'expression permet également à tout citoyen ordinaire de marmonner ce qu'il veut, bien que l'histoire états-unienne montre constamment que les gens ne sont pas libres de la menace d'arrestation et d'emprisonnement pour des propos impopulaires si les autorités craignent que ces propos aient un effet réel, au-delà de souffle gaspillée dans une conversation futile.

Dans la mythologie libérale, la démocratie repose sur l'idée que les gens se regroupent sous la protection d'un gouvernement et concluent un « contrat de gouvernés ». Mais il s'agit d'un contrat que nous ne pouvons ni négocier ni refuser. Nous naissons tous en tant que sujets de l'un ou l'autre État, « démocratique » ou non, et si nous nous opposons à notre assujettissement, nous ne pouvons rien y faire. Même si nous avons les moyens financiers de quitter notre pays d'origine (sans parler de la question de faire partir le gouvernement de nos maisons), nous n'avons pas d'autres options : Le « No Man's Land » n'existe pas. \emph{Si nous n'avons pas le choix pratique de refuser, notre acquiescement n'est pas un consentement, c'est une soumission}.

En réalité, le processus démocratique est conçu pour former et maintenir une coalition efficace au sein de l'élite, pour gagner la loyauté de la classe moyenne en lui accordant des droits et des privilèges symboliques, pour prévenir le mécontentement en créant l'illusion de l'équité et de l'égalité, et pour étouffer la rébellion en établissant un ensemble élaboré de canaux officiels pour la dissidence sanctionnée ; et d'étouffer la rébellion en établissant un ensemble complexe de canaux officiels pour la dissidence sanctionnée, en épuisant l'énergie des dissidents respectueux de la loi qui passent par toutes les étapes - et obtiennent éventuellement quelques concessions mineures, et en refusant la légitimité à ceux qui sortent du « processus démocratique » pour provoquer directement le changement qu'ils recherchent, plutôt que de participer au rituel de cour élaboré conçu pour démontrer leur loyauté en demandant au gouvernement de prendre en considération leurs requêtes. Une fois que ces rebelles peuvent être décrits comme « illégitimes », « imprudents », « impatients », « inconsidérés » ou « manquant de respect pour le processus démocratique », le gouvernement peut en toute sécurité les traiter beaucoup plus durement que ceux qui honorent encore le « contrat des gouvernés » par leur docilité et leur soumission.

\chapter*{\textbf{Septième Partie}}\hypertarget{septime-partie}{}\label{septime-partie}
\markboth{Septième Partie}{Septième Partie}

Notre analyse plus approfondie de ce système que nous appelons « démocratie » nous a conduits à l'hypothèse suivante : à la base, la démocratie est un système de gouvernement autoritaire et élitiste conçu pour former une coalition dirigeante efficace tout en créant l'illusion que les sujets sont en fait des membres égaux de la société, qui contrôlent donc la politique du gouvernement, ou du moins y sont représentés de manière bienveillante. L'objectif fondamental d'une démocratie, comme de tout autre gouvernement, est de maintenir la richesse et le pouvoir de la classe dirigeante. La démocratie est innovante en ce qu'elle permet à une plus grande diversité de voix de la classe dirigeante de défendre diverses stratégies de contrôle, et « progressiste » dans le sens qu'elle permet de s'adapter pour maintenir le contrôle dans des circonstances changeantes.

Le moyen le plus sûr de vérifier cette hypothèse est d'observer des exemples historiques dans lesquels les citoyens opprimés ou défavorisés d'une démocratie ont défendu leurs propres intérêts, en contradiction avec les intérêts des riches et des puissants. Si le mythe libérale concernant la démocratie est correct, les opprimés seront justement représentés, des représentants politiques défendront leur cause et un compromis équitable sera trouvé entre les privilégiés et les opprimés. Si les progressistes et autres réformistes ont raison de croire que le système est fondamentalement sain mais corrompu par diverses causes qui peuvent être résolues par une législation appropriée, alors les riches et les puissants bénéficieront d'avantages injustes dans les processus législatifs et judiciaires mis en œuvre pour parvenir à la justice. Si notre hypothèse sur la nature autoritaire et élitiste de la démocratie est correcte, les nombreuses institutions du pouvoir collaboreront pour diviser l'opposition, gagner les éléments réformistes et écraser l'opposition restante afin de conserver le contrôle par tous les moyens nécessaires, y compris la propagande, la calomnie, le harcèlement, l'agression, l'emprisonnement sur la base de fausses accusations et l'assassinat.

Les éléments les plus militants ou radicaux de la lutte contre l'oppression raciale dans les années 1960 en sont un excellent exemple. Les inégalités raciales de l'époque sont solidement documentées comme étant fortes et omniprésentes, et de nombreuses organisations se sont formées pour combattre cette oppression raciale. Les \emph{Black Panthers}, par exemple, réclamaient plus que de voir des personnes noires de la classe moyenne. Ils voulaient la libération des personnes noires, une transformation sociale totale qui éliminerait la suprématie blanche de tous les aspects de la vie. En réponse aux brutalités policières, ils ont également commencé à prôner l'autodéfense des personnes noires. Comment les contrôleurs du processus démocratique ont-ils réagi ? À la fin des années 1960, J. Edgar Hoover, chef du FBI, les a qualifiés de « plus grande menace pour la sécurité intérieure des États-Unis ». En grande partie grâce à un programme du FBI appelé COINTELPRO, les Black Panthers ont été harcelés, calomniés, battus, intimidés, leurs communications ont été interceptées et trafiquées pour provoquer des scissions entre les factions. Leurs efforts, y compris les programmes alimentaires pour les enfants des écoles, ont été sabotés ; le FBI et la police locale ont acheté des informateurs et placé des provocateurs dans leurs rangs, ou ont arrêté à plusieurs reprises des organisateurs des Panthers par d'accusations sans fondement pour les faire payer une caution, les harcelant et épuisant leurs ressources. Des Panthers ont été arrêtées et condamnées sur la base d'affaires fabriquées de toutes pièces. Dans un cas, un Black Panther a été emprisonné pendant plus de vingt ans pour des meurtres qu'il ne pouvait pas avoir commis, puisqu'il se trouvait à des centaines de kilomètres de là, dans une autre ville, au moment des faits. Il a défendu son alibi devant le tribunal en affirmant que le FBI avait installé des micros dans le bureau des Panthers où il travaillait et que les enregistrements du FBI prouveraient ses allées et venues. Au tribunal, les agents du FBI ont menti à la barre et nié qu'ils effectuaient une telle surveillance, bien qu'ils aient été contraints par la suite de publier des documents prouvant le contraire. Ils avaient « perdu » les enregistrements des jours en question (ça tombait bien!).

Et lorsque l'emprisonnement ne suffisait pas, les militants des Black Panthers étaient tout simplement assassinés. En l'espace de deux ans, vingt-huit Panthers ont été tuées (certaines dans leur sommeil) par la police et le FBI. Même si les Panthers étaient aussi violentes et impures que le prétendent les plus enragés et mal informés de leurs détracteurs, pourquoi le gouvernement a-t-il traité (aux niveaux local, étatique et national) une organisation bien plus violente, le Ku Klux Klan, avec autant de tolérance (et, dans de nombreux cas, de collaboration) ?

MOVE, une autre organisation de libération des personnes noires, basée à Philadelphie, a été bombardée par un hélicoptère de la police au cours d'un affrontement massif qui s'est soldé par la mort d'un policier. Plusieurs des membres de MOVE sortis de leur maison après le raid ont été battus presque à mort par la police. Huit membres du MOVE ont été emprisonnés, même si les preuves médico-légales (dont une grande partie a été falsifiée par la police) suggèrent que le policier a été tué par un tir d’un autre policier. Plus que la question de savoir si le policier a été tué par l'un des siens ou s'il a été abattu en état de légitime défense par des membres du MOVE, la question de savoir pourquoi exactement la police a organisé un assaut armé contre la maison du MOVE est plus importante.

Le \emph{American Indian Movement} a été traité de la même manière. Leurs membres étaient victimes de harcèlement, d'assassinats et de faux emprisonnements (leur prisonnier politique le plus célèbre étant Leonard Peltier, qui purge une peine de prison à vie pour avoir tué un agent du FBI lors d'un raid, même si l'accusation a admis qu'elle ne pourrait jamais être sûre de l'auteur du coup de feu fatal).

Le recours à la violence par notre gouvernement démocratique contre les dissidents se poursuit encore aujourd'hui. Lors des réunions de l'Organisation mondiale du commerce à Seattle, en 1999, lorsque les manifestants ont été plus nombreux que prévu à se mobiliser et à bloquer le sommet, la police a réagi violemment, en frappant les manifestants et les passants, en les aspergeant de gaz lacrymogène et en leur tirant des balles en caoutchouc. Pour disperser les manifestants enfermés, ils leur ont forcé la tête, les ont aspergés de gaz poivré sous les paupières et ont utilisé d'autres techniques de torture. Tout cela a été filmé, mais les médias nationaux ont ignoré les brutalités policières et ont préféré diffuser des clips montrant des manifestants brisant des vitrines, en les présentant comme la raison de l'intervention massive de la police, alors que la chronologie réelle était inversée.

Au cours de l'été 2002, la police de Washington a fait une descente dans la communauté Olive Branch, un collectif de pacifistes et d'anarchistes politiquement actifs, et a expulsé les résidents sous la menace d'une arme. En 2003, un homme a été arrêté à l'aéroport d'Atlanta parce qu'il tenait une pancarte pour protester contre l'arrivée du président GW Bush. Il a été accusé d'avoir mis en danger le président. Un peu plus tard la même année, l'anarchiste Sherman Austin, webmaster d'un site web à succès consacré aux luttes des personnes de couleur, a été condamné à un an de prison après qu'\emph{une autre personne a} publié sur son site web un lien vers des instructions pour la fabrication de cocktails Molotov sur son site web. Pour ce crime, des agents fédéraux munis d'armes automatiques ont encerclé sa maison, défoncé sa porte et l'ont tiré du lit. La violence et la répression autoritaires sont quotidiennes, trop fréquentes pour être toutes citées. Ces exemples ne sont donc que des exemples parmi d'autres. Pour le reste, vous devrez faire vos propres recherches.

Certains libéraux qui veulent croire que la violence du gouvernement américain n'est que le résultat de services de police corrompus et non un élément fondamental et nécessaire du système idéalisent souvent d'autres pays, en particulier les démocraties sociales d'Europe, en utilisant leur ignorance de la violence autoritaire dans ces pays comme preuve de l'absence d'une telle violence. Avec un peu de recherche, nous découvrons que les gouvernements démocratiques du Canada, de l'Allemagne, de la Grande-Bretagne, du Mexique, du Japon, de l'Italie et d'autres pays ont également recours à une violence régulière à l'encontre des dissidents.

\chapter*{\textbf{Huitième Partie}}\hypertarget{huitime-partie}{}\label{huitime-partie}
\markboth{Huitième Partie}{Huitième Partie}

La question demeure : Que faire ? Malheureusement, trop de gens adhèrent aux contraintes artificielles du système, choisissant toujours le moindre des deux maux, motivés uniquement par la peur d’un mal plus grand, comme s'ils étaient impuissants à remettre en question le cadre social et à créer de nouvelles alternatives (ce constat d'impuissance au sein du système démocratique devrait suffire pour que les gens se révoltent !). Rien dans les lois physiques de l'univers, ni aucune règle régissant le comportement humain, n'exige que le monde soit dominé par une élite ploutocratique exerçant un contrôle autoritaire et exploiteur sur tous les autres. En fait, la majorité des sociétés humaines se sont organisées très différemment, souvent sous des formes égalitaires, jusqu'à ce que l'impérialisme européen et américain interrompe toutes les autres expériences culturelles et les remplace par les nôtres, de sorte que presque tous les pays du monde pratiquent la démocratie représentative et le capitalisme industriel, qui sont des formes d'organisation socio-économique très particulières, entièrement eurocentriques et largement inadaptées (sauf en termes de maintien du contrôle et d'exploitation de la valeur).

Pour de nombreux progressistes états-uniens, l'idée d'envisager de nouvelles alternatives consiste à soutenir les troisièmes partis, comme si l'existence de troisièmes et de quatrièmes partis avait rendu les États européens moins oppressifs. Demandez aux Roms si le parti vert a fait la moindre différence lorsqu'ils ont été expulsés en masse d'Allemagne, plus de quarante ans après la fin du Troisième Reich. Demandez aux manifestants de Gênes, qui ont été mis contre un mur et battus jusqu'à ce que leur sang et leurs dents décorent le béton, ce qu'ils pensent d'un système parlementaire. D'autres progressistes sont favorables à ce qu'ils considèrent comme des changements structurels, tels que des amendements à la Constitution, sans se rendre compte que le pouvoir n'existe pas sur le papier. Ces réformistes croient peut-être que l'égalité raciale aux États-Unis a été atteinte en 1868, avec l'adoption de l'amendement 14th , ou que le mouvement des droits civiques s'est achevé en 1964, avec la loi sur les droits civiques. Pour corriger leur naïveté, ils n'ont qu'à passer un peu de temps dans une prison et à rechercher le degré de protection que le quatrième amendement a accordé aux détenus toxicomanes de ce pays.

Dans les rares cas où le processus démocratique a « fonctionné », le système entier n'hésite pas à ignorer les lois de réforme qui contredisent les intérêts des puissants. Jimmy Carter, le président le plus libéral que les États-Unis aient jamais connu (mais qui n'est pas un saint, si l'on en croit les expériences des Cambodgiens, des Indonésiens, des Haïtiens et d'autres), a interdit par décret plusieurs programmes de contre-espionnage de l'ère vietnamienne qui incluaient la torture et l'assassinat. Grâce à un officier consciencieux de l'École des Amériques de l'armée états-unienne, nous savons que l'armée a tout simplement ignoré l'ordre de Carter et a continué à enseigner ces tactiques. Combien d'exemples similaires sont restés secrets ?

Dans une société où le pouvoir est tellement concentré entre les mains de quelques-uns, \emph{le pouvoir se défend lui-même}. Pensons-nous vraiment que si nous élisons un président ou un congrès « décent », toutes les institutions de l'élite qui s'auto-perpétuent se contenteront d'abandonner et de céder leurs richesses ? Dans les pays où les organes élus du gouvernement ont cessé de représenter les intérêts des puissants, l'armée et les entreprises qui la soutiennent (la coalition de l'élite) ont conspiré pour renverser les parties rétives du gouvernement (au Chili, au Venezuela, en Espagne, au Congo, etc.). Les entreprises et les armées d'Europe et d'Amérique du Nord sont-elles d'une manière ou d'une autre plus pures ? Après tout, c'est le Pentagone (ou Exxon) qui a parrainé nombre de ces coups d'État élitistes (souvent fascistes ou nationalistes d'ultra-droite) à travers le monde.

Les citoyens des démocraties modernes sont tellement paralysés par une peur tenace de l'action autonome et directe - prendre l'initiative de faire les choses par nous-mêmes et de résoudre nos propres problèmes - que prôner le renversement révolutionnaire de l'ordre actuel semble revenir à prôner l'apocalypse ; pourtant, les deux actions essentielles que nous devons entreprendre pour nous libérer sont l'autonomie et l'abolition des relations sociales, politiques et économiques actuelles.

Nous ne pouvons tout simplement pas continuer à attendre que d'autres personnes nous sauvent. C'est notre dépendance à l'égard de Big Brother qui perpétue les erreurs du système. Comme un muscle inutilisé, notre capacité à prendre soin de nous-mêmes, à prendre nos propres décisions, à régir nos relations avec les autres, à créer des associations volontaires et à construire des communautés, à résoudre nos différends et, surtout, à nous faire confiance, s'est atrophiée, mais nous devons affiner ces capacités pour nous libérer de la domination autoritaire qui nous gouverne depuis des millénaires.

Deuxièmement, \emph{nous ne pouvons pas continuer à considérer l'égalité} - la \emph{véritable égalité} - \emph{comme une mesure extrême}. C'est le système actuel qui est extrême et nous devons en détruire tous les vestiges pour nous en libérer et l'empêcher d'évoluer vers une nouvelle forme déguisée. Le gouvernement, sous quelque forme que ce soit, est autoritaire. De même, le pendant du gouvernement démocratique - le système économique du « libre marché », qui n'est jamais né ou entré en contact avec le mythique « level playing field » que les économistes libéraux envisagent pour justifier leur système, est une autre structure de gouvernement (relative aux moyens de production et de consommation, plutôt qu'à l'appareil politique) qui permet à l'État de s'affranchir de l'autorité de l'État, plutôt que l'appareil politique) qui permet un certain degré de concurrence et de participation ayant l'apparence de l'équité et de l'ouverture mais qui, en réalité, est conçue pour augmenter l'efficacité du contrôle des moyens de production tout en conservant ce contrôle entre les mains d'un groupe dont la composition peut être quelque peu fluctuante mais qui reste clairement un groupe d'élite. Dans ce système de marché libre, un très petit nombre de personnes contrôlent les moyens de production (les usines, la terre, etc.), ce qui rend l'autosuffisance impossible. Pour se procurer les produits nécessaires à la survie et à une existence culturellement normale, tous les autres doivent vendre leur activité contre un salaire. La seule façon de corriger la situation est de reprendre ce qui nous a été volé.

La production et la prise de décision doivent être décentralisées, et la richesse et le pouvoir doivent être partagés au niveau de la communauté dont ils sont issus. Les structures étatiques doivent être démantelées, les richesses et les moyens de production doivent être confisqués à la minorité qui les contrôlent, les prisons détruites, les armées anéanties. Des formes plus intimes d'oppression, comme le patriarcat et la suprématie blanche, doivent être dénoncées et remises en question partout où elles persistent.

\chapter*{\textbf{Neuvième Partie}}\hypertarget{neuvime-partie}{}\label{neuvime-partie}
\markboth{Neuvième Partie}{Neuvième Partie}

L'expression « plus facile à dire qu'à faire » est un très grand euphémisme. Peut-être la raison pour laquelle tant de gens continuent à croire en l'efficacité de réformes mineures, face à de l’évidence contradictoire écrasante, est que l'énorme responsabilité à laquelle nous sommes confrontés en réalisant que les problèmes de notre société sont fondamentaux, et non superficiels, semble impossible à assumer. Mais nous ne savons jamais si quelque chose est possible tant que nous n'avons pas réussi. En attendant, notre préoccupation est de trouver les stratégies de résistance et d'organisation les plus efficaces.

Heureusement, l'histoire de la résistance est aussi longue que l'histoire de l'oppression, et nous avons donc de nombreux exemples dont nous pouvons nous inspirer. Pour améliorer nos propres efforts en vue de réaliser la révolution, nous devrions examiner comment les activistes à travers l'histoire ont réussi à affronter le pouvoir et à produire des changements et comment ils ont été inefficaces, tout en gardant à l'esprit leur contexte spécifique.

Dans l'histoire des États-Unis, le syndicat occupe une place traditionnelle en tant que vecteur de l'activité révolutionnaire. Au début du vingtième siècle, les syndicats ont proposé une critique radicale des inégalités sociales et ont offert aux esclaves salariés du pays la promesse d'une vie meilleure. Les syndicats sont devenus une force politique puissante, gagnant des millions de membres, organisant des grèves et des manifestations, et créant également des comités de défense lorsque la police a commencé à massacrer les travailleurs en grève. Bien qu'ils soient parvenus à réduire certaines des brutalités auxquelles les travailleurs étaient confrontés, les syndicats n'ont pas réussi à corriger les inégalités sociales sous-jacentes et ont fini par trahir les travailleurs. Aujourd'hui, la plupart des syndicats sont des rackets à l'esprit étroit qui n'ont que peu d'influence réelle. L'un des facteurs importants de leur échec est la structure hiérarchique de la plupart des syndicats. La hiérarchie s'est développée pour permettre à des groupes d'élite de contrôler des populations plus importantes. En conséquence, les organisations hiérarchiques sont facilement détournées par les gouvernements qu'elles contestent. Les syndicats ont été infiltrés et leurs dirigeants ont été cooptés. Les dirigeants syndicaux ont facilement confondu les intérêts de leur organisation avec les intérêts de la lutte sociale pour laquelle les syndicats avaient été créés. Les activités syndicales radicales étant sévèrement réprimées, les dirigeants syndicaux ont développé des relations plus coopératives avec les politiciens et les patrons afin d'assurer la survie de leur syndicat et le maintien de leurs positions de pouvoir de plus en plus confortables. Les syndicalistes radicaux qui ne pouvaient pas être achetés ont été emprisonnés ou neutralisés d'une manière ou d'une autre.

Une autre faiblesse majeure de la plupart des syndicats était leur rejet des questions de race et de genre, qui étaient inséparables des questions économiques. En refusant de s'attaquer au racisme, au sexisme et à la xénophobie, et en maintenant au contraire une critique privilégiée et étroitement économique du capitalisme, les syndicats sont devenus des organisations d'hommes blancs, perdant le soutien vital des ouvrières de l'habillement et des employées de maison, des métayers noirs et des travailleurs immigrés dans les usines. Leur incapacité à critiquer les aspects suprémacistes du capitalisme a permis aux patrons de conserver le pouvoir en divisant et en privant les travailleurs de leurs moyens d'action, en faisant des étrangers et des personnes noires émancipées les boucs émissaires de leur pauvreté.

C'est en partie le désir des grands syndicats d'être respectables qui les a conduits à perpétuer les comportements racistes, sexistes et élitistes de la structure de pouvoir qu'ils cherchaient à l'origine à vaincre. Leurs positions d'autorité et les négociateurs du gouvernement ont fait miroiter le pouvoir - le confort, la dignité et le respect - aux dirigeants syndicaux, qui ont fini par oublier les causes des maux sociaux qu'ils dénonçaient, et se sont plutôt appuyés sur la satisfaction d'être acceptés par la société (la haute société) pour endormir les symptômes. En refusant les compromis, en utilisant des tactiques radicales ou militantes, ou en remettant en cause le statu quo racial et sexuel, ils savaient qu'ils seraient ostracisés par le gouvernement et vilipendés par les médias. Les syndicats se sont donc efforcés de devenir respectables aux yeux du grand public, et comme ce qui est grand public est déterminé par les médias, cela signifiait qu'ils devaient plaire aux classes moyennes et supérieures blanches. Ce faisant, les syndicats ont dû renoncer à leur plus grande source de force, la détermination des opprimés à gagner leur liberté, qui se manifeste souvent par une rage inconvenante pour ceux qui ont beaucoup à perdre à ce que des mécontents fassent tanguer le bateau.

Malgré les échecs historiques des syndicats, tant que le travail salarié sera répandu dans la société et dans la vie de l'individu, la relation entre le travailleur et le patron sera un point nodal important pour l'agitation. L'Industrial Workers of the World, un syndicat qui recherche le contrôle des travailleurs sur les moyens de production et l'abolition finale du capitalisme, a fait preuve d'un esprit anti-autoritaire plus résistant que ses contemporains, qui ne font aujourd'hui guère plus que fournir des tampons au parti démocrate.

Récemment, de nombreux militants luttant contre l'oppression ne s'affilient pas à une seule organisation, mais s'efforcent de dénoncer et d'atténuer l'oppression là où elle est le plus durement ressentie. Souvent, les radicaux privilégiés qui se méfient du réformisme sont réticents à travailler pour une cause qui n'a pas d'objectifs révolutionnaires clairement articulés et à long terme, et ils rejoignent donc des organisations plus abstraites qui sont orientées au niveau national ou mondial, plutôt que local. Cependant, les pauvres et les personnes de couleur n'ont pas besoin de sortir de leur propre communauté pour trouver des brutalités et des dépravations qui doivent être surmontées. En conséquence, les radicaux des groupes privilégiés seront séparés des radicaux des groupes ciblés par l'oppression. Les militants masculins blancs de la classe moyenne doivent réaliser que les programmes de lecture, les cliniques du SIDA, les soupes populaires, les refuges pour personnes sans domicile fixe, les refuges pour femmes battues, les programmes de surveillance des flics et les groupes de soutien aux prisonniers, ainsi que d'autres programmes de « premiers secours » peuvent être révolutionnaires et, plus important encore, qu'ils sont nécessaires à la santé et à la survie des communautés opprimées.

Certaines organisations nationales, telles que Food Not Bombs ou Homes Not Jails, combinent les efforts visant à traiter directement les symptômes de l'oppression avec une mise en cause radicale des structures de pouvoir qui sont à l'origine de ces symptômes. Food Not Bombs sert des repas gratuits dans des lieux publics, invitant à la prise de conscience de problèmes tels que la faim et la pauvreté, et remettant en question les causes de ces problèmes. Homes Not Jails squatte et répare des appartements abandonnés, condamnés ou vacants, en violation des « droits de propriété » des propriétaires, afin d'offrir un toit aux familles sans domicile fixe. En recourant à l'action directe illégale et à la désobéissance civile, ils illustrent la manière dont le système juridique protège les propriétaires aux dépens des pauvres et mettent en évidence le rôle du gouvernement et du capitalisme dans la création et le maintien de la pauvreté. Il convient de noter que ces groupes sont organisés de manière décentralisée et non hiérarchique. Food Not Bombs, par exemple, est plus une idée qu'une institution. N'importe qui, n'importe où, peut créer une section de Food Not Bombs, sans avoir à demander l'autorisation du siège national (il n'y en a pas) ni à payer de cotisation. En conséquence, les membres de chaque chapitre peuvent adapter le modèle Food Not Bombs aux conditions et aux besoins locaux, et sans politique institutionnelle ni conférence nationale, les membres ne perdent pas d'efforts dans le maintien de l'organisation et peuvent consacrer plus de temps à répondre aux besoins locaux. Cependant, comme Food Not Bombs est en grande partie le produit de cercles d'activistes blancs privilégiés de la classe moyenne, de nombreux chapitres s'enlisent dans un schéma consistant à fournir un repas hebdomadaire gratuit symbolique et à ne pas aller plus loin dans la lutte contre la faim. La plupart des membres de Food Not Bombs ne connaissent pas personnellement la faim, et il semble qu'au moins certains d'entre eux aient l'idée qu'en fournissant un service aux personnes pauvres et opprimées de la communauté, ils les « radicaliseront », créeront des alliances et déclencheront une masse critique, puis tout le monde se soulèvera en révolution, d'une manière vague et magique. Si, au lieu de reprocher inconsciemment aux opprimés (qu'ils ont été formés depuis leur naissance à considérer comme des ignorants) de ne pas s'engager dans la lutte contre le « militarisme » et le « capitalisme », ils décidaient d'augmenter continuellement le niveau de la lutte contre la faim, au-delà d'un repas par semaine, ils pourraient peut-être constater qu'il n'y a pas eu de changement dans la façon dont les opprimés ont été traités, ils découvriront peut-être qu'il n'y a pas de moyen plus efficace de lutter contre le capitalisme et, dans le même temps, de soulager les symptômes de ceux qui en souffrent le plus, car le capitalisme ne peut tout simplement pas fonctionner si la faim n'est pas une menace imminente qui motive les gens à travailler pour le profit d'autrui.

Les personnes qui luttent contre l'oppression continuent d'être confrontées à de nombreux problèmes et à des lacunes dans leurs méthodes. Il est évident que nous devons rester flexibles et nous adapter à notre situation spécifique ; il n'existe pas de programme en douze étapes pour la révolution. Mais certaines erreurs sont suffisamment courantes pour que nous puissions établir des modèles et les éviter. Pour être efficace, une organisation ou un mouvement doit prendre plusieurs mesures fondamentales :

\emph{Remettre en question les} comportements oppressifs et privilégiés intériorisés et agir de manière inclusive, sans se plier aux opinions dominantes (et en fin de compte élitistes).

\emph{Identifier} la nature fondamentale de l'oppression au sein du système et fournir une critique radicale ou un ensemble d'objectifs.

\emph{Baser la} lutte sur des segments moins privilégiés et plus opprimés de la société, plutôt que d'essayer d'établir un lien avec le courant dominant de la classe moyenne.

\emph{S'organiser} de manière localisée, non hiérarchique, décentralisée et autonomiste, afin de promouvoir l'égalité et l'épanouissement au sein du groupe, de créer une plus grande flexibilité et une meilleure adaptation aux conditions locales, et de se protéger contre la répression et l'infiltration de l'État.

\chapter*{\textbf{Dixième Partie}}\hypertarget{dixime-partie}{}\label{dixime-partie}
\markboth{Dixième Partie}{Dixième Partie}

Envisager un modèle utopique pour le monde entier serait irréaliste et culturellement biaisé, même autoritaire. Chacun doit faire ses propres recherches et parvenir à ses propres conclusions sur le mode de vie qui lui convient le mieux. L'exigence minimale est que nous ne devrions tolérer aucun système qui impose un modèle « correct » à de nombreuses personnes, indépendamment de leur volonté. L'histoire regorge d'exemples (partiellement occultés) d'autres formes d'organisation que nous pouvons utiliser pour déterminer l'organisation la mieux adaptée et la plus réaliste pour répondre à nos besoins actuels.

Chaque communauté devrait décider elle-même des questions d'organisation sociale et économique et s'associer à d'autres communautés dans des associations volontaires pour répondre aux besoins qui ne peuvent être satisfaits par une seule communauté. En attendant, nous avons tous beaucoup en commun et devons lutter ensemble contre le système d'exploitation et de contrôle généralisé à l'échelle mondiale. Ce n'est qu'en détruisant le système d'oppression, quels que soient sa forme et son nom, et en mettant fin au continuum, que nous pourrons ouvrir la voie à une autre forme de lutte : construire des sociétés qui assurent la protection et la subsistance sans recourir à la coercition ni créer de nouveaux systèmes d'oppression.




\backmatter
\tableofcontents
 
\end{document}

L'une des objections les plus courantes au communisme est que les gens ne sont pas assez bons pour vivre dans un état de choses communiste. Ils ne se soumettraient pas à un communisme imposé, mais ils ne sont pas encore assez mûrs pour un communisme libre et anarchique. Des siècles d'éducation individualiste les ont rendus trop égoïstes. L'esclavage, la soumission au plus fort et le travail sous le fouet de la nécessité les ont rendus inaptes à une société où chacun serait libre et ne connaîtrait d'autre contrainte que celle qui résulte d'un engagement volontairement pris envers les autres, et de leur désapprobation . C'est pourquoi, nous dit-on, un état intermédiaire de transition de la société est nécessaire comme étape vers le communisme.

Des mots anciens sous une forme nouvelle, des mots dits et répétés depuis la première tentative de réforme, politique ou sociale, dans toute société humaine. Des mots que nous avons entendus avant l'abolition de l'esclavage ; des mots prononcés il y a vingt et quarante siècles par ceux qui aiment trop leur propre tranquillité pour aimer les changements rapides, que l'audace de la pensée effraie, et qui eux-mêmes n'ont pas assez souffert des iniquités de la société actuelle pour sentir la nécessité profonde de nouveaux problèmes !

Les gens ne sont pas assez bons pour le communisme, mais le sont-ils pour le capitalisme ? Si tout le monde était bon, gentil et juste, ils ne s'exploiteraient jamais les uns les autres, même en ayant les moyens. Avec de tels personnes, la propriété privée du capital ne serait point un danger. Le capitaliste s'empresserait de partager ses profits avec les travailleurs, et les travailleurs les mieux rémunérés avec ceux qui souffrent de problèmes occasionnels. Si les gens étaient prévoyants, ils ne produiraient pas du velours et des marchandises de luxe pendant que l'on manque de nourriture dans les chaumières ; ils ne construiraient pas des palais tant qu'il y aura des taudis.

Si les gens avaient un sentiment d'équité profondément développé, ils n'opprimeraient pas d'autres personnes. Les politiciens ne tromperaient pas leurs électeurs ; le Parlement ne serait pas une boîte à bavardages et à tricheries, et les policiers de Charles Warren refuseraient de matraquer les orateurs et les auditeurs de Trafalgar Square. Et si les gens étaient galants, respectueux d'eux-mêmes et moins égoïstes, même un mauvais capitaliste ne serait pas un danger ; les ouvriers l'auraient bientôt réduit au rôle de simple « camarade-gérant ». Même un roi ne serait pas dangereux, car le peuple le considérerait simplement comme un type incapable de faire mieux, et donc chargé de signer quelques papiers stupides envoyés à d'autres grincheux qui se disent rois.

Mais les gens ne sont pas ces compagnons libres d'esprit, indépendants, prévoyants, aimants et compatissants que nous aimerions voir. C'est pourquoi, précisément, ils ne doivent pas continuer à vivre dans le système actuel qui leur permet de s'opprimer et de s'exploiter les uns les autres. Prenons, par exemple, ces tailleurs misérables qui ont défilé dimanche dernier dans les rues, et supposons que l'un d'entre eux ait hérité de cent livres d'un oncle américain. Avec ces cent livres, il ne va certainement pas créer une association productive pour une douzaine de tailleurs aussi misérables que lui et essayer d'améliorer leur condition. Il deviendra un \emph{sweater}\footnote{Mot désignant un patron qui surexploite des travailleurs en payant peu et demandant beaucoup de travail, littéralement les faisant « suer ».}. Nous disons donc que dans une société où les gens sont aussi mauvais que cet héritier américain, il est très difficile pour lui d'avoir des tailleurs misérables autour de lui. Dès qu'il le pourra, il les fera suer ; tandis que si ces mêmes tailleurs avaient un gagne-pain assuré, aucun d'eux ne suerait pour enrichir son ex-camarade, et le jeune \emph{sweater} ne deviendrait pas lui-même la très mauvaise bête qu'il deviendra sûrement s'il continue à être un \emph{sweater}.

On nous dit que nous sommes trop serviles, trop arrogants, pour être placés sous des institutions libres ; mais nous disons que, parce que nous sommes effectivement si serviles, nous ne devrions pas rester plus longtemps sous les institutions actuelles, qui favorisent le développement de la servilité. Nous voyons que les Britanniques, les Français et les Américains font preuve de la servilité la plus dégoûtante à l'égard de Gladstone, Boulanger ou Gould. Et nous concluons que dans une humanité déjà dotée de tels instincts serviles, il est très mauvais que les masses soient privées d'une éducation supérieure et contraintes de vivre dans l'inégalité actuelle de la richesse, de l'éducation et du savoir. L'instruction supérieure et l'égalité des conditions seraient les seuls moyens de détruire les instincts serviles hérités, et nous ne pouvons comprendre comment les instincts serviles peuvent servir d'argument pour maintenir, même un jour de plus, l'inégalité des conditions, pour refuser l'égalité d'instruction à tous les membres de la communauté.

Notre espace est limité, mais soumettez à la même analyse n'importe quel aspect de notre vie sociale, et vous verrez que le système capitaliste et autoritaire actuel est absolument inadapté à une société de personnes aussi négligentes, aussi avares, aussi égoïstes et aussi serviles qu'elles le sont aujourd'hui. C'est pourquoi, lorsque nous entendons des gens dire que les anarchistes imaginent les humains bien meilleurs qu'ils ne le sont en réalité, nous nous demandons simplement comment des gens intelligents peuvent répéter une telle absurdité. Ne répétons-nous pas sans cesse que le seul moyen de rendre les gens à la fois moins avares et égoïstes, moins cupides et moins serviles, c'est d'éliminer les conditions qui favorisent le développement de l'égoïsme et de l’avarice, de la servilité et de la cupidité ? La seule différence entre nous et ceux qui font l'objection ci-dessus est la suivante : Nous n'exagérons pas, comme eux, les instincts inférieurs des masses, et nous ne fermons pas complaisamment les yeux sur les mêmes mauvais instincts dans les classes supérieures. Nous soutenons que les gouvernants et les gouvernés sont \emph{tous les deux} gâtés par l'autorité ; que les exploiteurs et les exploités sont \emph{tous les deux} gâtés par l'exploitation ; tandis que nos adversaires semblent admettre qu'il existe une sorte de crème de la crème - les gouvernants, les employeurs, les dirigeants - qui, heureusement, empêchent ces mauvaises personnes- les gouvernés, les exploités, les dirigés - de devenir encore pires qu'ils ne le sont.

Voilà la différence, et qu’est-ce qu’elle est importante ! \emph{Nous} admettons les imperfections de la nature humaine, mais nous ne faisons pas d'exception pour les dirigeants. \emph{Ils} en font, même si c'est parfois inconscient, et parce que nous ne faisons pas cette exception, ils disent que nous sommes des rêveurs, des « gens peu pratiques ».

Une vieille querelle, celle qui oppose les « gens pratiques », « pragmatiques » aux « gens peu pratiques », « utopistes » : une querelle renouvelée à chaque changement proposé, et qui se termine toujours par la défaite totale de ceux qui se nomment eux-mêmes « pragmatiques ».

Beaucoup d'entre nous doivent se souvenir de la querelle qui a fait rage en Amérique avant l'abolition de l'esclavage. Lorsque l'émancipation totale des personnes noires a été préconisée, les gens pratiques avaient l'habitude de dire que si les personnes noires n'étaient plus contraints de travailler sous les fouets de leurs propriétaires, ils ne travailleraient plus du tout et deviendraient bientôt une charge pour la communauté. On pouvait interdire les fouets épais, disaient-ils, et l'épaisseur des fouets pouvait être progressivement réduite par la loi à un demi-pouce d'abord, puis à une simple bagatelle de quelques dixièmes de pouce ; mais il fallait conserver un certain type de fouet. Et lorsque les abolitionnistes dirent - tout comme nous le disons aujourd'hui - que la jouissance du produit de son travail serait une incitation au travail bien plus puissante que le fouet le plus épais, on leur répondit - tout comme on nous le dit aujourd'hui - « C'est absurde, mon ami. Vous ne connaissez pas la nature humaine ! Des années d'esclavage les ont rendus négligents, paresseux et serviles, et la nature humaine ne peut être changée en un jour. Vous êtes bien sûr animé des meilleures intentions, mais vous n'êtes pas du tout pragmatique. ».

Pendant un certain temps, les pragmatistes ont pu élaborer des projets d'émancipation progressive des personnes noires. Mais, hélas, ces plans se sont révélés peu pratiques et la guerre civile - la plus sanglante jamais enregistrée - a éclaté. Mais la guerre a abouti à l'abolition de l'esclavage, sans aucune période de transition ; - et voyez, aucune des terribles conséquences prévues par les gens pratiques n'a suivi. Les esclaves libérées travaillent, ils sont industrieux et laborieux, ils sont prévoyants - oui, très prévoyants, en fait - et le seul regret que l'on puisse exprimer est que le projet préconisé par l'aile gauche du camp anti-pragmatique - égalité totale et redistribution de terres - n'ait pas été réalisé : il aurait permis d'éviter bien des problèmes aujourd'hui.

À peu près à la même époque, une querelle similaire faisait rage en Russie, et sa cause était la suivante. La Russie comptait 20 millions de serfs. Depuis des générations, ils étaient soumis à l'autorité, ou plutôt au fouet, de leurs propriétaires. Ils étaient fouettés pour avoir mal cultivé leur sol, pour avoir manqué de propreté dans leur foyer, pour avoir mal tissé leurs étoffes, pour n'avoir pas marié plus tôt leurs enfants - fouettés pour tout. L'avarice, la négligence étaient leurs caractéristiques réputées.

Les utopistes sont arrivés et n'ont rien demandé d'autre que ce qui suit : La libération complète des serfs ; l'abolition immédiate de toute obligation du serf envers le seigneur. Plus encore : abolition immédiate de la juridiction du seigneur et abandon de toutes les affaires sur lesquelles il jugeait auparavant, au profit de tribunaux paysans élus par les paysans et jugeant, non pas selon la loi qu'ils ne connaissent pas, mais selon leurs coutumes non écrites. Tel était le projet utopiste du camp utopiste, de ceux dont les idées étaient impraticables. Il était considéré comme une simple folie par les gens pratiques et pragmatiques.

Mais heureusement, il y avait à cette époque en Russie une bonne dose d'impraticabilité dans l'air, et elle était entretenue par l'impraticabilité des paysans, qui se révoltaient avec des bâtons contre des fusils, et refusaient de se soumettre, malgré les massacres, et renforçaient ainsi l'état d'esprit impraticable au point de permettre au camp impraticable de forcer le tsar à signer leur projet - encore mutilé dans une certaine mesure. Les gens les plus pragmatiques se sont empressés de fuir la Russie pour ne pas être égorgés quelques jours après la promulgation de ce projet impraticable.

Mais tout se passa très bien, malgré les nombreuses maladresses que commettaient encore les pragmatistes. Ces esclaves réputés négligents, égoïstes, brutes, etc., firent preuve d'un tel bon sens, d'une telle capacité d'organisation qu'ils dépassèrent les espérances des utopistes les plus impraticables ; et trois ans après l'émancipation, la physionomie générale des villages avait complètement changé. Les esclaves devenaient des Humains !

Les utopistes ont gagné la bataille. Ils ont prouvé qu'\emph{ils} étaient les gens vraiment pratiques, et que ceux qui prétendaient être pragmatiques n’étaient que des imbéciles. Et le seul regret exprimé aujourd'hui par tous ceux qui connaissent la paysannerie russe est que trop de concessions aient été faites à ces imbéciles pragmatistes et à ces égoïstes à l’esprit borné : que les conseils de la gauche du camp utopiste n'aient pas été suivis dans leur intégralité.

Nous ne pouvons pas donner plus d'exemples. Mais nous invitons sincèrement ceux qui aiment raisonner par eux-mêmes à étudier l'histoire de n'importe lequel des grands changements sociaux qui se sont produits dans l'humanité depuis l'avènement des Communes jusqu'à la Réforme et aux temps modernes. Ils verront que l'histoire n'est rien d'autre qu'une lutte entre les gouvernants et les gouvernés, les oppresseurs et les opprimés, dans laquelle le camp pratique et pragmatiste se range toujours du côté des gouvernants et des oppresseurs, tandis que le camp non pratique et utopiste se range du côté des opprimés ; et ils verront que la lutte se termine toujours par une défaite finale du camp pratique après beaucoup de sang versé et de souffrances, en raison de ce qu'ils appellent leur « bon sens pratique ».

Si, en disant que nous ne sommes pas pratiques, nos adversaires veulent dire que nous prévoyons mieux la marche des événements que les lâches gens pratiques à courte vue, alors ils ont raison. Mais s'ils veulent dire qu'eux, les pragmatistes, les gens « pratiques », ont une meilleure vision des événements, alors nous les renvoyons à l'histoire et leur demandons de se mettre en accord avec leurs enseignements avant de faire cette affirmation présomptueuse.


L'ancien mouvement ouvrier est organisé en partis. La croyance dans les partis est la principale raison de l'impuissance de la classe ouvrière ; c'est pourquoi nous évitons de former un nouveau parti - non pas parce que nous sommes trop peu nombreux, mais parce qu'un parti est une organisation qui vise à diriger et à contrôler la classe ouvrière. À l'inverse, nous soutenons que la classe ouvrière ne peut remporter la victoire que lorsqu'elle s'attaque de manière indépendante à ses problèmes et décide de son propre destin. Les travailleurs ne doivent pas accepter aveuglément les mots d’ordre des autres, ni ceux de nos propres groupes, mais doivent penser, agir et décider par eux-mêmes. Cette conception est en contradiction flagrante avec la tradition du parti en tant que moyen le plus important d'éduquer le prolétariat. C'est pourquoi beaucoup, tout en répudiant les partis socialiste et communiste, nous résistent et s'opposent à nous. Cela est dû en partie à leurs concepts traditionnels, car après avoir considéré la lutte des classes comme une lutte des partis, il devient difficile de la considérer comme la lutte de la classe ouvrière, comme une lutte de classe. Mais cette conception repose en partie sur l'idée que le parti joue néanmoins un rôle essentiel et important dans la lutte du prolétariat. Examinons cette dernière idée de plus près.

Essentiellement, le parti est un groupement selon des vues, des conceptions ; les classes sont des groupements selon des intérêts économiques. L'appartenance à une classe est déterminée par le rôle que l'on joue dans le processus de production ; l'appartenance à un parti est l'adhésion de personnes qui s'accordent dans leurs conceptions des problèmes sociaux. Autrefois, on pensait que cette contradiction disparaîtrait dans le parti de classe, le parti « ouvrier ». Lors de la montée en puissance de la social-démocratie, il semblait que le parti engloberait progressivement l'ensemble de la classe ouvrière, en partie en tant que membres, en partie en tant que sympathisants. Comme la théorie marxienne déclare que des intérêts similaires engendrent des points de vue et des objectifs similaires, on s'attendait à ce que la contradiction entre le parti et la classe disparaisse progressivement. L'histoire a prouvé le contraire. La social-démocratie est restée minoritaire, d'autres groupes de la classe ouvrière se sont organisés contre elle, des sections se sont séparées d'elle, et son propre caractère a changé. Son propre programme a été révisé ou réinterprété. L'évolution de la société ne se fait pas sur une ligne lisse et régulière, mais dans le cadre de conflits et de contradictions.

Avec l'intensification de la lutte des travailleurs, la puissance de l'ennemi augmente également et assaille les travailleurs de nouveaux doutes et de nouvelles craintes quant à la meilleure voie à suivre. Et chaque doute entraîne des scissions, des contradictions et des batailles fractionnelles au sein du mouvement ouvrier. Il est vain de se lamenter sur ces conflits et ces scissions qui divisent et affaiblissent la classe ouvrière. La classe ouvrière n'est pas faible parce qu'elle est divisée - elle est divisée parce qu'elle est faible. Parce que l'ennemi est puissant et que les anciennes méthodes de guerre s'avèrent inefficaces, la classe ouvrière doit chercher de nouvelles méthodes. Sa tâche n'apparaîtra pas clairement à la suite d'une illumination venue d'en haut. Elle doit découvrir ses tâches par un travail acharné, par la réflexion et le conflit d'opinions. Elle doit trouver sa propre voie, d'où la lutte interne. Elle doit abandonner ses vieilles idées et illusions et en adopter de nouvelles, et c'est parce que cela est difficile que l'ampleur et la gravité des scissions s'expliquent.

Nous ne pouvons pas non plus nous bercer d'illusions en croyant que cette période de luttes partisanes et idéologiques n'est que temporaire et qu'elle fera place à une nouvelle harmonie. Il est vrai qu'au cours de la lutte des classes, il y a des occasions où toutes les forces s'unissent autour d'un grand objectif réalisable et où la révolution se poursuit avec la force d'une classe ouvrière unie. Mais après cela, comme après chaque victoire, il y a des divergences sur la question : que faire ensuite ? Et même si la classe ouvrière est victorieuse, elle est toujours confrontée à la tâche la plus difficile qui soit : soumettre davantage l'ennemi, réorganiser la production, créer un nouvel ordre. Il est impossible que tous les travailleurs, tous les niveaux et tous les groupes, avec leurs intérêts souvent encore différents, soient, à ce stade, d'accord sur toutes les questions et prêts à une action ultérieure unie et décisive. Ce n'est qu'après les controverses et les conflits les plus vifs qu'ils trouveront la véritable voie à suivre, et ce n'est qu'ainsi qu'ils parviendront à la clarté.

Si, dans cette situation, des personnes ayant les mêmes conceptions fondamentales s’unissent pour discuter des mesures pratiques et cherchent à clarifier leurs conclusions par des discussions et en faisant de la propagande, de tels groupes pourraient être appelés partis, mais ils seraient des partis dans un sens tout à fait différent de celui d’aujourd’hui. L'action, la lutte de classe proprement dite, est la tâche des masses ouvrières elles-mêmes, dans leur ensemble, dans leurs groupements réels d'ouvriers d'usines et de filature, ou d'autres groupes productifs, parce que l'histoire et l'économie les ont placées dans une situation où elles doivent et peuvent mener la lutte de la classe ouvrière. Il serait insensé que les partisans d'un parti se mettent en grève tandis que ceux d'un autre continuent à travailler. Mais les deux tendances défendront leur position de grève ou de non-grève dans les assemblées d'usine, ce qui leur donnera l'occasion de parvenir à une décision bien fondée. La lutte est si grande, l'ennemi si puissant que seules les masses dans leur ensemble peuvent remporter la victoire, résultat de la force matérielle et morale de l'action, de l'unité et de l'enthousiasme, mais aussi résultat de la force mentale de la pensée, de la clarté. C'est là la grande importance de ces partis ou groupements d'opinion : ils apportent de la clarté dans leurs conflits, leurs discussions et leur propagande. Ils sont les organes de l'auto-illumination de la classe ouvrière grâce à laquelle les ouvriers trouvent le chemin de la liberté.

Bien entendu, ces partis ne sont pas statiques et immuables. Chaque situation nouvelle, chaque problème nouveau amène les esprits à diverger et à se regrouper en de nouveaux groupes avec de nouveaux programmes. Ils ont un caractère fluctuant et s'adaptent constamment aux situations nouvelles.

Les partis ouvriers actuels ont un caractère tout à fait différent de ces groupes, car ils ont un objectif différent : ils veulent s'emparer du pouvoir. Ils ne veulent pas aider la classe ouvrière dans sa lutte pour l'émancipation, mais la diriger eux-mêmes et proclamer que c'est là l'émancipation du prolétariat. La social-démocratie qui est née à l'époque du parlementarisme a conçu ce pouvoir comme un gouvernement parlementaire. Le Parti communiste a poussé l'idée du pouvoir du parti jusqu'à son extrême dans la dictature du parti.

Ces partis, à la différence des groupes décrits ci-dessus, doivent être des structures rigides avec des lignes de démarcation claires par des cartes de membre, des statuts, une discipline de parti et des procédures d'admission et d'exclusion. Car ce sont des instruments du pouvoir : ils luttent pour le pouvoir, brident leurs membres par la force et cherchent constamment à étendre leur champ d'action. Leur tâche n'est pas de développer l'initiative des travailleurs, mais plutôt de former des membres fidèles et inconditionnels de leur foi. Alors que la classe ouvrière, dans sa lutte pour le pouvoir et la victoire, a besoin d'une liberté intellectuelle illimitée, le pouvoir du parti doit réprimer toutes les opinions, sauf la sienne. Dans les partis « démocratiques », la répression est voilée ; dans les partis dictatoriaux, elle est ouverte et brutale.

De nombreux travailleurs se rendent déjà compte que le pouvoir du parti socialiste ou communiste ne sera que la forme cachée du pouvoir de la classe bourgeoise dans lequel l'exploitation et la suppression de la classe ouvrière demeurent. Au lieu de ces partis, ils préconisent la formation d'un « parti révolutionnaire » qui visera réellement à la domination des travailleurs et à la réalisation du communisme. Il ne s'agit pas d'un parti au sens nouveau tel que décrit ci-dessus, mais d'un parti comme ceux d'aujourd'hui, qui luttent pour le pouvoir en tant qu'« avant-garde » de la classe, en tant qu'organisation de minorités conscientes et révolutionnaires, qui s'emparent du pouvoir afin de l'utiliser pour l'émancipation de la classe.

Nous affirmons qu'il y a une contradiction interne dans le terme « parti révolutionnaire » : « parti révolutionnaire ». Un tel parti ne peut pas être révolutionnaire. Il n'est pas plus révolutionnaire que ne l'étaient les créateurs du Troisième Reich. Quand nous parlons de révolution, nous parlons de la révolution prolétarienne, de la prise du pouvoir par la classe ouvrière elle-même.

Le « parti révolutionnaire » repose sur l'idée que la classe ouvrière a besoin d'un nouveau groupe de dirigeants qui vainquent la bourgeoisie pour les travailleurs et construisent un nouveau gouvernement - (notez que la classe ouvrière n'est pas encore considérée comme apte à réorganiser et à réguler la production). Mais n'est-ce pas ainsi qu'il devrait en être ? Puisque la classe ouvrière ne semble pas capable de faire la révolution, n'est-il pas nécessaire que l'avant-garde révolutionnaire, le parti, fasse la révolution à sa place ? Et cela n'est-il pas vrai tant que les masses supportent volontairement le capitalisme ?

Face à cela, nous soulevons la question suivante : quelle force un tel parti peut-il mettre en place pour la révolution ? Comment peut-il vaincre la classe capitaliste ? Seulement si les masses le soutiennent. Seulement si les masses se lèvent et, par des attaques de masse, des luttes de masse et des grèves de masse, renversent l'ancien régime. Sans l'action des masses, il ne peut y avoir de révolution.

Deux choses peuvent en résulter. Les masses restent dans l'action : elles ne rentrent pas chez elles et ne laissent pas le gouvernement au nouveau parti. Elles organisent leur pouvoir dans les usines et les ateliers et se préparent à un nouveau conflit pour vaincre le capital ; par l'intermédiaire des conseils ouvriers, elles créent un syndicat type pour prendre en charge la direction complète de toute la société ; en d'autres termes, elles prouvent qu'elles ne sont pas aussi incapables de révolution qu'il y paraît. Il est donc nécessaire qu'un conflit surgisse avec le parti qui veut lui-même prendre le contrôle et qui ne voit dans l'action autonome de la classe ouvrière que le désordre et l'anarchie. Il est possible que les ouvriers développent leur mouvement et balayent le parti. Ou bien le parti, avec l'aide des éléments bourgeois, vainc les ouvriers. Dans les deux cas, le parti est un obstacle à la révolution parce qu'il veut être plus qu'un moyen de propagande et d'information ; parce qu'il se sent appelé à diriger et à gouverner en tant que parti.

D'un autre côté, les masses peuvent suivre la foi du parti et lui laisser la pleine direction des affaires. Ils suivent les mots d’ordre d’en haut, ils ont confiance dans le nouveau gouvernement (comme en Allemagne et en Russie) qui doit réaliser le communisme, et ils retournent chez eux et se mettent au travail. Aussitôt, la bourgeoisie exerce toute sa puissance de classe dont les racines ne sont pas brisées : ses forces financières, ses grandes ressources intellectuelles, sa puissance économique dans les usines et les grandes entreprises. Face à cela, le parti gouvernemental est trop faible. Ce n’est qu’en faisant preuve de modération, de concessions et de concessions qu’il peut soutenir que c’est une folie pour les ouvriers de vouloir imposer des revendications impossibles. Ainsi, le parti privé de pouvoir de classe devient l’instrument du maintien du pouvoir bourgeois.

Nous avons déjà dit que le terme « parti révolutionnaire » était contradictoire d'un point de vue prolétarien. Nous pouvons l'énoncer autrement : dans le terme « parti révolutionnaire », « révolutionnaire » signifie toujours une révolution bourgeoise. Toujours, lorsque les masses renversent un gouvernement puis permettent à un nouveau parti de prendre le pouvoir, nous avons une révolution bourgeoise - la substitution d'une caste dirigeante par une nouvelle caste dirigeante. Ce fut le cas à Paris en 1830 lorsque la bourgeoisie financière supplanta les propriétaires terriens, en 1848 lorsque la bourgeoisie industrielle prit les rênes.

Dans la révolution russe, la bureaucratie du parti est arrivée au pouvoir en tant que caste dirigeante. Mais en Europe occidentale et en Amérique, la bourgeoisie est beaucoup plus puissamment enracinée dans les usines et les banques, de sorte qu'une bureaucratie de parti ne peut pas les écarter aussi facilement. La bourgeoisie dans ces pays ne peut être vaincue que par des actions répétées et unies des masses dans lesquelles elles s'emparent des moulins et des usines et construisent leurs organisations de conseil.

Ceux qui parlent de « partis révolutionnaires » tirent des conclusions incomplètes et limitées de l'histoire. Lorsque les partis socialistes et communistes sont devenus des organes du pouvoir bourgeois pour la perpétuation de l'exploitation, ces gens bien intentionnés ont simplement conclu qu'ils devraient faire mieux. Ils ne peuvent pas comprendre que l'échec de ces partis est dû au conflit fondamental entre l'auto-émancipation de la classe ouvrière par son propre pouvoir et le pacification de la révolution par une nouvelle clique dirigeante sympathique. Ils pensent être l'avant-garde révolutionnaire parce qu'ils voient les masses indifférentes et inactives. Mais les masses sont inactives seulement parce qu'elles ne peuvent pas encore comprendre le cours de la lutte et l'unité des intérêts de classe, bien qu'elles sentent instinctivement le grand pouvoir de l'ennemi et l'immensité de leur tâche. Une fois que les conditions les obligeront à agir, elles attaqueront la tâche de l'auto-organisation et la conquête du pouvoir économique du capital.


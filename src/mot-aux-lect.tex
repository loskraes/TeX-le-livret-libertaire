\chapter*{Mot aux lecteuricexs}

Le Livret Libertaire est une petite tentative pour apporter la théorie libertaire qui n'est pas largement accessible en français. Nous croyons fermement qu'une bonne action, une action concrète, une action productive découle nécessairement d'une stratégie bien réfléchie, d'une compréhension du monde tel qu'il est réellement, d'une analyse systématique et critique - ainsi, nous avons souhaité apporter quelques outils à ceux qui sont toujours à la recherche de nouveaux instruments.

Nous avons donc choisi des textes que nous considérons comme intéressants, pertinents et importants pour la compréhension des théories libertaires. Les questions des croyances anarchistes fondamentales, des analyses de l'appareil d'État moderne, et des soucis de l'organisation libertaire sont abordées ici, et ce recueil contient à la fois des textes utiles pour les nouveaux libertaires et pour ceux qui sont convaincus depuis longtemps. Peut-être que le texte de Graeber « Es-tu anarchiste ? » te donnera les mots justes pour articuler tes points de vue anarchistes à ton oncle qui parfois s’approche timidement de l’anarchisme ou le communisme, ou que le texte peu connu de Kropotkin « Sommes-nous assez bons ? » répondra à la question de quelqu'un « Comment une société anarchiste peut-elle fonctionner alors que l'humanité est si corrompue ? Peut-être même que Pannekoek et la Fédération italienne des communistes anarchistes te donneront les clés pour résoudre les problèmes de ton collectif ou organisation.

Dans ce recueil spécifique, la plupart des textes sont produits par des anarchistes, d'un point de vue anarchiste-communiste, à l'exception de l'avant-dernier texte, écrit par un conseilliste. Cependant, nous souhaitons également mettre à disposition d'autres points de vue libertaires.

Nous avons l'intention de produire d'autres brochures, contenant d'autres traductions et même des textes obscurs provenant d'organisations libertaires romandes et françaises. Nous verrons si le projet se poursuivra, mais tant que tu as ce livre entre les mains, nous espérons que tu y trouveras quelque chose qui t’intéressera, ou qui t’ouvrira la voie vers quelque chose à quoi tu n’as pas encore pensé.

N’hésite pas à nous contacter si tu repères des erreurs, des phrases insensées, si tu veux aider le projet, si tu veux un autre exemplaire, ou simplement pour poser des questions.

Tous les textes seront disponibles dans la bibliothèque anarchiste de langue française, à l'adresse \url{https://fr.anarchistlibraries.net/}.

\bsc{La Corvina}


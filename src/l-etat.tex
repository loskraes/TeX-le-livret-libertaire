\chapter*{Avant-propos}

\section*{Crise de l'État Providence et la consociation}

N'est-il pas contradictoire que face à la crise de l'État Providence, nous, communistes anarchistes, nous retrouvions dans notre action syndicale et politique parmi les rares partisans de l'intervention de l'État ? N'est-il pas paradoxal que ce soit précisément nous qui finissions par défendre la nécessité d'étendre l'intervention de l'État, alors qu'une des caractéristiques fondamentales de notre idéologie politique est celle de l'extinction de l'État ?

\section*{Crise de l'État Providence et des économies planifiées}

Comme on le sait, l'État Providence est né du keynésianisme et, par les forces de sociales-démocrates, il est devenu l'une des pierres angulaires du développement des sociétés régies par des systèmes capitalistes avancés. Il permet d'absorber les conflits sociaux et de les orienter vers une croissance plus équilibrée de l'accumulation, de réinvestir les salaires de manière à assurer une croissance régulière de l'économie, tout en garantissant de meilleures conditions de vie aux citoyens. L'État Providence n'élimine pas la pauvreté et l'inégalité dans la répartition des ressources, mais il rend certainement moins dramatique le conflit entre la misère et la richesse, il aurait dû garantir que certains services dits essentiels tels que les soins de santé, l'éducation, le droit au logement, à un salaire minimum, donc que des conditions de vie globalement acceptables puissent être fournies à tous.

Cette conception du rôle de l'État a été opposée par les pays dits du « socialisme réel » avec  celle de l’État planificateur qui, par la planification des ressources et de la production, était censé mettre en œuvre une répartition équitable des biens. Ce schéma de fonctionnement de l'État n'est pas exempt, comme celui de l'État Providence, de défauts ; il souffre de bureaucratisme pour l'un et de malhonnêteté et d'affairisme pour l'autre, à tel point que les raisons de critiquer les deux systèmes s'entremêlent souvent.

La longue phase d'expansion qu'a connue l'économie mondiale a mis en crise les deux systèmes de gestion sociale. D'où la crise des systèmes de « démocratie populaire » et celle des systèmes d'État Providence sous le néolibéralisme. Dans la nouvelle situation, le rôle de l'État change à l'Est comme à l'Ouest, et les systèmes de gestion de l'accumulation caractérisés par la déréglementation et la maximisation du profit se poursuivent à travers l'abandon des États-nations aux multinationales et la concentration économique et financière progressive qui a désormais atteint des dimensions planétaires. Cette stratégie du capital a pour corollaire nécessaire l'appauvrissement progressif et inéluctable du quart-monde, la dégradation des conditions de vie et de travail des habitants des pays riches eux-mêmes, la disparition de la sécurité sociale, la barbarisation des relations intersubjectives avec la poussée toujours plus forte de l'individualisme et de la satisfaction compétitive de ses propres besoins par rapport à ceux des autres. Bref, c'est ce phénomène que l'on appelle communément la logique de la privatisation.

\section*{Choix consociatifs}

L'une des formes de défense partielle contre ce processus de transformation adoptée par les groupes les plus forts est la création d'agrégats sociaux par segments sociaux. Il s'agit de groupes d'individus unis par de forts motifs d'identification - position sociale, recensement, religion, race, etc. - qui se mobilisent pour défendre les intérêts collectifs communs à l'ensemble auquel ils appartiennent. Une telle société peut trouver des règles de coexistence et une répartition équilibrée des ressources, mais ce n'est certainement pas celle que nous voulons.

Cependant, la crise de la structure de l'État Providence incite certains d'entre nous à émettre l'hypothèse de la création de structures et de services autogérés qui reflètent notre orientation culturelle et répondent à nos besoins. Si dans le passé - rappelons l'expérience de la Colonie Cecilia ou de l'École moderne de Ferrer - de telles expériences étaient acceptables soit comme des expériences immatures (c'est le cas du premier exemple), soit comme des instruments de lutte, aujourd'hui une telle hypothèse est tout à fait réabsorbable, voire facilitée et nourrie parce qu'elle est parfaitement organique à la logique consociative.

\section*{Les luttes pour le renforcement des services publics}

En tant que communistes anarchistes, nous pouvons et devons lutter pour la liquidation de l'État, mais cela ne signifie pas qu'il faille lutter pour que le coût et la responsabilité de la fourniture de certains services n'incombent pas à la structure sociale. Bien sûr, pour nous, les organes de gestion politique et administrative changent – nous ne soutiendrons jamais le système de santé comme il existe – mais nous devrons certainement penser à un service de santé qui offre assistance et aide à tous les citoyens et dont les coûts incombent à la communauté. Il en va de même pour les écoles, le nettoyage des rues, la distribution de l'eau et les services de transport que l'on appelle communément les services publics. Le problème n'est donc pas celui de la nature publique ou privée de certains services, mais celui de l'organe ou des organes politiques qui gèrent et géreront l'entreprise, de la composition des organes administratifs, même s'ils sont techniquement nécessaires, et de la manière dont les contrôles politiques sont exercés sur eux par la collectivité.

\chapter*{Introduction}

Un des piliers de l'anarchisme historique est sans aucun doute l'anti-étatisme. Sans vouloir aller jusqu'aux excès de ceux qui nient même l'État Providence à cause de la présence de ce terrible petit mot et tombent dans les bras du libéralisme le plus féroce, même chez nous, trop souvent, la conception de la nécessité d'une société sans État conduit à des distorsions, qui proviennent, à notre avis, d'une prise en charge hâtive du bagage historique de l'anarchisme. Ce bagage doit en fait être contextualisé et analysé en profondeur, à une époque où le capitalisme rampant prône la dissolution de l'État en tant qu'appareil administratif bureaucratique qui perçoit des impôts et fournit des services.

\chapter{La naissance de l'État et ce qui l'a précédé}

Un peu d'histoire ne nuit à personne ! Le moloch d'État est né, dans sa configuration moderne, il y a plus de deux siècles. Cela coïncide étroitement avec l'émergence de la classe bourgeoise comme nouvelle classe dirigeante. Ce n'est pas un hasard si la plupart des fonctions typiques de l'État moderne prennent forme dans la France de la révolution de 1789. Il est logique de s'interroger sur les raisons profondes de cette transformation des structures de pouvoir dans la société, sur les relations sociales qui ont cessé d'exister pour faire place à d'autres, sur les changements que cela a entraînés dans les relations de classe et, surtout, sur la manière dont la domination de la classe bourgeoise émergente s'est articulée.

\section{Les relations sociales dans l'organisation féodale}

Lorsque certains anarchistes dénoncent à juste titre les échecs que l'État, en tant qu'organisation bourgeoise de la société, provoque dans les classes subalternes, ils s'abstraient trop superficiellement de la situation que ces classes ont connue avant la naissance de l'État libéral. L'absence totale de règles permettait aux détenteurs du pouvoir n'importe quel arbitraire au détriment des subalternes: la lecture des Fiancés est, nous le croyons, une expérience commune à tous. À la réflexion, il apparaît clairement qu'il s'agit là, après tout, de la véritable essence du pouvoir absolu.

Les pays pauvres étaient non seulement très pauvres (ils le sont toujours), mais ils fournissaient de la main-d'œuvre sous la forme extrême de l'esclavage.

Le concept des droits n'existait même pas. Dans l'Antiquité, celle-ci ne s'appliquait qu'aux citoyens libres de la cité-État, mais dans la dégénérescence féodale, elle était encore plus restreinte aux membres de l'aristocratie et du haut clergé. La grande majorité de la population vivait dans une situation de déni total de la dignité humaine.

\section{L'État libéral et le droit}

\emph{Liberté, fraternité, égalité,} telle est la devise fondatrice de l'État libéral moderne. Inutile de répéter, entre nous, l'hypocrisie qu'elle recèle. Ce qui est intéressant, c'est une autre considération. Le passage d'une organisation sociale dépourvue de règles (seule celle du plus fort) à une organisation qui se prétend fondée sur des règles fondamentales au-dessus de tout individu est loin d'être sans intérêt. Le principe, même s'il tend toujours à être ignoré, est là et il produit ses effets, malgré l'arrogance du pouvoir.

Pour donner un exemple, l'organisation des travailleurs serait inconcevable dans la société féodale ; attention, il s'agit d'une organisation des travailleurs et non d'une révolte. En effet, des révolutions même sanglantes (et même réussies) étaient possibles avant la révolution bourgeoise, mais la conquête progressive de portions croissantes de richesses ne l'était pas. Il est évident que ces conquêtes sont partielles, souvent temporaires car (comme nous le constatons aujourd'hui) elles sont résorbables, et que la seule étape qui compte est l'étape révolutionnaire. Cela n'enlève rien à deux choses: d'une part, comme le disait Malatesta, la gymnastique de la lutte est une gymnastique pour la révolution, d'autant plus nécessaire pour ceux qui, comme nous, croient en une révolution consciente et conscientisée, non réabsorbable par les prétentions d'une nouvelle classe dominante en vertu de son savoir. Et d'autre part, le fait que tout ce qui améliore la vie des gens aujourd'hui n'est pas méprisable simplement parce que ce n'est pas le communisme libertaire en soi.

La société libérale, en se couvrant du voile des droits humains, voile nécessaire à sa lutte contre les anciennes classes dominantes, sanctionne un principe qui est progressif, en fait et en résultat, même pour les classes qui restent subordonnées.

\section{Participation progressive}

« \emph{Le solidarisme kropotkinien, développé sur le terrain naturaliste et ethnographique, confondait l'harmonie de la nécessité biologique des abeilles avec la discordia concors et la concordia discors propres à l'agrégat social, et avait trop (sic !) de formes primitives de sociétés-associations présentes pour comprendre l'ubi societas ibi jus, “ là où il y a une société, il y a de la loi ” , inhérent à des formes politiques qui ne sont pas préhistoriques ».}

Cette citation nous fournit deux bases de réflexion utiles.

La première est qu'il n'y a pas de société possible sans règles: on peut et on doit discuter, et les anarchistes le font, de la manière dont elles devraient être formulées, de qui a le pouvoir de les établir, de la manière dont elles devraient être universellement partagées, etc. En l'absence de règles, il n'y a pas d'anarchie, mais une jungle qui pénalise toujours les plus faibles et profite aux plus forts.

La seconde est que les règles « partagées » auront une double valeur: lier la liberté individuelle d'une part, et garantir la justice sociale et la protection pour tous d'autre part.

\chapter{L'État du 19\ieme{} siècle et la naissance de la théorie anarchiste}

Le point de départ de la réflexion anarchiste sur le rôle de l'État avant, pendant et après la révolution sociale est sans aucun doute Bakounine. Il faut dire tout de suite que pour comprendre le rôle de l'État moderne et les moyens de le dépasser, l'approche bakouniniste n'est pas d'un grand secours, parce qu'elle est trop liée aux nécessités de sa lutte et son contexte. Malheureusement, certaines de ses affirmations, prises hors contexte et sans aucun effort d'interprétation, ont été rendues incontestables et pour des principes inflexibles et immuables de l'anarchisme. Pour sortir d'une supposition superficielle de mots d'ordre qui finissent par fausser toute entreprise politique, il est nécessaire de faire quelques rappels.

L'élaboration de Bakounine s'est faite dans la dernière décennie de sa vie, au milieu de ses actions au sein de l'Association internationale des travailleurs et de sa polémique avec la composante marxiste ; en outre, les principales références, étroitement liées au développement de l'action révolutionnaire du groupe anti-autoritaire, étaient l'Italie, l'Espagne, la Russie et l'Autriche, auxquelles il faut ajouter l'empire allemand, à la fois en raison de son rôle émergent en tant que première puissance continentale européenne et parce que le noyau fort des antagonistes sociaux-démocrates s'y trouvait.

Dans ce cadre, les préoccupations immédiates de Bakounine sont au nombre de trois:

\begin{itemize}
\item établir définitivement que la conquête de l'État (par les élections) ou sa transformation par les réformes ne sont pas des voies viables vers la société égalitaire et solidaire ;
\item démontrer que là où il y a une forme de pouvoir, il y a toujours une forme d'exploitation et qu'il n'y a donc pas d'organisation sociale meilleure qu'une autre, si ce n'est la société sans propriété, sans classes et sans hiérarchie ;
\item enfin, conséquence logique, cette organisation étatique ne peut et ne doit pas survivre à la révolution sociale.
\end{itemize}

Ces points restent incontestablement les traits distinctifs et fondateurs de toute conception anarchiste.

Dans l'urgence de fixer les coordonnées ci-dessus, Bakounine, convaincu de l'imminence du soulèvement révolutionnaire des masses grâce au développement irrésistible de l'Internationale, n'a pas trouvé le temps ni l'espace de réflexion nécessaires à une analyse approfondie du rôle que l'État assumait déjà depuis trois quarts de siècle, dans un processus lent, contradictoire, souvent difficile à cerner, mais certain et en quelque sorte irréversible. Pour lui, l'État est essentiellement l'État allemand ou le tsarisme autocratique russe. À tel point qu'il ne considère même pas l'État anglais comme un véritable État, puisqu'il ne correspond pas aux critères qu'il estime distinctifs de l'État moderne, à savoir: la centralisation militaire, policière et bureaucratique. On voit bien la distorsion qu'implique, d'un point de vue théorique, l'échange des organisations étatiques, ou mieux centralisées, résiduelles du passé, avec l'État moderne que l'on identifie précisément en Grande-Bretagne et dans l'État français en pleine transformation, même si c'est avec l'héritage historique d'une centralisation pluriséculaire.

En effet, le moloch étatique est entré dans la théorie anarchiste précisément à partir de cette conception de la centralisation militaire, policière et bureaucratique, terreau de toutes les déformations futures et de l'incapacité d'adapter l'analyse. Chaque évolution de l'État a reçu l'interprétation d'un approfondissement de ces centralisations, ce qui a empêché le discernement de nouvelles fonctions, pas toujours négatives, et conduit aujourd'hui beaucoup d'anarchistes à un démantèlement théorique face à des formes de décentralisation et de dissolution apparente même de l'appareil oppressif.

Bakounine avait également averti que le non-État anglais (décentralisé) n'était pas moins dangereux pour cela, bien que sa polémique, nécessaire pour que l'urgence de la révolution soit correctement finalisée et pour balayer les illusions pernicieuses, ait eu tendance à assimiler différentes formes de régime bourgeois, sans savourer les différences même pour les besoins des conditions de vie matérielles des masses ; en effet, à certains moments, l'illusion démocratique a été considérée comme encore plus négative pour le développement de la conscience révolutionnaire du peuple.

Cependant, Bakounine ne semble pas toujours indifférent aux règles de la société dans laquelle se déroule la lutte révolutionnaire, ce qui prouve que cet aspect n'est resté que peu développé dans sa pensée.

\chapter{L'évolution de l'État}

Bien qu'au milieu du siècle dernier l'évolution de l'organisme étatique ait déjà pris des proportions repérables (mais qui ont échappé non seulement à Bakounine pour les raisons mentionnées plus haut, mais aussi à Marx), les connotations qu'il allait prendre étaient en effet difficiles à prévoir. Il y a deux considérations qu'il est intéressant de développer: d'une part, l’amalgame des compétences qu'il en est venu à assumer et l'évaluation de leurs retombées dans l'organisation sociale dans son ensemble ; d'autre part, si l’existence de l'État n’a que eu des impacts négatifs sur le « progrès » du genre humain, il doit être considéré comme une parenthèse de la tendance humaine originelle à la solidarité mutuelle. Il est évident que la réponse à ces deux questions est loin d'être négligeable pour l'évaluation des luttes d'aujourd'hui, bien qu'elle puisse difficilement constituer, comme nous le verrons, un changement de perspective pour la réalisation d'une société sans classes et, d'ailleurs, sans État.

\section{L'État entrepreneur}

Lorsque nous parlons de l'État moderne, nous avons tendance à confondre trois fonctions que l'appareil d'État lui-même exerce, mais qui sont profondément différentes les unes des autres et ne sont pas du tout nécessaires l'une à l'autre: la régulation du cours du cycle économique, l'intervention directe dans l'économie des entreprises et l'aide sociale. Ces trois fonctions se sont ajoutées au cours de ce siècle, se superposant au rôle traditionnel de gendarme des intérêts bourgeois, bien connu des révolutionnaires du 19\ieme{} siècle.

Les théoriciens de l'avènement de la technobureaucratie ont vu dans cette multiplication des pouvoir privilégiées la confirmation de leurs prédictions d'une incorporation totale de la société dans ce monstre omnivore que serait l'État.

En parfaite continuité avec le déterminisme de Kropotkine, pour eux l'histoire est à sens unique et les voies de l'évolution sociale sont déjà tracées, de sorte que les tendances qui se manifestent entre les années 1930 et 1970 indiqueraient sans équivoque les débouchés futurs: leur vision téléologique n'est que l'envers de la vision marxiste, puisque toutes deux manquent de la connaissance de la fonctionnalité de l'organisation sociale aux intérêts contingents du capital et par conséquent de la réversibilité de choix qui leur paraissent au contraire définitifs. Ce n'est donc pas un hasard si la désintégration de l'appareil d'État, qui a commencé à se manifester au cours des deux dernières décennies, les trouve théoriquement désarticulés et bredouillants dans leurs propositions, sinon résolument et irrémédiablement cohérents avec les mouvements en cours dans les hautes sphères de l'économie mondiale.

\subsection{Contrôler le cycle}

L'impossibilité de prévenir des crises cycliques de plus en plus dévastatrices, après l'échec des théories marginalistes visant à interpréter scientifiquement les tendances du marché, a conduit le capital à une mutation drastique de ses caractéristiques. Au cours des années allant du début de la quatrième décennie à la fin de la septième, l'État, de simple gendarme des intérêts capitalistes (drainage fiscal, contrôle policier, politique douanière, etc.), est devenu le moteur de l'économie, se chargeant, par une augmentation substantielle de la pression fiscale et l’initiative de travaux publics grandioses, de relancer le cycle économique précipité vers l'abîme de la crise.

Cette nouvelle approche économique (keynésianisme) a eu pour conséquence nécessaire l'expansion du marché, condition indispensable à l'absorption d'une quantité toujours croissante de biens, selon un cycle perpétuellement progressif. Les salaires deviennent le volant de l'économie (fordisme) et augmentent mais en dessous de la productivité, sous l'effet de l'innovation technologique dans l'organisation du travail (taylorisme). L'objectif est de réduire la lutte des classes à un outil permanent de rationalisation du système.

Il est clair que le capitalisme invente une nouvelle ère pour sa propre prospérité, mais en même temps des masses croissantes du prolétariat métropolitain dans les pays industrialisés accèdent à la consommation de biens qui leur étaient auparavant inaccessibles. La saison des luttes de la fin des années 60 a clairement montré que cette circonstance ne se traduisait pas par une intégration définitive des classes subalternes dans la logique de l'entreprise ; au contraire, c'est précisément à partir des secteurs les plus identifiables comme représentants des soi-disant masses ouvrières que la contestation systémique a commencé et a continué à s'articuler autour d'eux.

\subsection{Gestion directe des capitaux}

Une nouvelle étape a été franchie dans les années 1930. L'évolution se fait presque naturellement, mais elle n'est pas nécessaire ; à tel point qu'elle ne se produit pas dans le système capitaliste central: les États-Unis. Une lecture superficielle pourrait assimiler ce qui se passe dans les deux mondes antagonistes de l'économie planifiée globale (zone soviétique) et de l'économie planifiée directionnelle (Europe capitaliste). Mais, comme nous le verrons, les deux cas d'école présentent des caractéristiques qui ne les rendent pas assimilables.

Le premier stimulus est apparu presque par hasard dans l'Italie fasciste: face à la crise de nombreux complexes industriels, le régime a créé (1933) l'Institut pour la reconstruction industrielle (IRI), qui a repris les entreprises dites en déclin et devait les remettre sur le marché une fois qu'elles se seraient rétablies. Au contraire, au bout d'un certain temps, l'Institut s'est retrouvé en possession d'une partie considérable des forces de production industrielle et a fini par les gérer lui-même, donnant naissance au secteur des holdings d'État. L'IRI a survécu au fascisme et est devenu, après la Seconde Guerre mondiale, un acteur majeur de la vie économique nationale. Son succès à aplanir les aspérités du cycle économique, grâce à l'énorme disponibilité de capitaux, y compris, mais pas seulement, de capitaux d'État, est tel que, dans les années 1950, les travaillistes britanniques en viennent à étudier son fonctionnement pour le reproduire en Grande-Bretagne, imités par les Français et les Allemands. C'est ainsi qu'est né l'État qui participe directement à la vie économique avec son propre capital, l'État entrepreneur.

Il en va tout autrement de l'économie soviétique, où la gestion étatique de l'économie est globale et ne s'inscrit pas dans un régime concurrentiel, en réponse à l'arrivée au pouvoir d'une classe autre que la bourgeoisie entrepreneuriale: la petite bourgeoisie éduquée, avec ses propres mécanismes d'extraction du produit excédentaire. Il en résulte deux types différents de planification économique, qui ne se ressemblent que par leur nom.

Il n'est pas possible d'éviter, à ce stade, un jugement rapide sur le nouveau rôle que l'État a assumé, en continuité mais non en conséquence avec le rôle déjà examiné de régulateur et de stimulateur du cycle économique. Ceux qui ont connu les luttes syndicales des années 1960 et 1970 se souviennent certainement que deux contrats distincts étaient alors signés pour les travailleurs employés par les entreprises privées et pour ceux employés par les entreprises publiques: les seconds anticipaient souvent sur les premiers, jouant le rôle de précurseurs et les obligeant, par analogie, à faire des concessions que les patrons privés n'acceptaient pas volontiers de faire. A l'ère du libéralisme galopant, les participations de l'État sont devenues synonymes de gaspillage clientéliste, et sur cette vague émotionnelle se sont démantelées, vendant leur patrimoine instrumental à des particuliers. C'est ainsi qu'une entreprise modèle comme le Nuovo Pignone de Florence, après avoir été rachetée par AGIP (de l'IRI), après avoir été reconvertie à de nouveaux types de production, après avoir développé une technologie de pointe, après avoir conquis des parts très importantes du marché mondial du secteur et être devenue une source de profits substantiels pour l'État, a été vendue au concurrent américain General Electric.

Il ne fait aucun doute qu'une classe de gestionnaires publics s'est enrichie grâce à la gestion des entreprises publiques, mais il ne fait aucun doute non plus que les salaires privilégiés et le cadre réglementaire des travailleurs de ces entreprises ont servi de point de référence pour les autres travailleurs en poussant à la hausse les exigences de tous. Il est donc légitime de douter que la ferveur pour la destruction du secteur de la participation publique découle davantage de la nécessité pour l'entreprise privée d'éliminer un concurrent gênant que d'un vague besoin de moralisation peu crédible.

En revanche, l'élimination physique d'Enrico Mattei, président de l'AGIP et partisan d'une politique autonome d'approvisionnement en pétrole brut qui couperait les ponts avec le cartel pétrolier international (les sept sœurs), par les compagnies pétrolières elles-mêmes, est plus qu'une piste de réflexion.

\subsection{Bien-être}

L'État, au cours du siècle dernier, a progressivement assumé le rôle de fournisseur de services sociaux (éducation, santé, bien-être, transports, etc.). L'avantage pour les patrons est évident: ils se déchargent sur la fiscalité générale (à laquelle ils contribuent relativement moins que les salariés) de la préparation, de la récupération, d'une timide forme de sécurité et de la mobilité de la main-d'œuvre, ce qui se traduit par une meilleure qualité des performances professionnelles et, on l'espère, par une diminution des conflits sociaux. Cela n'empêche pas que, même pour les travailleurs, tout cela ne se traduit pas par un avantage indéniable, notamment parce que l'alternative n'est pas une baisse de la charge fiscale, sur laquelle il conviendra de revenir, mais l'abandon des formes de protection de la vie associées à la jungle du profit, comme nous le constatons avec une clarté absolue.

A tel point qu'à une certaine époque, l'aide sociale portait le nom de salaire social et était considérée par les associations de travailleurs comme une forme de rémunération de leur travail. Il faut donc considérer que si l'enseignement public était contraint à l'acquisition d'un métier, d'un autre point de vue, il constituait un contact avec l'acquisition d'instruments culturels et critiques, auparavant totalement interdits aux classes subalternes ; si les soins de santé ne tendaient qu'à restaurer la force de travail endommagée, d'un autre point de vue, ils garantissaient la guérison des maladies qui fauchaient auparavant le prolétariat ; Si les régimes de retraite tendent souvent à répercuter sur la société les coûts d'une main-d'œuvre licenciée ou obsolète, sous un autre angle, ils offrent une alternative à l'enfermement dans des maisons de repos et à la dégradation totale de la vieillesse à laquelle étaient soumis les travailleurs subalternes ; si le système de transport public permet la marginalisation de la main-d'œuvre massivement urbanisée dans des banlieues aliénantes, sous un autre angle, il garantit également une meilleure jouissance du temps libre à des couches de la population qui en étaient autrefois exclues.

Refuser d'examiner la réalité complexe de l'État et de toutes ses formes, c'est tout simplement faire preuve de manque de lucidité.

Ainsi, si l'État est l'ennemi, tout ce qui vient de lui doit être rejeté, quel que soit l'autre ennemi, le capitalisme, qui vise aujourd'hui précisément à la destruction de l'État. Mais il en est une autre, plus insidieuse mais non moins erronée. Le prolétariat et le capital étant antagonistes dans leurs intérêts, tout ce qui profite au second ne peut être qu'un désavantage pour le premier. S'il en était ainsi, puisqu'il est indéniable que les salaires sont le moins que les patrons aient à donner pour obtenir la pleine exploitation de la force de travail et qu'ils sont en eux-mêmes un avantage pour les employeurs, ils devraient être rejetés par les salariés. En effet, de même que l'on lutte (ou plutôt il serait souhaitable que l'on lutte) pour améliorer la part des marchandises en faveur des salaires et contre celle du profit, de même on devrait s'efforcer de tourner les services de plus en plus dans le sens utile aux classes exploitées et de moins en moins en faveur des classes aisées. Sans que cela signifie, bien entendu, que l'on puisse renoncer au bouleversement révolutionnaire pour parvenir à une société juste, libre et égalitaire.

\section{De l'État primitif à l'État moderne}

Il résulte des remarques sommaires qui précèdent qu'en un siècle et demi (et comment pourrait-il en être autrement ?) l'État a substantiellement modifié son rôle, son fonctionnement, sa structure. Si, d'une part, le marxisme, en séparant le rôle du gouvernement (le comité d'entreprise de la bourgeoisie, selon l'aphorisme bien connu de Marx) de celui de l'État en tant qu'appareil, a fini par émettre l'hypothèse de l'utilisation à des fins révolutionnaires de la machine étatique, soumise à une nouvelle gestion, une partie de l'anarchisme, en identifiant les deux fonctions, a prétendu perdre, au fil du temps, la capacité de distinction et, par conséquent, celle de l'orientation politique.

Il est donc nécessaire de reconsidérer l'ensemble de la question si l'on veut échapper à l'emprise de l'acceptation de l'appareil d'État tel qu'il est ou à la négation a priori de tout ce qui en découle, ce qui nous conduirait tout autant dans les bras du néolibéralisme le plus agressif.

\chapter{L'ambiguïté du rôle de l'État}

Si l'on fait abstraction de l'État absolutiste ou théocratique, pure expression du pouvoir d'une caste privilégiée (contre laquelle s'exerçait la critique de Bakounine, comme nous l'avons vu), encore en vigueur dans de nombreux pays au milieu du XIXe siècle, mais en tant que phénomène résiduel, notre attention doit se porter sur l'État libéral, désormais solidement implanté dans tout le monde du haut développement capitaliste (et dont on sait qu'il représente un moindre mal dans les pays tiers encore opprimés par des dictatures féroces).

Les droits bourgeois sont, il est vrai, des fictions ; l'État n'est jamais impartial ; dans une société divisée en classes, différentes classes vivent et pratiquent même l'anarchie avec des conséquences tout à fait différentes en termes de vie et de punition. Pourtant, l'aphorisme bien connu de l'eau sale et du bébé doit être pris en compte, même si l'eau est grande et le bébé vraiment petit, et ce pour deux bonnes raisons. La première est qu'il serait de toute façon stupide de sacrifier l'enfant ; la seconde est que nous aiderions l'ennemi de classe qui vise précisément à conserver l'eau sale en éliminant l'enfant, qui serait le premier à disparaître.

\section{L'État dans la révolution}

Le point sur lequel les anarchistes se sont toujours opposés aux marxistes a été celui de la nécessité ou non de la survie de l'État dans la période de transition: centralisation des fonctions pour propager et défendre les résultats révolutionnaires pour les adeptes du socialisme dit scientifique ; décentralisation et prise en charge par le prolétariat de la gestion sociale pour que le prolétariat s'approprie immédiatement l'événement révolutionnaire comme solution aux problèmes générés par la société divisée en classes, pour les communistes anarchistes.

Les marxistes ont qualifié la position des anarchistes de corporatisme, arguant que suivre leur méthode créerait des conflits et des inégalités et que personne ne serait en mesure de contrer efficacement l'inévitable réaction de la bourgeoisie. Les anarchistes, quant à eux, soutenaient que la survie d'un pouvoir centralisé (l'État) régénérerait une classe expropriatrice et détournerait les masses de la révolution. L'expérience a donné raison à ces derniers sans équivoque, notamment parce que des exemples admirables de solidarité entre les dépossédés se sont toujours produits là où l'autogestion révolutionnaire du prolétariat disposait de quelques timides espaces de libre expression.

Cela dit, venons-en au fond. Tout d'abord, dans leur critique vertueuse, certains anarchistes se sont engagés sur une pente glissant qui pourrait s'avérer dangereux s'il n'était pas suffisamment étudié: la solidarité est un projet de civilisation auquel l'homme doit être éduqué, et ce n'est pas un hasard si les exemples cités plus haut se sont tous produits là où les militants révolutionnaires avaient exercé le plus longtemps et le plus efficacement leur influence, et donc là où les masses étaient les plus préparées à la révolution. En d'autres termes, il serait pernicieux de confondre l'anarchie, qui est la condition finale de l'évolution de l'homme (résultat d'une croissance de la civilisation, de la prise de conscience de son rôle social et de sa sensibilité), avec le comportement primordial de l'homme animal, qui est violent, grossier et agressif (féral).

Deuxièmement, il faut éviter le glissement de contenu: l'administration des affaires publiques ne doit pas être centralisée. Au contraire, c’est les services sociaux qui doivent conserver un rôle centralisé (sur la base d'un libre accord ascendant, bien entendu), afin de garantir les mêmes droits à tous, quelle que soit leur situation géographique.

Les anarchistes espagnols de 1936 n'en doutaient pas, et sachant que la révolution ne marche que si dès le premier jour (dans la mesure du possible) tout fonctionne, de l'approvisionnement aux services, ils ont organisé les travailleurs des services publics (par exemple les transports à Barcelone) pour les rendre utilisables. Il s'ensuit que s'il est juste de démolir et de ne pas changer l'appareil d'État bourgeois (comme on disait autrefois), cela ne doit pas concerner la fourniture de services sociaux: apprentissage des enfants, protection des personnes âgées, soins aux malades, transport des citoyens, etc. Il semble également évident de déduire que là où ces services fonctionnent déjà sur la base de normes valables pour tous et sont fournis au citoyen en tant que tel, la transition des travailleurs du secteur vers une gestion collectivisée et uniforme est plus facile et plus efficace que là où les mêmes services sont éclatés dans des mains privées et soumis à la logique du profit.

\section{Le premier ennemi}

Les marxistes ont toujours soutenu que toute l'évolution historique est déterminée par la structure (la structure de production, avec les relations sociales qui y sont associées), tandis que les autres aspects (politique, culture, guerre, etc.) n'en sont que des conséquences plus ou moins directes, mais néanmoins nécessairement déterminées (superstructure).

Les anarchistes, au contraire, pensaient que oui, la structure était la source première de l'ordre social (l'histoire est l'histoire de la lutte des classes), mais que la superstructure n'en était pas si strictement dépendante, c'est-à-dire qu'elle possédait ses propres marges de vitalité et pouvait même à son tour interagir, en contribuant à la déterminer, avec la structure elle-même. {[}Curieusement, notons-le au passage, les marxistes ont développé un très grand intérêt pour les médiations politiques et électorales (les formes naissantes de l'économie, comme les appelait Marx), tandis que les anarchistes ont cultivé un désintérêt fanatique à leur égard{]}.

S'agissant de l'État, les marxistes en ont tiré la conséquence qu'une fois les rapports de production (structures de propriété) modifiés par la révolution, la superstructure étatique devait en suivre les impératifs jusqu'à disparaître par consommation de sa fonction (les trotskistes, partant de cet axiome même, parlaient d'un État prolétarien dégénéré pour l'URSS, n'admettant pas le renversement complet des objectifs révolutionnaires par le nouvel appareil bureaucratique soviétique). Les anarchistes, convaincus que le pouvoir pouvait à son tour régénérer l'exploitation, initialement abolie (ce qui s'est évidemment produit), préconisaient l'abolition immédiate de l'appareil d'État, remplacé par des formes alternatives d'associationnisme coopératif.

Là encore, le principe était bon, mais au fil du temps et de la mauvaise propagande, il a été corrompu au point de devenir dangereux, voire très dangereux. Oubliant que l'ennemi principal est l'exploitation de l'homme par l'homme (comme le savait bien Bakounine) et que l'État est une des formes historiques de sa manifestation, ni unique ni nécessaire, ils ont confondu la théorie de la phase transitoire avec la théorie de l'histoire et ont proclamé l'État comme premier ennemi du prolétariat (quand ce n'est pas le seul). Ils ont opposé à la « statolâtrie » marxiste une « statophobie » non moins obtuse. En d'autres termes, ils ont centralisé leur critique sur l'instrument de domination du capital dans une phase spécifique, négligeant la domination elle-même et ses autres formes d'existence possibles, uniquement par crainte que dans la phase révolutionnaire, l'État survive et reproduise l'exploitation.

C'est pour cette raison que, dans de nombreux écrits anarchistes, on affirme que l'État est le premier ennemi et que ceux qui affirment que le premier ennemi est la classe bourgeoise sont accusés de cryptomarxisme ; il est dommage qu'à présent les patrons eux-mêmes visent à la dissolution de l'État, tel qu'il était connu au XXe siècle, et que dans certaines franges extrêmes du néolibéralisme états-unien (Friedman Jr.), ils envisagent de privatiser même les forces de police, revenant ainsi aux Bravi de la mémoire de Manzoni, ou à toutes ces formes de police privée (et/ou de travail criminel) utilisées pour la répression sous différentes formes et à différents stades par presque tous les États du monde.

Rappelons au passage que les mafias du monde entier naissent ou résistent précisément comme une forme de contrôle social et policier, là où - les rapports d'exploitation n'ayant pas été abolis - les formes étatiques, incapables de garantir même par la force à la bourgeoisie le plein contrôle du territoire, se voient contraintes de le partager avec les pouvoirs forts des mafias, en les absorbant ou en se laissant imprégner par eux à tous les niveaux institutionnels.

\section{Fonctions collectives et coercitives}

En conclusion, une approche généralisante ne nous fait pas avancer d'un pas (mais reculer de beaucoup). Il est donc nécessaire de distinguer les différentes fonctions que l'État moderne remplit (ou plutôt qu'il remplissait avant le récent assaut néolibéral): les fonctions de maintien de l'ordre social existant, tant à l'intérieur d'une région qu'au niveau international (la \emph{guerre}, comme on l'a appelée), des fonctions de fourniture d'un niveau minimum de sécurité aux citoyens (l'\emph{aide sociale}, précisément). Les premières sont purement coercitives et n'ont pas de raison d'être dans une société égalitaire, les secondes visent à une intégration sociale douce et jouent un rôle que toute société qui veut s'appeler telle doit assumer, même si c'est sous une forme variée.

Les tendances actuelles indiquent une voie très différente de la voie souhaitable, une voie que le capitalisme a empruntée avec beaucoup de zèle. L'élimination du bien-être et le maintien ou plutôt le renforcement de la guerre. Les traités de l'Union européenne, le renforcement de l'OTAN, l'élargissement de l'armée professionnelle en Italie et dans d'autres pays vont dans ce sens, ce qui exclut, entre autres, une réduction conséquente de la charge fiscale, au moins pour les salariés.

On peut d'ailleurs ajouter que le développement de la protection sociale marque une voie dont le renversement ne fait que le jeu de l'adversaire de classe, et qui prépare, plus qu'elle n'éloigne l'homme, en tant qu'animal, social, à une gestion collective et solidaire des relations. Il semble cependant que pour certains anarchistes autoproclamés, le mal soit la santé publique, l'éducation publique, l'assistance publique, dans la mesure où elles sont assurées par des organismes d'État, et non l'exploitation de la maladie, du savoir et de la vieillesse à des fins de profit.

Et n'oublions pas que si l'État est un obstacle à toute réalisation révolutionnaire, et qu'il doit disparaître dès le premier instant de tout renversement des rapports de force entre la bourgeoisie et le prolétariat, même son apparition historique représente un progrès par rapport à l'arbitraire barbare qui l'a précédé, et que sa disparition, sans renversement des rapports de propriété actuels, éloigne plus qu'elle ne rapproche du but.

\chapter{À propos des règles}

L'anti-étatisme anarchiste a sans doute le mérite d'avoir historiquement porté l'attention sur des aspects que le marxisme a résolument négligés: le rôle du pouvoir politique, le rôle des institutions pendant et après l'événement révolutionnaire, le rôle des classes intellectuelles, la logique interne de l'administration et sa capacité d'auto-reproduction, l'autonomie évolutive de la superstructure dans certaines conditions et son influence sur l'évolution générale. Dans tous ces domaines, les acquisitions sont théoriquement irréversibles et prouvées par l'expérience des diverses tentatives de construction du socialisme autour des paramètres des formes les plus variées du marxisme.

Il est cependant nécessaire de nettoyer l'antiétatisme des débris qu'il traîne derrière lui en raison de l'accumulation d'interprétations trop souvent superficielles et fondées sur de simples assonances nominales. En particulier, la confusion pernicieuse entre État et public, entre bureaucratie et services, entre top-down et collectif. Il est bien vrai que les services publics souffrent de la bureaucratisation et d'une faible perméabilité aux besoins des individus qui sont censés les utiliser. Mais il est tout aussi vrai que les polémiques que les médias de pouvoir déposent quotidiennement sur ces inefficacités sur les tables des télé-utilisateurs au cerveau lavé ne servent qu'à ouvrir la voie au profit privé. Le chemin qui mène des services publics très critiqués d'aujourd'hui à la société égalitaire et sans classe ne traverse pas le territoire infranchissable du capitalisme débridé et de l'intérêt supposé du citoyen individuel ; le chemin est autre et va dans la direction opposée:

\begin{itemize}
\item leur reconnaissance en tant que salaire d'égalisation indirect ;
\item la demande de services plus étendus, plus efficaces et gratuits pour tous ;
\item un contrôle de plus en plus efficace par la communauté, non compris sous la forme de ses représentants politiques, sur la qualité de leur prestation.
\end{itemize}

C'est ainsi que l'on peut préparer la voie à une autogestion efficace de la société et à des services qui comblent les inégalités que la nature crée entre les êtres humains, ce qui est le sens véritable et le plus profond du service public.

\chapter{Pour la liquidation de l'État}

Avant d'aborder le problème de la période de transition, il est nécessaire que l'organisation politique des communistes anarchistes clarifie brièvement, non seulement sur le plan terminologique, mais fondamentalement sur le plan stratégique, les différentes conceptions qui préfigurent la fin de l'État bourgeois, suite à la rupture politico-institutionnelle provoquée par une révolution victorieuse du prolétariat.

Nous dépassons les concepts « d'abolition de l'État » ou de « destruction de l'État » car ils préfigurent deux aspects liés à la fin de l'État, centrés sur l'action violente d'un groupe de professionnels de la politique et sur l'instantanéité ou la rapidité de cette action.

A l'opposé de ces deux conceptions, nous en trouvons deux autres que nous rejetons également. Il s'agit de la conception de la « décadence de l'État » ou de « l'extinction de l'État ». Nous les dépassons toutes deux dans la mesure où elles préfigurent deux aspects relatifs à la fin de l'État, le premier se référant à un processus entièrement objectif et mécanique qui conduirait à la disparition de l'État, et le second à une sorte de gradualité de ce processus.

Si, dans les deux premiers cas, nous ne voyons pas l'intérêt d'une action minoritaire violente d'un groupe politique contre l'État s'il n'y a pas d'auto-organisation réelle du prolétariat, en même temps, dans les deux autres cas, nous considérons qu'un processus spontané et automatique d'extinction de l'État est impossible sans l'action révolutionnaire de la classe subordonnée qui travaille dans ce sens.

Le choix stratégique fondamental des communistes anarchistes s'oriente vers la conception de la « liquidation de l'État », comme action politique et économique d'organisation de l'autonomie prolétarienne visant à rendre impossible toute reconstruction de l'État et à lui ôter ainsi toute base sur le plan social.

La liquidation de l'État est donc l'acte final d'un processus qui est déjà né et s'est développé au sein de la société divisée en classes et en opposition absolue à celle-ci, et qui marque la rupture définitive et totale entre le système classiste et autoritaire et la nouvelle société communiste anarchiste.

La liquidation de l'État s'identifie donc à la destruction des structures d'exploitation et des appareils de domination, à la transition de la société divisée en classes à la société communiste anarchiste, réalisant l'objectif révolutionnaire de la destruction des institutions légales, militaires et administratives des rapports de classes sociales pour permettre la mise en oeuvre de méthodes communistes de production, de distribution et de régulation sociale sous le contrôle et l'autogestion des structures prolétariennes fédérées et autogérées de façon libertaire.


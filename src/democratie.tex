\chapter*{\textbf{Première Partie}}
\markboth{Première Partie}{Première Partie}

On nous dit que nous, États-Uniens, vivons dans le pays le plus riche et le plus démocratique du monde. Nos droits incluent la liberté d'expression et de religion, ainsi que la liberté de voter pour nos dirigeants. Notre pays possède plus de richesses que n'importe quel autre - plus de richesses, en fait, que la majeure partie du reste du monde réuni. À la télévision et dans la vie réelle, nous voyons des États-Uniens avec de grandes maisons, des voitures de luxe, de nombreux gadgets dernier cri et des abonnements à des terrains de golf ou à des stations de ski.

Mais nous savons très bien qu'il ne s'agit pas d'un tableau complet. Il s'agit plutôt d'une publicité. Bien que nos quartiers soient séparés, riches et pauvres, blancs et noirs, latinos et amérindiens, peu de gens ignorent que la plupart des États-Uniens ne vivent pas comme les personnages des sitcoms télévisés. Les habitants des banlieues riches sont souvent confrontés à la pauvreté dans les villes où ils travaillent pour diverses entreprises et administrations. Les habitants des zones défavorisées sont souvent contraints de se rendre dans les banlieues pour travailler et servir du café aux personnes riches et blanches.

Que l'économie aille « bien » ou « mal », des millions de personnes sont au chômage et incapables de se vendre aux employeurs pour acheter les choses dont elles ont besoin. Beaucoup de ceux qui ont un emploi travaillent 40, 60 ou 80 heures par semaine, dans des emplois éreintants, dangereux, malsains et dégradants, juste pour payer un logement, des vêtements, de la nourriture et des médicaments pour eux-mêmes et leur famille. Pendant ce temps, leurs patrons, dont le travail est plus facile et plus sûr, gagnent deux fois plus d'argent, et les personnes qui siègent au conseil d'administration des entreprises ne travaillent pas et gagnent des millions. Les gens sont refoulés des hôpitaux, même en cas d'urgence, et se voient refuser des soins médicaux parce qu'ils n'ont pas les moyens de s'assurer, alors même que les compagnies d'assurance gagnent des centaines de millions de dollars en surfacturant les gens et en essayant de se soustraire au paiement des procédures médicales qu'elles considèrent comme « non essentielles ».

Dans ce pays d'abondance, des personnes dorment dans la rue, mourant dans le froid de l'hiver ou la chaleur de l'été, tandis que des propriétaires conservent des logements vacants en attendant que les prix augmentent. Et la police n'a manifestement aucun problème à battre ou à emprisonner les personnes sans domicile fixe qui squattent des appartements vacants. Pourquoi une grande partie des États-Unis vit-elle dans la pauvreté, alors que d'autres ont plus d'argent qu'ils ne peuvent en utiliser ?

La pauvreté n'est pas notre seul problème. Chaque jour, des policiers racistes battent ou abattent des personnes de couleur, et des millions de personnes, en particulier des personnes noires et des latinos, croupissent en prison, soumises à des peines extrêmement longues et à des conditions horribles pour des délits mineurs souvent anodins. Les femmes font l'objet de discriminations et sont souvent victimes de violences et de viols. Les lesbiennes, les gays, les queers et les transsexuels sont également victimes d'exclusion, de harcèlement et de violence. Les enfants sont traités comme des sous-humains, sans aucun droit et forcés d'aller dans des usines éducatives (« écoles ») où ils sont endoctrinés avec de nombreux mythes néfastes de notre société et où on leur apprend à accepter les problèmes de notre monde comme étant « naturels ». Les entreprises abattent nos forêts, conduisent les plantes et les animaux à l'extinction, empoisonnent le sol, les rivières, l'air et les personnes, tout cela dans l'intérêt du profit. Notre gouvernement déclenche des guerres auxquelles de nombreuses personnes s'opposent et obtient l'obéissance de tous les autres en utilisant les médias pour raconter des mensonges qui entraînent des milliers de morts.

Mais plus que notre conscience de tous ces problèmes, nous savons que nous vivons dans une démocratie et que nous pouvons utiliser nos droits et nos pouvoirs en tant que citoyens pour redresser la situation.

\chapter*{\textbf{Deuxième Partie}}
\markboth{Deuxième Partie}{Deuxième Partie}

Mais que signifie vivre dans une démocratie ? On nous dit que la démocratie se distingue de la « dictature » par le fait que les citoyens d'une démocratie participent à la prise de décision, alors que dans une dictature, toutes les décisions sont prises par un dirigeant ou un petit groupe de dirigeants. Cependant, dans les sociétés démocratiques, la plupart des gens ne sont pas membres du gouvernement et n'ont pas de contrôle direct sur les décisions qui affectent leur vie, mais ils doivent néanmoins se conformer à ces décisions. La justification est que les sociétés humaines avancées ne peuvent fonctionner sans gouvernement, et que les citoyens concluent donc un « contrat de gouvernés ». Ils s'engagent à suivre les règles et à honorer les décisions du gouvernement, et le gouvernement, en retour, est tenu de protéger ses citoyens et de défendre le bien commun.

Par conséquent, dans une démocratie, les personnes qui ne peuvent pas devenir membres du gouvernement en raison du nombre limité de postes à pourvoir peuvent voter pour leurs dirigeants, qui sont appelés « représentants » parce qu'ils doivent représenter les intérêts de leurs électeurs sous peine de ne pas être réélus. Le vote est donc le droit fondamental dans un État démocratique, et l'État ne peut être considéré comme démocratique que si la majorité de ses citoyens bénéficient de ce droit. Le deuxième droit le plus important est que chacun doit avoir la possibilité d'être élu à un poste gouvernemental, afin d'empêcher l'existence d'une élite permanente ou héréditaire. L'impossibilité apparente de permettre à chacun de participer de manière égale aux fonctions gouvernementales est surmontée par le mécanisme du vote, qui permet aux citoyens d'\emph{exercer leur contrôle sur le gouvernement tout en minimisant leur participation}, en choisissant des dirigeants qui, dépendant de l'élection, doivent « servir » ceux qu'ils « dirigent ».

Les représentants élus votent également sur les décisions proposées, un vote à la majorité décidant de la question en jeu. L'objectif de la prise de décision majoritaire, du moins selon les mythologies des sociétés démocratiques, est que la règle de la majorité résout les injustices antérieures de la règle de l'élite. D'un autre côté, la règle de la majorité menace les droits des populations minoritaires, en particulier dans les sociétés pluralistes. Pour éviter la loi du plus grand nombre, les sociétés démocratiques offrent également des garanties juridiques, ou « droits », à la plus petite des minorités, l'individu. Ainsi, un groupe minoritaire peut fréquemment devoir accepter des décisions qu'il ne soutient pas, mais au moins les membres de ce groupe jouiront toujours d'un ensemble de droits garantis, tels que la liberté d'expression, de religion et de propriété, afin de préserver leur dignité et leur bien-être fondamentaux. Si les droits d'une personne sont violés, elle a en outre le droit d'intenter une action en justice et d'exiger que ses droits soient respectés.

Pour éviter que le gouvernement ne devienne dictatorial, les différentes fonctions du gouvernement sont séparées et des équilibres structurels sont créés pour garantir qu'aucune branche du gouvernement n'accumule une part disproportionnée de pouvoir. Dans une démocratie, une force de police est nécessaire pour protéger les droits individuels, en particulier les droits à la vie et à la propriété, et (en conjonction avec le pouvoir judiciaire) pour punir ceux qui ne respectent pas les décisions de la majorité (lois) telles qu'elles sont exprimées par le pouvoir législatif. Pour protéger la souveraineté de la population et défendre ses droits de propriété dans les pays étrangers, une armée est nécessaire, bien que, pour éviter une dictature militaire, elle soit exclue du processus décisionnel et de l'application des lois (en articulant la mythologie libérale, nous devons proférer quelques faussetés, en ignorant les nombreuses violations nationales du \emph{posse comitatus} tout au long de l'histoire des États-Unis, et l'utilisation constante de l'armée pour appliquer la politique gouvernementale en dehors de nos frontières).

La dernière question est d'ordre économique. De nombreuses questions importantes ne relèvent pas de la sphère politique, mais de la sphère économique. Par conséquent, les États démocratiques vont de pair avec des économies de marché. Dans une économie de marché, chacun a le droit (légalement garanti par le gouvernement) de posséder la propriété privée, de vendre son travail, d'acheter et de vendre des marchandises et de profiter des bénéfices de son travail et de ses entreprises. D'un point de vue \emph{juridique}, tout le monde a \emph{les mêmes chances} de réussir, et la richesse sera donc distribuée à ceux qui la gagnent, plutôt que thésaurisée dans les mains d'une élite.

C'est du moins ainsi que l'on dit que la démocratie fonctionne, presque exactement comme dans les manuels scolaires produits en masse que les enfants sont obligés de lire, et dans des tautologies et des clichés parfois plus éloquents lorsqu'ils sont régurgités par les commentateurs érudits des médias d'information et des universités. Tout ce qui dépasse l'analyse symbolique du fonctionnement réel de notre système démocratique contredit les explications de la mythologie libérale.

\chapter*{\textbf{Troisième Partie}}
\markboth{Troisième Partie}{Troisième Partie}

Bien sûr, pour beaucoup de gens, l'idéal démocratique n'a pas de sens. La démocratie états-unienne, en particulier, va de pair avec le « libre marché », ce qui signifie que les riches politiciens blancs du Congrès et de la Maison Blanche ne votent aucune loi ni ne prennent aucune mesure susceptible de restreindre la liberté des riches hommes blancs qui siègent dans les conseils d'administration des entreprises et à Wall Street (les politiciens d'hier et de demain) de gagner des milliards de dollars en exploitant leurs travailleurs. Et ce sont ces travailleurs qui constituent la majorité de la population. Ils n'ont pas la possibilité de voter pour leurs patrons ou de décider collectivement de la politique de l'entreprise qu'ils enrichissent grâce à leur travail. Si c'était le cas, ils pourraient voter pour s'octroyer un salaire décent au lieu d'accorder au PDG une nouvelle augmentation de 100 millions de dollars.

À moins d'appartenir au 1 \% le plus riche de la population et d'avoir suffisamment d'argent pour acheter des terres, une usine ou d'autres moyens de production, et d'embaucher des personnes moins fortunées pour travailler pour nous et nous enrichir, notre seule véritable option est de vendre une partie importante de notre vie pour travailler à l'enrichissement de quelqu'un d'autre. Nous sommes certes libres de choisir, parmi une gamme limitée d'options liées à notre classe économique et à notre éducation, pour quelle entreprise travailler, mais elles sont toutes très similaires, car en fin de compte, le patron détient le pouvoir sur le travailleur, et les entreprises peuvent exploiter les travailleurs à des fins lucratives, mais tous les moyens pratiques dont disposent les travailleurs pour obtenir un peu d'équité de la part des entreprises ont été érigés en infractions. Dans ce pays, tout a un propriétaire, et partout où nous allons, pour tout ce que nous utilisons, nous devons payer un loyer. Toutes les activités nécessaires au maintien de la vie sont taxées, de sorte que notre survie dépend de notre capacité à servir les riches qui ont l'argent pour nous payer. C'est ce qu'on appelle l'esclavage salarié. N’est-il pas absurde de parler de liberté et de démocratie à quelqu'un qui est né dans un ghetto, ou à quelqu'un qui vient d'immigrer pour échapper à la pauvreté ou à la persécution, à quelqu'un qui n'a jamais eu la possibilité de recevoir une bonne éducation et qui travaille 80 heures par semaine dans un emploi dangereux et exténuant, sans dignité ni respect, juste pour pouvoir payer le loyer d'un taudis bon marché et un maigre régime alimentaire ?

Et que signifie la démocratie pour les personnes de couleur, qui sont confrontées au profilage, au harcèlement et à la violence de la police, à des taux de pauvreté plus élevés et à des possibilités d'éducation et d'emploi plus limitées ? Sont-elles vraiment censées croire que les riches politiciens blancs se soucient de représenter leurs intérêts ? La société est tellement habituée à considérer les femmes comme des êtres humains de seconde zone que des problèmes tels que le viol, le harcèlement, la discrimination au travail et les idéaux de beauté imposés par la société, qui entraînent de graves problèmes de santé, ne sont pas considérés comme des injustices pertinentes pour notre démocratie, mais plutôt comme des aspects naturels de l'existence humaine. En réalité, les patrons et les travailleurs ne sont pas égaux, les riches et les pauvres ne sont pas égaux, les personnes blanches et les personnes de couleur ne sont pas égales, les hommes, les femmes, et les autres ne sont pas égaux, mais nos attentes à l'égard de la démocratie sont si faibles que peu de gens considèrent ces « problèmes d’ordre sociétal » comme pertinents pour les affaires de notre gouvernement. Tout ce que nous attendons de notre démocratie, c'est le droit de vote et le droit pour les personnes blanches de la classe moyenne de pouvoir se plaindre sans être persécutés. Attendre davantage est un idéal irréaliste, précisément parce que notre gouvernement a rarement accordé plus que ces quelques droits symboliques.

Ainsi, notre expérience ultime de la démocratie est la suivante: une fois toutes les quelques années, nous avons la possibilité de voter pour l'un des deux hommes riches, blancs et chrétiens, tous deux redevables aux intérêts des entreprises, et nous savons que notre vote n'a pas vraiment d'importance, mais si nous participons, c'est généralement parce que nous pensons que l'un des candidats ne nous trahira pas aussi rapidement que l'autre. Le reste du temps, le fait que nous vivions dans une démocratie ne signifie pas grand-chose. Nous avons le droit de critiquer les politiciens, mais le fait de se plaindre ne semble pas changer le fait que le même groupe est au pouvoir. Nous sommes également libres de nous plaindre de l'aspect le plus important de notre vie, notre travail, mais bien sûr, si les patrons nous entendent, ils sont libres de nous licencier. Tout le monde sait que nous vivons dans une démocratie, mais face au racisme et à l'inégalité économique, peu de gens peuvent dire en quoi ce système de gouvernement nous donne réellement du pouvoir.

\chapter*{\textbf{Quatrième Partie}}
\markboth{Quatrième Partie}{Quatrième Partie}

Il est cependant facile de rejeter ces affirmations d'impuissance et d'injustice récurrente en reprochant simplement aux victimes d'être trop paresseuses pour sortir de la pauvreté ou pour faire fonctionner le processus démocratique en leur faveur, par le biais de pétitions, de votes, d'envois de lettres et de toutes les autres méthodes facilement disponibles pour remédier à l'injustice alléguée. Bien entendu, il serait plus que ridicule pour les experts blancs privilégiés qui guident les opinions de la nation depuis leurs talk-shows et leurs colonnes d'opinion de reprocher aux personnes nées dans les ghettos de ne pas surmonter le racisme et la pauvreté s'ils n'avaient pas au moins quelques exemples historiques de la manière dont la démocratie peut réellement fonctionner pour aider les personnes dans le besoin. Mais nos livres d'histoire regorgent d'exemples de groupes de personnes opprimées qui ont gagné leur égalité grâce au processus démocratique. Tout le monde connaît l'histoire de Martin Luther King et du mouvement des droits civiques et, comme n'importe quel écolier peut vous le dire, cette histoire se termine bien, car les personnes noires ont obtenu leurs droits. Face à des préjugés séculaires, le processus démocratique a prévalu. Ou bien, a-t-il vraiment prévalu ?

En fait, le processus démocratique avait déjà réussi à vaincre officiellement le racisme au 19éme siècle, lorsque notre gouvernement a accordé tous les droits légaux sans distinction de race, du moins sur le papier. Et en 1954, une décennie entière avant que le mouvement des droits civiques n'atteigne son apogée, la Cour suprême a ordonné la reconnaissance de ces droits légaux, en réponse au travail inlassable, au sein des voies démocratiques légales, de la NAACP et d'autres organisations. Cependant, il n'y a pas eu de véritable changement dans les relations raciales en Amérique. Toutes les réformes obtenues par le biais du processus démocratique étaient symboliques. Ce n'est que lorsque les personnes noires sont descendues dans la rue, souvent illégalement, en dehors du processus démocratique, que ce que nous appelons aujourd'hui le mouvement des droits civiques a pris toute son ampleur. Le mouvement des droits civiques a utilisé l'activisme illégal (la « désobéissance civile ») en tandem avec la pression légale sur le processus démocratique pour apporter des changements, et même dans ce cas, ce n'est que lorsque des émeutes raciales se sont produites dans presque toutes les grandes villes et que des organisations noires plus militantes se sont formées que l'appareil politique blanc a commencé à coopérer avec les éléments pacifistes et de classe moyenne du mouvement, comme la \emph{Southern Christian Leadership Conference} de Martin Luther King, Jr.

Et quel a été le résultat de ce compromis politique ? Les personnes de couleur aux États-Unis sont toujours confrontées à un chômage plus élevé, à des salaires plus bas, à un accès moindre à un logement et à des soins de santé de qualité, à une mortalité infantile plus élevée, à une espérance de vie plus faible, à des taux d'incarcération et de brutalité policière plus élevés, à une représentation disproportionnellement plus faible au sein du gouvernement, de la direction des entreprises et des médias (sauf en tant que méchants à Hollywood ou coupables dans la série télévisée COPS). En fait, le Dr Kenneth Clark, dont les travaux sur les effets psychologiques de la ségrégation sur les élèves noirs ont contribué à la victoire dans l'\emph{affaire Brown v. Board of Education} en 1954, a déclaré en 1994 que les écoles états-uniennes étaient plus ségréguées qu'elles ne l'étaient quarante ans plus tôt. La suprématie blanche existe toujours dans tous les domaines de la vie états-unienne.

Qu'est-ce que le mouvement des droits civiques a accompli exactement ? L'accès aux institutions dominées par les blancs a été ouvert à un très petit nombre de personnes noires, latines, et asiatiques, en particulier ceux qui adhèrent à l'idéologie conservatrice du statu quo suprématiste blanc, comme le juge de la Cour suprême Clarence Thomas, qui s'oppose aux quotas ou à d'autres mesures juridiques visant à atténuer les inégalités raciales, ou le général Colin Powell, qui est prêt à bombarder des personnes de couleur dans des pays étrangers, au mépris total de leur vie. Martin Luther King est donc mort, mais son rêve se perpétue à travers la poignée disproportionnée de membres noirs et latinos du Congrès, les un ou deux PDG de couleur parmi les 500 plus riches et les émissions de télévision occasionnelles qui dépeignent des familles noires aisées de la classe moyenne, comme les Cosby, à l'abri des brutalités policières ou de l'exploitation économique.

Le gouvernement a conservé son caractère suprémaciste blanc et, plus important encore, il est \emph{plus puissant aujourd'hui} qu'il ne l'était avant le mouvement des droits civiques, car il a largement éliminé la menace des conflits raciaux et des soulèvements motivés par l'oppression ; quelques personnes de couleur tokenisées accèdent à des postes de pouvoir, donnant l'illusion de l'égalité, mais les populations de couleur restent dans l'ensemble un réservoir de main-d'œuvre excédentaire bon marché dont le système peut user et abuser en fonction des besoins. Quand l'on considère la manière dont le gouvernement a réellement réagi au mouvement des droits civiques et les changements qui en ont résulté dans notre société, il apparaît clairement que le processus démocratique a été plus efficace pour sauver les dirigeants d'une situation d'urgence potentielle que pour apporter un réel soulagement ou une véritable libération à un groupe de personnes opprimées.

Et ce ne sont pas seulement les groupes minoritaires qui sont ignorés par le gouvernement. Même dans les situations historiques où la majorité de la population souhaite un changement, ce sont les intérêts des riches et des puissants qui dirigent la décision. Avant l'ère Reagan, une majorité de citoyens était en faveur d'une aide sociale fournie par le gouvernement \emph{afin de garantir à chacun l'accès à un minimum de nourriture, de logement et de soins médicaux}. Puis, pendant plusieurs années, une campagne a été menée par les politiciens et les médias (appartenant aux mêmes entreprises qui faisaient élire les politiciens grâce à des dons massifs), à l'aide de slogans, de publicités, de statistiques manipulées et d'une couverture sélective, pour dépeindre les bénéficiaires de l'aide sociale comme des drogués paresseux profitant d'une situation privilégiée.

Après cette vaste campagne de propagande, une majorité d'États-Uniens interrogés se sont déclarés opposés à l' « aide sociale », mais, curieusement, ils se sont tout de même déclarés en faveur d'un filet de sécurité fourni par le gouvernement \emph{pour garantir à chacun un accès minimal à la nourriture, au logement et aux soins médicaux}. Les médias les avaient programmés pour associer le mot « aide sociale » à un certain nombre de mauvaises choses, même s'ils soutenaient l'idée de l'aide sociale. Les politiciens pouvaient prétendre qu'ils agissaient dans l'intérêt du peuple lorsqu'ils démantelaient l'aide sociale au profit des bénéfices des entreprises, mais en réalité, l'élite travaillait très dur pour s'assurer que le peuple croyait ce qu'elle voulait qu'il croie. Le consentement démocratique a été fabriqué d'en haut.

\chapter*{\textbf{Cinquième Partie}}
\markboth{Cinquième Partie}{Cinquième Partie}

Ce sont les détenteurs du pouvoir et de l'argent qui décident des politiciens à élire. Une personne ne peut être désignée comme candidate à l'un des deux grands partis sans avoir de solides alliances au sein du parti. Par conséquent, avant même que quelqu’un puisse être considéré comme un candidat possible à l'élection, il (ou parfois elle !) doit faire appel à ceux qui sont déjà au pouvoir. Et après avoir reçu la nomination du parti, il est impossible d'être élu au Congrès ou à la Maison Blanche sans une énorme campagne de publicité, qui coûte des millions de dollars. Les entreprises et les particuliers fortunés fournissent la majorité de ces dons, et ils ne contribuent qu'aux campagnes des candidats qui promettent de servir les intérêts des riches. Un politicien qui trahit les entreprises qui le soutiennent, par exemple en soutenant une loi qui obligerait les employeurs à verser un salaire décent à leurs employés, ne sera pas réélu.

Mais le fait que les entreprises de médias, qui informent les opinions et les décisions de chacun, ne sont pas des institutions publiques, mais d'énormes sociétés de divertissement privées, conglomérées et à but lucratif, qui possèdent ou sont possédées par des sociétés d'autres secteurs, est encore plus important. Les entreprises qui fabriquent les produits que vous achetez dans les magasins, qui fabriquent les armes utilisées dans les guerres, les voitures que vous conduisez, l'essence que vous utilisez ; les entreprises qui sous-payent leurs travailleurs, détruisent l'environnement, polluent votre air, achètent vos « représentants » politiques. L'entreprise pour laquelle vous travaillez peut-être.

En outre, ces entreprises reçoivent leur argent d'autres sociétés qui achètent des publicités, et elles représentent les intérêts de ces sociétés, et de leurs PDG riches et blancs, avant de représenter vos intérêts. Que vendent-elles lorsqu'elles vendent de l'espace publicitaire ? Elles vous vendent vous. Vous achetez donc ce qu'on vous dit, vous votez comme on vous le dit et vous n'exercez que les choix limités qu'elles jugent acceptables. Parce qu'il n'est pas directement lié au gouvernement, ce réseau d'entreprises (qui vous fournit la quasi-totalité de vos informations sur le monde) constitue la machine de propagande la plus efficace et la plus crédible de l'histoire du monde.

Un dernier fait important est que les personnes qui contrôlent le gouvernement, les médias et les entreprises sont le même groupe de personnes. Les politiciens de haut niveau arrivent souvent au pouvoir après avoir fait carrière dans de puissantes entreprises et, après avoir mené une carrière fructueuse en tant qu'élus, « au service de leur pays », ils retournent généralement à la vie d'entreprise, gagnant encore plus d'argent en tant que consultants, lobbyistes et dirigeants d'entreprise. Le gouvernement n'a pas besoin de contrôler directement les médias, et les entreprises n'ont pas besoin de contrôler directement le gouvernement, parce qu'ils sont tous dans le même bateau, et qu'ils servent tous les mêmes intérêts: à savoir, les leurs. Après tout, les politiciens travaillent pour les mêmes personnes que les présentateurs de journaux télévisés. Ils ont fréquenté les mêmes écoles, ils vivent dans les mêmes banlieues riches et les mêmes communautés protégées, et entre les sessions du Congrès ou avant le tournage du journal télévisé du soir, ils jouent au golf ensemble.

\chapter*{\textbf{Sixième Partie}}
\markboth{Sixième Partie}{Sixième Partie}

Comment se fait-il que les riches et les puissants soient pris en charge, alors que tous les autres bénéficient de réformes symboliques qui ne résolvent pas leurs problèmes fondamentaux ? Quand est-ce que notre gouvernement démocratique est devenu si corrompu ? La réponse est en fait très simple. Il n'a jamais été corrompu, parce que le gouvernement démocratique a toujours existé pour protéger les intérêts des riches et des puissants. Si l'on va au-delà de ce qui est prôné dans les manuels scolaires et que l'on examine l'évolution réelle de la démocratie, on s'aperçoit qu'il s'agit simplement d'une autre forme de gouvernement dans le continuum historique des royaumes féodaux et des monarchies constitutionnelles. La démocratie n'est pas un nouveau produit de la lutte populaire et de la demande d'égalité face à la tyrannie. Il s'agit d'une \emph{évolution directe d'institutions élitaires antérieures}, créées pour, par et par les élites libérales d'Europe et d'Amérique.

Tout au long de l'histoire de l'Europe post-romaine, l'évolution vers des formes constitutionnelles et électorales n'a pas été le résultat d'une lutte populaire pour la libération. Au contraire, le gouvernement démocratique a été formulé pour apaiser l'aristocratie et la bourgeoisie, qui souhaitaient une \emph{coalition incluant l'ensemble de l'élite économique} dans la direction politique, et pas seulement le monarque et la bureaucratie à lui subordonnée. La démocratie, après tout, n'est pas un concept des Lumières. Le terme même que ces hommes d'État européens et États-Uniens éclairés ont choisi pour décrire le système politique qu'ils souhaitaient est emprunté aux cités-États de la Grèce antique, dans lesquelles tous les citoyens masculins possédant des biens avaient la possibilité d'influencer les dirigeants. Bien entendu, les classes inférieures étaient des esclaves et non des citoyens, de sorte que seuls 10 \% environ de la population pouvaient voter. Dans les premières cités-États, il n'y avait que peu ou pas de distinction entre le pouvoir politique et le pouvoir économique, car l'élite économique était, bien entendu, la bénéficiaire du pouvoir consolidé par les nouvelles structures politiques qu'elle avait créées. Au fur et à mesure que les empires se développaient, une grande partie de l'élite économique - les propriétaires terriens aristocratiques - était souvent exclue, dans une certaine mesure, du groupe d'élite détenant le pouvoir sur et à partir de l'appareil politique centralisé. C'est la lutte de l'aristocratie, et plus tard des marchands bourgeois, des banquiers et des propriétaires d'usines, pour se réincorporer dans l'élite politique, qui est à l'origine de l'évolution de ce processus politique que nous appelons la démocratie.

Aujourd'hui, nous voyons plus clairement l'évolution de la démocratie dans les États-nations européens, dont on dit souvent qu'elle a commencé avec la Magna Carta. Ce célèbre document, ainsi que les droits et garanties juridiques qu'il établit, a été créé lorsque le roi Jean d'Angleterre, confronté à la perspective d'être déposé militairement par l'aristocratie, a jugé bon d'étendre le pouvoir politique à une partie plus large de l'élite économique qu'auparavant, en garantissant des \emph{droits} aux principaux propriétaires terriens et en créant le précédent d'un conseil de barons, ou de leurs représentants, chargé de conseiller le roi et de négocier avec lui.

Le chancelier Bismarck, qui a unifié l'Allemagne en une démocratie constitutionnelle, n'était pas un populiste. Au contraire, son règne s'est caractérisé par une répression sévère des éléments progressistes et radicaux et par un désir machiavélique de renforcer l'État allemand. Il a tellement bien réussi que dans l'espace de quelques décennies, l'Allemagne est passée d'un ensemble de provinces arriérées et désunies à un État-nation uniforme, capable de menacer à lui seul le reste du continent. \emph{Bismarck savait que l’établissement d'élections et de droits constitutionnels ne ferait que consolider le pouvoir de l'élite dirigeante allemande}, en gagnant la loyauté de la bourgeoisie et de l'aristocratie, en épuisant ou en cooptant les efforts des progressistes qui cherchaient à changer la société par le biais du processus électoral et en marginalisant les radicaux qui rejettaient le « processus démocratique », éliminant ainsi le spectre de la résistance ou de la non-coopération qui nuisaient à l'efficacité de nombreux autres États européens qui essayaient constamment de gagner l'obéissance de leurs sujets opprimés. En outre, le \emph{pouvoir} politique et économique, jamais redistribué, \emph{était déjà consolidé entre les mains de l'élite}, qui pouvait s'assurer que seuls ses candidats étaient élus et que seules des lois favorables étaient adoptées, par divers moyens légaux ou illégaux (la légalité étant ici une question farfelue, car la police, historiquement partie intégrante de l'appareil monarchique, n'était pas prête à arrêter ses propres maîtres).

L'évolution sporadique de la démocratie en Russie a suivi une voie similaire à celle de l'Angleterre et de l'Allemagne, à la différence près que la plupart des réformes libérales ont été abrogées par un tsar jaloux, peu enclin à partager son pouvoir. L'existence d'un parlement russe a temporairement atténué l'agitation populaire, mais lors de sa dissolution, les courants subversifs qui ont finalement conduit à la révolution bolchevique ont repris de plus belle. Le parlement russe, actuellement appelé Douma, s'appelait au 19th siècle, avec un peu plus de candeur, la « Douma de Boyarskoe » (« Douma » voulant dire pensée, et les « boyards » étant l'aristocratie russe). Avant cela, les serfs russes ont été « libérés » en tant qu'étape nécessaire de l'évolution démocratique. Bien entendu, ils n'ont pas reçu la \emph{terre} qu'ils avaient travaillée et sur laquelle ils vivaient (et dont ils dépendaient pour leur survie) ; cette terre est restée entre les mains de l'aristocratie, bien que les serfs aient été autorisés à en acheter environ un tiers. Étant donné qu'ils étaient jusque là des travailleurs non rémunérés et qu'ils n'avaient pas d'argent pour acheter la terre, certains des serfs « libérés » ont dû se rendre dans les villes et travailler comme salariés dans les nouvelles usines (un arrangement qui, par coïncidence, convenait parfaitement aux propriétaires des usines et à l'élite politique russe, qui avait besoin de l'industrialisation pour rester une puissance européenne compétitive), tandis que les autres anciens serfs sont restés à la campagne et ont travaillé comme métayers pour leurs anciens maîtres.

Les premiers organes représentatifs du gouvernement, les précurseurs du Congrès ou du Parlement moderne, étaient dès le départ censés représenter l'aristocratie, les propriétaires fonciers, les banquiers et toutes les autres personnes riches qui contrôlaient la vie économique de la nation. La représentation de l'élite économique garantissait que les dirigeants politiques (anciennement le monarque) qui contrôlaient l'armée, la police, le système fiscal et d'autres bureaucraties, protégeraient et serviraient les intérêts des riches. La singularité du monarque a été remplacée par une coalition de l'élite, divisée en partis politiques et rivalisant pour l'influence, mais surtout collaborant au niveau fondamental \emph{pour maintenir le contrôle}. Le vote a permis de garantir que le parti ayant la stratégie de contrôle la plus populaire puisse la mettre en œuvre, alors qu'auparavant, le conservatisme et l'obstination d'un souverain unique et incontesté risquaient d'être moins flexibles pour s'adapter aux changements de circonstances.

Au fur et à mesure que le droit de vote s'est étendu à tous les citoyens adultes (en phase, ce qui n'est pas une coïncidence, avec l'essor des médias de masse contrôlés par les entreprises), le vote a également servi à donner l'illusion de l'égalité, à créer une soupape de décompression pour le mécontentement populaire et, surtout, à \emph{maintenir l'efficacité du contrôle gouvernemental} en favorisant les partis politiques qui réussissaient le mieux à duper la population et à gagner son obéissance. L'absence de participation réelle de la population est d'autant plus évidente que les choix des électeurs sont principalement guidés par la reconnaissance du nom, l'affiliation à un parti et le bombardement de slogans superficiels par les médias publicitaires, et que \emph{peu d'électeurs sont capables de formuler une différence factuelle entre les programmes des candidats opposés,} - et encore moins une analyse critique de leurs politiques.

Le système bicaméral, caractéristique des États-Unis et d'autres démocraties, s'est re-développé en Angleterre, où les deux chambres parlementaires ont été nommées avec plus d'honnêteté qu'il ne serait permis de le faire à l'époque moderne. La Chambre des Lords a été créée pour les représentants de l'aristocratie, et la Chambre des Communes pour ceux qui n'avaient pas de titre de noblesse - plus précisément, pour les représentants de la bourgeoisie ou de la classe moyenne supérieure. L'exclusion de la majorité de la population, même de la chambre la plus basse du parlement, devient évidente lorsqu'on essaie de trouver des roturiers pauvres et issus de la classe ouvrière parmi les membres du parlement, tout au long de l'histoire de la Chambre des communes jusqu'à aujourd'hui. Il en va de même pour les membres du Congrès américain, dont le revenu moyen avant leur élection ne s'est jamais approché de la faible moyenne de l'ensemble de la population états-unienne. Les quelques représentants issus de la classe moyenne inférieure occupent généralement des postes bien rémunérés de consultants d'entreprise après un mandat réussi au Congrès, et aucun homme politique au niveau national n'est issu de la classe inférieure, qui constitue la grande majorité de la population totale.

Il ne s'agit en aucun cas d'une évolution récente du gouvernement américain. Certains des pères fondateurs envisageaient le rôle du président comme celui d'un roi et ont suggérés divers titres majestueux. Comme à l'époque la majorité des gens étaient analphabètes, l'élite pouvait être beaucoup plus directe, et ses commentaires sont très éclairants. Le père de la Constitution, James Madison, a écrit que: « La minorité des opulents {[}les classes riches{]} doit être protégée de la majorité ». Son ami et collègue fédéraliste influent, John Jay, a dit plus clairement que « les gens qui possèdent le pays doivent le gouverner ». La révolution démocratique en Amérique a été la tentative réussie de l'élite économique états-unienne de s'emparer du pouvoir politique des mains des Britanniques. Les plaintes concernant la fiscalité britannique injuste étaient les plaintes d'hommes d'affaires. Lorsque les paysans États-Uniens, déçus que leur situation économique difficile ne s'améliore pas après la révolution, se sont révoltés contre la nouvelle élite états-unienne dans les capitales des États, les Pères fondateurs (qui étaient des marchands, des banquiers et des avocats du Nord et des propriétaires terriens esclavagistes du Sud) se sont réunis pour créer un gouvernement plus fort et centralisé qui protégerait les intérêts des minorités, c'est-à-dire les intérêts de l'élite dirigeante.

La nouvelle Constitution a créé un certain nombre de structures et de droits, les droits étant les privilèges codifiés de l'élite. Un système électoral permettait à ceux qui possédaient la terre, les banques et les usines de décider quels hommes politiques représenteraient le mieux leurs intérêts. Avec l'élargissement du droit de vote, les élections ont également eu pour fonction de tester quel candidat avait la meilleure rhétorique populiste, la meilleure stratégie pour conserver la soumission et la loyauté de l'ensemble de la population. Le fameux équilibre des pouvoirs aux États-Unis, entre les juges, les sénateurs, les présidents et les généraux, est une coalition dirigeante au sein de l'élite. La liberté d'expression était et reste la liberté pour les membres de l'élite de critiquer la politique gouvernementale afin de formuler des stratégies de domination plus efficaces. Incidemment, la liberté d'expression permet également à tout citoyen ordinaire de marmonner ce qu'il veut, bien que l'histoire états-unienne montre constamment que les gens ne sont pas libres de la menace d'arrestation et d'emprisonnement pour des propos impopulaires si les autorités craignent que ces propos aient un effet réel, au-delà de souffle gaspillée dans une conversation futile.

Dans la mythologie libérale, la démocratie repose sur l'idée que les gens se regroupent sous la protection d'un gouvernement et concluent un « contrat de gouvernés ». Mais il s'agit d'un contrat que nous ne pouvons ni négocier ni refuser. Nous naissons tous en tant que sujets de l'un ou l'autre État, « démocratique » ou non, et si nous nous opposons à notre assujettissement, nous ne pouvons rien y faire. Même si nous avons les moyens financiers de quitter notre pays d'origine (sans parler de la question de faire partir le gouvernement de nos maisons), nous n'avons pas d'autres options: Le « No Man's Land » n'existe pas. \emph{Si nous n'avons pas le choix pratique de refuser, notre acquiescement n'est pas un consentement, c'est une soumission}.

En réalité, le processus démocratique est conçu pour former et maintenir une coalition efficace au sein de l'élite, pour gagner la loyauté de la classe moyenne en lui accordant des droits et des privilèges symboliques, pour prévenir le mécontentement en créant l'illusion de l'équité et de l'égalité, et pour étouffer la rébellion en établissant un ensemble élaboré de canaux officiels pour la dissidence sanctionnée ; et d'étouffer la rébellion en établissant un ensemble complexe de canaux officiels pour la dissidence sanctionnée, en épuisant l'énergie des dissidents respectueux de la loi qui passent par toutes les étapes - et obtiennent éventuellement quelques concessions mineures, et en refusant la légitimité à ceux qui sortent du « processus démocratique » pour provoquer directement le changement qu'ils recherchent, plutôt que de participer au rituel de cour élaboré conçu pour démontrer leur loyauté en demandant au gouvernement de prendre en considération leurs requêtes. Une fois que ces rebelles peuvent être décrits comme « illégitimes », « imprudents », « impatients », « inconsidérés » ou « manquant de respect pour le processus démocratique », le gouvernement peut en toute sécurité les traiter beaucoup plus durement que ceux qui honorent encore le « contrat des gouvernés » par leur docilité et leur soumission.

\chapter*{\textbf{Septième Partie}}
\markboth{Septième Partie}{Septième Partie}

Notre analyse plus approfondie de ce système que nous appelons « démocratie » nous a conduits à l'hypothèse suivante: à la base, la démocratie est un système de gouvernement autoritaire et élitiste conçu pour former une coalition dirigeante efficace tout en créant l'illusion que les sujets sont en fait des membres égaux de la société, qui contrôlent donc la politique du gouvernement, ou du moins y sont représentés de manière bienveillante. L'objectif fondamental d'une démocratie, comme de tout autre gouvernement, est de maintenir la richesse et le pouvoir de la classe dirigeante. La démocratie est innovante en ce qu'elle permet à une plus grande diversité de voix de la classe dirigeante de défendre diverses stratégies de contrôle, et « progressiste » dans le sens qu'elle permet de s'adapter pour maintenir le contrôle dans des circonstances changeantes.

Le moyen le plus sûr de vérifier cette hypothèse est d'observer des exemples historiques dans lesquels les citoyens opprimés ou défavorisés d'une démocratie ont défendu leurs propres intérêts, en contradiction avec les intérêts des riches et des puissants. Si le mythe libérale concernant la démocratie est correct, les opprimés seront justement représentés, des représentants politiques défendront leur cause et un compromis équitable sera trouvé entre les privilégiés et les opprimés. Si les progressistes et autres réformistes ont raison de croire que le système est fondamentalement sain mais corrompu par diverses causes qui peuvent être résolues par une législation appropriée, alors les riches et les puissants bénéficieront d'avantages injustes dans les processus législatifs et judiciaires mis en œuvre pour parvenir à la justice. Si notre hypothèse sur la nature autoritaire et élitiste de la démocratie est correcte, les nombreuses institutions du pouvoir collaboreront pour diviser l'opposition, gagner les éléments réformistes et écraser l'opposition restante afin de conserver le contrôle par tous les moyens nécessaires, y compris la propagande, la calomnie, le harcèlement, l'agression, l'emprisonnement sur la base de fausses accusations et l'assassinat.

Les éléments les plus militants ou radicaux de la lutte contre l'oppression raciale dans les années 1960 en sont un excellent exemple. Les inégalités raciales de l'époque sont solidement documentées comme étant fortes et omniprésentes, et de nombreuses organisations se sont formées pour combattre cette oppression raciale. Les \emph{Black Panthers}, par exemple, réclamaient plus que de voir des personnes noires de la classe moyenne. Ils voulaient la libération des personnes noires, une transformation sociale totale qui éliminerait la suprématie blanche de tous les aspects de la vie. En réponse aux brutalités policières, ils ont également commencé à prôner l'autodéfense des personnes noires. Comment les contrôleurs du processus démocratique ont-ils réagi ? À la fin des années 1960, J. Edgar Hoover, chef du FBI, les a qualifiés de « plus grande menace pour la sécurité intérieure des États-Unis ». En grande partie grâce à un programme du FBI appelé COINTELPRO, les Black Panthers ont été harcelés, calomniés, battus, intimidés, leurs communications ont été interceptées et trafiquées pour provoquer des scissions entre les factions. Leurs efforts, y compris les programmes alimentaires pour les enfants des écoles, ont été sabotés ; le FBI et la police locale ont acheté des informateurs et placé des provocateurs dans leurs rangs, ou ont arrêté à plusieurs reprises des organisateurs des Panthers par d'accusations sans fondement pour les faire payer une caution, les harcelant et épuisant leurs ressources. Des Panthers ont été arrêtées et condamnées sur la base d'affaires fabriquées de toutes pièces. Dans un cas, un Black Panther a été emprisonné pendant plus de vingt ans pour des meurtres qu'il ne pouvait pas avoir commis, puisqu'il se trouvait à des centaines de kilomètres de là, dans une autre ville, au moment des faits. Il a défendu son alibi devant le tribunal en affirmant que le FBI avait installé des micros dans le bureau des Panthers où il travaillait et que les enregistrements du FBI prouveraient ses allées et venues. Au tribunal, les agents du FBI ont menti à la barre et nié qu'ils effectuaient une telle surveillance, bien qu'ils aient été contraints par la suite de publier des documents prouvant le contraire. Ils avaient « perdu » les enregistrements des jours en question (ça tombait bien!).

Et lorsque l'emprisonnement ne suffisait pas, les militants des Black Panthers étaient tout simplement assassinés. En l'espace de deux ans, vingt-huit Panthers ont été tuées (certaines dans leur sommeil) par la police et le FBI. Même si les Panthers étaient aussi violentes et impures que le prétendent les plus enragés et mal informés de leurs détracteurs, pourquoi le gouvernement a-t-il traité (aux niveaux local, étatique et national) une organisation bien plus violente, le Ku Klux Klan, avec autant de tolérance (et, dans de nombreux cas, de collaboration) ?

MOVE, une autre organisation de libération des personnes noires, basée à Philadelphie, a été bombardée par un hélicoptère de la police au cours d'un affrontement massif qui s'est soldé par la mort d'un policier. Plusieurs des membres de MOVE sortis de leur maison après le raid ont été battus presque à mort par la police. Huit membres du MOVE ont été emprisonnés, même si les preuves médico-légales (dont une grande partie a été falsifiée par la police) suggèrent que le policier a été tué par un tir d’un autre policier. Plus que la question de savoir si le policier a été tué par l'un des siens ou s'il a été abattu en état de légitime défense par des membres du MOVE, la question de savoir pourquoi exactement la police a organisé un assaut armé contre la maison du MOVE est plus importante.

Le \emph{American Indian Movement} a été traité de la même manière. Leurs membres étaient victimes de harcèlement, d'assassinats et de faux emprisonnements (leur prisonnier politique le plus célèbre étant Leonard Peltier, qui purge une peine de prison à vie pour avoir tué un agent du FBI lors d'un raid, même si l'accusation a admis qu'elle ne pourrait jamais être sûre de l'auteur du coup de feu fatal).

Le recours à la violence par notre gouvernement démocratique contre les dissidents se poursuit encore aujourd'hui. Lors des réunions de l'Organisation mondiale du commerce à Seattle, en 1999, lorsque les manifestants ont été plus nombreux que prévu à se mobiliser et à bloquer le sommet, la police a réagi violemment, en frappant les manifestants et les passants, en les aspergeant de gaz lacrymogène et en leur tirant des balles en caoutchouc. Pour disperser les manifestants enfermés, ils leur ont forcé la tête, les ont aspergés de gaz poivré sous les paupières et ont utilisé d'autres techniques de torture. Tout cela a été filmé, mais les médias nationaux ont ignoré les brutalités policières et ont préféré diffuser des clips montrant des manifestants brisant des vitrines, en les présentant comme la raison de l'intervention massive de la police, alors que la chronologie réelle était inversée.

Au cours de l'été 2002, la police de Washington a fait une descente dans la communauté Olive Branch, un collectif de pacifistes et d'anarchistes politiquement actifs, et a expulsé les résidents sous la menace d'une arme. En 2003, un homme a été arrêté à l'aéroport d'Atlanta parce qu'il tenait une pancarte pour protester contre l'arrivée du président GW Bush. Il a été accusé d'avoir mis en danger le président. Un peu plus tard la même année, l'anarchiste Sherman Austin, webmaster d'un site web à succès consacré aux luttes des personnes de couleur, a été condamné à un an de prison après qu'\emph{une autre personne a} publié sur son site web un lien vers des instructions pour la fabrication de cocktails Molotov sur son site web. Pour ce crime, des agents fédéraux munis d'armes automatiques ont encerclé sa maison, défoncé sa porte et l'ont tiré du lit. La violence et la répression autoritaires sont quotidiennes, trop fréquentes pour être toutes citées. Ces exemples ne sont donc que des exemples parmi d'autres. Pour le reste, vous devrez faire vos propres recherches.

Certains libéraux qui veulent croire que la violence du gouvernement américain n'est que le résultat de services de police corrompus et non un élément fondamental et nécessaire du système idéalisent souvent d'autres pays, en particulier les démocraties sociales d'Europe, en utilisant leur ignorance de la violence autoritaire dans ces pays comme preuve de l'absence d'une telle violence. Avec un peu de recherche, nous découvrons que les gouvernements démocratiques du Canada, de l'Allemagne, de la Grande-Bretagne, du Mexique, du Japon, de l'Italie et d'autres pays ont également recours à une violence régulière à l'encontre des dissidents.

\chapter*{\textbf{Huitième Partie}}
\markboth{Huitième Partie}{Huitième Partie}

La question demeure: Que faire ? Malheureusement, trop de gens adhèrent aux contraintes artificielles du système, choisissant toujours le moindre des deux maux, motivés uniquement par la peur d’un mal plus grand, comme s'ils étaient impuissants à remettre en question le cadre social et à créer de nouvelles alternatives (ce constat d'impuissance au sein du système démocratique devrait suffire pour que les gens se révoltent !). Rien dans les lois physiques de l'univers, ni aucune règle régissant le comportement humain, n'exige que le monde soit dominé par une élite ploutocratique exerçant un contrôle autoritaire et exploiteur sur tous les autres. En fait, la majorité des sociétés humaines se sont organisées très différemment, souvent sous des formes égalitaires, jusqu'à ce que l'impérialisme européen et américain interrompe toutes les autres expériences culturelles et les remplace par les nôtres, de sorte que presque tous les pays du monde pratiquent la démocratie représentative et le capitalisme industriel, qui sont des formes d'organisation socio-économique très particulières, entièrement eurocentriques et largement inadaptées (sauf en termes de maintien du contrôle et d'exploitation de la valeur).

Pour de nombreux progressistes états-uniens, l'idée d'envisager de nouvelles alternatives consiste à soutenir les troisièmes partis, comme si l'existence de troisièmes et de quatrièmes partis avait rendu les États européens moins oppressifs. Demandez aux Roms si le parti vert a fait la moindre différence lorsqu'ils ont été expulsés en masse d'Allemagne, plus de quarante ans après la fin du Troisième Reich. Demandez aux manifestants de Gênes, qui ont été mis contre un mur et battus jusqu'à ce que leur sang et leurs dents décorent le béton, ce qu'ils pensent d'un système parlementaire. D'autres progressistes sont favorables à ce qu'ils considèrent comme des changements structurels, tels que des amendements à la Constitution, sans se rendre compte que le pouvoir n'existe pas sur le papier. Ces réformistes croient peut-être que l'égalité raciale aux États-Unis a été atteinte en 1868, avec l'adoption de l'amendement 14th , ou que le mouvement des droits civiques s'est achevé en 1964, avec la loi sur les droits civiques. Pour corriger leur naïveté, ils n'ont qu'à passer un peu de temps dans une prison et à rechercher le degré de protection que le quatrième amendement a accordé aux détenus toxicomanes de ce pays.

Dans les rares cas où le processus démocratique a « fonctionné », le système entier n'hésite pas à ignorer les lois de réforme qui contredisent les intérêts des puissants. Jimmy Carter, le président le plus libéral que les États-Unis aient jamais connu (mais qui n'est pas un saint, si l'on en croit les expériences des Cambodgiens, des Indonésiens, des Haïtiens et d'autres), a interdit par décret plusieurs programmes de contre-espionnage de l'ère vietnamienne qui incluaient la torture et l'assassinat. Grâce à un officier consciencieux de l'École des Amériques de l'armée états-unienne, nous savons que l'armée a tout simplement ignoré l'ordre de Carter et a continué à enseigner ces tactiques. Combien d'exemples similaires sont restés secrets ?

Dans une société où le pouvoir est tellement concentré entre les mains de quelques-uns, \emph{le pouvoir se défend lui-même}. Pensons-nous vraiment que si nous élisons un président ou un congrès « décent », toutes les institutions de l'élite qui s'auto-perpétuent se contenteront d'abandonner et de céder leurs richesses ? Dans les pays où les organes élus du gouvernement ont cessé de représenter les intérêts des puissants, l'armée et les entreprises qui la soutiennent (la coalition de l'élite) ont conspiré pour renverser les parties rétives du gouvernement (au Chili, au Venezuela, en Espagne, au Congo, etc.). Les entreprises et les armées d'Europe et d'Amérique du Nord sont-elles d'une manière ou d'une autre plus pures ? Après tout, c'est le Pentagone (ou Exxon) qui a parrainé nombre de ces coups d'État élitistes (souvent fascistes ou nationalistes d'ultra-droite) à travers le monde.

Les citoyens des démocraties modernes sont tellement paralysés par une peur tenace de l'action autonome et directe - prendre l'initiative de faire les choses par nous-mêmes et de résoudre nos propres problèmes - que prôner le renversement révolutionnaire de l'ordre actuel semble revenir à prôner l'apocalypse ; pourtant, les deux actions essentielles que nous devons entreprendre pour nous libérer sont l'autonomie et l'abolition des relations sociales, politiques et économiques actuelles.

Nous ne pouvons tout simplement pas continuer à attendre que d'autres personnes nous sauvent. C'est notre dépendance à l'égard de Big Brother qui perpétue les erreurs du système. Comme un muscle inutilisé, notre capacité à prendre soin de nous-mêmes, à prendre nos propres décisions, à régir nos relations avec les autres, à créer des associations volontaires et à construire des communautés, à résoudre nos différends et, surtout, à nous faire confiance, s'est atrophiée, mais nous devons affiner ces capacités pour nous libérer de la domination autoritaire qui nous gouverne depuis des millénaires.

Deuxièmement, \emph{nous ne pouvons pas continuer à considérer l'égalité} - la \emph{véritable égalité} - \emph{comme une mesure extrême}. C'est le système actuel qui est extrême et nous devons en détruire tous les vestiges pour nous en libérer et l'empêcher d'évoluer vers une nouvelle forme déguisée. Le gouvernement, sous quelque forme que ce soit, est autoritaire. De même, le pendant du gouvernement démocratique - le système économique du « libre marché », qui n'est jamais né ou entré en contact avec le mythique « level playing field » que les économistes libéraux envisagent pour justifier leur système, est une autre structure de gouvernement (relative aux moyens de production et de consommation, plutôt qu'à l'appareil politique) qui permet à l'État de s'affranchir de l'autorité de l'État, plutôt que l'appareil politique) qui permet un certain degré de concurrence et de participation ayant l'apparence de l'équité et de l'ouverture mais qui, en réalité, est conçue pour augmenter l'efficacité du contrôle des moyens de production tout en conservant ce contrôle entre les mains d'un groupe dont la composition peut être quelque peu fluctuante mais qui reste clairement un groupe d'élite. Dans ce système de marché libre, un très petit nombre de personnes contrôlent les moyens de production (les usines, la terre, etc.), ce qui rend l'autosuffisance impossible. Pour se procurer les produits nécessaires à la survie et à une existence culturellement normale, tous les autres doivent vendre leur activité contre un salaire. La seule façon de corriger la situation est de reprendre ce qui nous a été volé.

La production et la prise de décision doivent être décentralisées, et la richesse et le pouvoir doivent être partagés au niveau de la communauté dont ils sont issus. Les structures étatiques doivent être démantelées, les richesses et les moyens de production doivent être confisqués à la minorité qui les contrôlent, les prisons détruites, les armées anéanties. Des formes plus intimes d'oppression, comme le patriarcat et la suprématie blanche, doivent être dénoncées et remises en question partout où elles persistent.

\chapter*{\textbf{Neuvième Partie}}
\markboth{Neuvième Partie}{Neuvième Partie}

L'expression « plus facile à dire qu'à faire » est un très grand euphémisme. Peut-être la raison pour laquelle tant de gens continuent à croire en l'efficacité de réformes mineures, face à de l’évidence contradictoire écrasante, est que l'énorme responsabilité à laquelle nous sommes confrontés en réalisant que les problèmes de notre société sont fondamentaux, et non superficiels, semble impossible à assumer. Mais nous ne savons jamais si quelque chose est possible tant que nous n'avons pas réussi. En attendant, notre préoccupation est de trouver les stratégies de résistance et d'organisation les plus efficaces.

Heureusement, l'histoire de la résistance est aussi longue que l'histoire de l'oppression, et nous avons donc de nombreux exemples dont nous pouvons nous inspirer. Pour améliorer nos propres efforts en vue de réaliser la révolution, nous devrions examiner comment les activistes à travers l'histoire ont réussi à affronter le pouvoir et à produire des changements et comment ils ont été inefficaces, tout en gardant à l'esprit leur contexte spécifique.

Dans l'histoire des États-Unis, le syndicat occupe une place traditionnelle en tant que vecteur de l'activité révolutionnaire. Au début du vingtième siècle, les syndicats ont proposé une critique radicale des inégalités sociales et ont offert aux esclaves salariés du pays la promesse d'une vie meilleure. Les syndicats sont devenus une force politique puissante, gagnant des millions de membres, organisant des grèves et des manifestations, et créant également des comités de défense lorsque la police a commencé à massacrer les travailleurs en grève. Bien qu'ils soient parvenus à réduire certaines des brutalités auxquelles les travailleurs étaient confrontés, les syndicats n'ont pas réussi à corriger les inégalités sociales sous-jacentes et ont fini par trahir les travailleurs. Aujourd'hui, la plupart des syndicats sont des rackets à l'esprit étroit qui n'ont que peu d'influence réelle. L'un des facteurs importants de leur échec est la structure hiérarchique de la plupart des syndicats. La hiérarchie s'est développée pour permettre à des groupes d'élite de contrôler des populations plus importantes. En conséquence, les organisations hiérarchiques sont facilement détournées par les gouvernements qu'elles contestent. Les syndicats ont été infiltrés et leurs dirigeants ont été cooptés. Les dirigeants syndicaux ont facilement confondu les intérêts de leur organisation avec les intérêts de la lutte sociale pour laquelle les syndicats avaient été créés. Les activités syndicales radicales étant sévèrement réprimées, les dirigeants syndicaux ont développé des relations plus coopératives avec les politiciens et les patrons afin d'assurer la survie de leur syndicat et le maintien de leurs positions de pouvoir de plus en plus confortables. Les syndicalistes radicaux qui ne pouvaient pas être achetés ont été emprisonnés ou neutralisés d'une manière ou d'une autre.

Une autre faiblesse majeure de la plupart des syndicats était leur rejet des questions de race et de genre, qui étaient inséparables des questions économiques. En refusant de s'attaquer au racisme, au sexisme et à la xénophobie, et en maintenant au contraire une critique privilégiée et étroitement économique du capitalisme, les syndicats sont devenus des organisations d'hommes blancs, perdant le soutien vital des ouvrières de l'habillement et des employées de maison, des métayers noirs et des travailleurs immigrés dans les usines. Leur incapacité à critiquer les aspects suprémacistes du capitalisme a permis aux patrons de conserver le pouvoir en divisant et en privant les travailleurs de leurs moyens d'action, en faisant des étrangers et des personnes noires émancipées les boucs émissaires de leur pauvreté.

C'est en partie le désir des grands syndicats d'être respectables qui les a conduits à perpétuer les comportements racistes, sexistes et élitistes de la structure de pouvoir qu'ils cherchaient à l'origine à vaincre. Leurs positions d'autorité et les négociateurs du gouvernement ont fait miroiter le pouvoir - le confort, la dignité et le respect - aux dirigeants syndicaux, qui ont fini par oublier les causes des maux sociaux qu'ils dénonçaient, et se sont plutôt appuyés sur la satisfaction d'être acceptés par la société (la haute société) pour endormir les symptômes. En refusant les compromis, en utilisant des tactiques radicales ou militantes, ou en remettant en cause le statu quo racial et sexuel, ils savaient qu'ils seraient ostracisés par le gouvernement et vilipendés par les médias. Les syndicats se sont donc efforcés de devenir respectables aux yeux du grand public, et comme ce qui est grand public est déterminé par les médias, cela signifiait qu'ils devaient plaire aux classes moyennes et supérieures blanches. Ce faisant, les syndicats ont dû renoncer à leur plus grande source de force, la détermination des opprimés à gagner leur liberté, qui se manifeste souvent par une rage inconvenante pour ceux qui ont beaucoup à perdre à ce que des mécontents fassent tanguer le bateau.

Malgré les échecs historiques des syndicats, tant que le travail salarié sera répandu dans la société et dans la vie de l'individu, la relation entre le travailleur et le patron sera un point nodal important pour l'agitation. L'Industrial Workers of the World, un syndicat qui recherche le contrôle des travailleurs sur les moyens de production et l'abolition finale du capitalisme, a fait preuve d'un esprit anti-autoritaire plus résistant que ses contemporains, qui ne font aujourd'hui guère plus que fournir des tampons au parti démocrate.

Récemment, de nombreux militants luttant contre l'oppression ne s'affilient pas à une seule organisation, mais s'efforcent de dénoncer et d'atténuer l'oppression là où elle est le plus durement ressentie. Souvent, les radicaux privilégiés qui se méfient du réformisme sont réticents à travailler pour une cause qui n'a pas d'objectifs révolutionnaires clairement articulés et à long terme, et ils rejoignent donc des organisations plus abstraites qui sont orientées au niveau national ou mondial, plutôt que local. Cependant, les pauvres et les personnes de couleur n'ont pas besoin de sortir de leur propre communauté pour trouver des brutalités et des dépravations qui doivent être surmontées. En conséquence, les radicaux des groupes privilégiés seront séparés des radicaux des groupes ciblés par l'oppression. Les militants masculins blancs de la classe moyenne doivent réaliser que les programmes de lecture, les cliniques du SIDA, les soupes populaires, les refuges pour personnes sans domicile fixe, les refuges pour femmes battues, les programmes de surveillance des flics et les groupes de soutien aux prisonniers, ainsi que d'autres programmes de « premiers secours » peuvent être révolutionnaires et, plus important encore, qu'ils sont nécessaires à la santé et à la survie des communautés opprimées.

Certaines organisations nationales, telles que Food Not Bombs ou Homes Not Jails, combinent les efforts visant à traiter directement les symptômes de l'oppression avec une mise en cause radicale des structures de pouvoir qui sont à l'origine de ces symptômes. Food Not Bombs sert des repas gratuits dans des lieux publics, invitant à la prise de conscience de problèmes tels que la faim et la pauvreté, et remettant en question les causes de ces problèmes. Homes Not Jails squatte et répare des appartements abandonnés, condamnés ou vacants, en violation des « droits de propriété » des propriétaires, afin d'offrir un toit aux familles sans domicile fixe. En recourant à l'action directe illégale et à la désobéissance civile, ils illustrent la manière dont le système juridique protège les propriétaires aux dépens des pauvres et mettent en évidence le rôle du gouvernement et du capitalisme dans la création et le maintien de la pauvreté. Il convient de noter que ces groupes sont organisés de manière décentralisée et non hiérarchique. Food Not Bombs, par exemple, est plus une idée qu'une institution. N'importe qui, n'importe où, peut créer une section de Food Not Bombs, sans avoir à demander l'autorisation du siège national (il n'y en a pas) ni à payer de cotisation. En conséquence, les membres de chaque chapitre peuvent adapter le modèle Food Not Bombs aux conditions et aux besoins locaux, et sans politique institutionnelle ni conférence nationale, les membres ne perdent pas d'efforts dans le maintien de l'organisation et peuvent consacrer plus de temps à répondre aux besoins locaux. Cependant, comme Food Not Bombs est en grande partie le produit de cercles d'activistes blancs privilégiés de la classe moyenne, de nombreux chapitres s'enlisent dans un schéma consistant à fournir un repas hebdomadaire gratuit symbolique et à ne pas aller plus loin dans la lutte contre la faim. La plupart des membres de Food Not Bombs ne connaissent pas personnellement la faim, et il semble qu'au moins certains d'entre eux aient l'idée qu'en fournissant un service aux personnes pauvres et opprimées de la communauté, ils les « radicaliseront », créeront des alliances et déclencheront une masse critique, puis tout le monde se soulèvera en révolution, d'une manière vague et magique. Si, au lieu de reprocher inconsciemment aux opprimés (qu'ils ont été formés depuis leur naissance à considérer comme des ignorants) de ne pas s'engager dans la lutte contre le « militarisme » et le « capitalisme », ils décidaient d'augmenter continuellement le niveau de la lutte contre la faim, au-delà d'un repas par semaine, ils pourraient peut-être constater qu'il n'y a pas eu de changement dans la façon dont les opprimés ont été traités, ils découvriront peut-être qu'il n'y a pas de moyen plus efficace de lutter contre le capitalisme et, dans le même temps, de soulager les symptômes de ceux qui en souffrent le plus, car le capitalisme ne peut tout simplement pas fonctionner si la faim n'est pas une menace imminente qui motive les gens à travailler pour le profit d'autrui.

Les personnes qui luttent contre l'oppression continuent d'être confrontées à de nombreux problèmes et à des lacunes dans leurs méthodes. Il est évident que nous devons rester flexibles et nous adapter à notre situation spécifique ; il n'existe pas de programme en douze étapes pour la révolution. Mais certaines erreurs sont suffisamment courantes pour que nous puissions établir des modèles et les éviter. Pour être efficace, une organisation ou un mouvement doit prendre plusieurs mesures fondamentales:

\emph{Remettre en question les} comportements oppressifs et privilégiés intériorisés et agir de manière inclusive, sans se plier aux opinions dominantes (et en fin de compte élitistes).

\emph{Identifier} la nature fondamentale de l'oppression au sein du système et fournir une critique radicale ou un ensemble d'objectifs.

\emph{Baser la} lutte sur des segments moins privilégiés et plus opprimés de la société, plutôt que d'essayer d'établir un lien avec le courant dominant de la classe moyenne.

\emph{S'organiser} de manière localisée, non hiérarchique, décentralisée et autonomiste, afin de promouvoir l'égalité et l'épanouissement au sein du groupe, de créer une plus grande flexibilité et une meilleure adaptation aux conditions locales, et de se protéger contre la répression et l'infiltration de l'État.

\chapter*{\textbf{Dixième Partie}}
\markboth{Dixième Partie}{Dixième Partie}

Envisager un modèle utopique pour le monde entier serait irréaliste et culturellement biaisé, même autoritaire. Chacun doit faire ses propres recherches et parvenir à ses propres conclusions sur le mode de vie qui lui convient le mieux. L'exigence minimale est que nous ne devrions tolérer aucun système qui impose un modèle « correct » à de nombreuses personnes, indépendamment de leur volonté. L'histoire regorge d'exemples (partiellement occultés) d'autres formes d'organisation que nous pouvons utiliser pour déterminer l'organisation la mieux adaptée et la plus réaliste pour répondre à nos besoins actuels.

Chaque communauté devrait décider elle-même des questions d'organisation sociale et économique et s'associer à d'autres communautés dans des associations volontaires pour répondre aux besoins qui ne peuvent être satisfaits par une seule communauté. En attendant, nous avons tous beaucoup en commun et devons lutter ensemble contre le système d'exploitation et de contrôle généralisé à l'échelle mondiale. Ce n'est qu'en détruisant le système d'oppression, quels que soient sa forme et son nom, et en mettant fin au continuum, que nous pourrons ouvrir la voie à une autre forme de lutte: construire des sociétés qui assurent la protection et la subsistance sans recourir à la coercition ni créer de nouveaux systèmes d'oppression.


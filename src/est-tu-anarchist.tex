
Il est probable que tu aies déjà entendu parler des anarchistes et de ce qu'ils sont censés croire. Il y a aussi de fortes chances que tout ce que tu as entendu soit absurde. Beaucoup de gens semblent penser que les anarchistes sont des partisans de la violence, du chaos et de la destruction, qu'ils sont opposés à toute forme d'ordre et d'organisation, ou qu'ils sont des nihilistes fous qui veulent tout faire sauter. En réalité, rien n'est plus éloigné de la vérité. Les anarchistes sont simplement des personnes qui pensent que les êtres humains sont capables de se comporter de manière raisonnable sans y être contraints. C'est une notion très simple. Mais c'est une notion que les riches et les puissants ont toujours trouvée extrêmement dangereuse.

Dans leur plus simple expression, les convictions anarchistes reposent sur deux hypothèses élémentaires. La première est que les êtres humains sont, dans des circonstances ordinaires, à peu près aussi raisonnables et décents qu'ils sont autorisés à l'être, et qu'ils peuvent s'organiser eux-mêmes et organiser leurs communautés sans avoir besoin qu'on leur dise comment faire. La seconde est que le pouvoir corrompt. Par-dessus tout, l'anarchisme consiste simplement à avoir le courage de prendre les principes simples de la décence commune que nous vivons tous et de les suivre jusqu'à leurs conclusions logiques. Aussi étrange que cela puisse paraître, tu es probablement déjà anarchiste à bien des égards, mais tu ne t'en rends pas compte.

Prenons d'abord quelques exemples de la vie quotidienne.

\begin{quotation}
\textbf{S'il y a une file d'attente pour monter dans un bus bondé, attends-tu ton tour et évites-tu de jouer des coudes pour passer devant les autres, même en l'absence de la police ?}
\end{quotation}

Si tu as répondu « oui », tu as l'habitude d'agir comme un anarchiste ! Le principe anarchiste le plus fondamental est l'auto-organisation : l'hypothèse selon laquelle les êtres humains n'ont pas besoin d'être menacés de poursuites judiciaires pour parvenir à des accords raisonnables entre eux ou pour se traiter avec dignité et respect.

Tout le monde pense être capable de se comporter raisonnablement. S'ils pensent que les lois et la police sont nécessaires, c'est uniquement parce qu'ils ne croient pas que les autres le sont. Mais si tu y réfléchis bien, ces personnes ne pensent-elles pas exactement la même chose de vous ? Les anarchistes affirment que presque tous les comportements antisociaux qui nous font penser qu'il est nécessaire d'avoir des armées, des polices, des prisons et des gouvernements pour contrôler nos vies, sont en fait causés par les inégalités et les injustices systématiques que ces armées, ces polices, ces prisons et ces gouvernements rendent possibles. C'est un cercle vicieux. Si les gens sont habitués à être traités comme si leurs opinions ne comptaient pas, ils sont susceptibles de devenir furieux et cyniques, voire violents - ce qui, bien sûr, permet à ceux qui détiennent le pouvoir de dire que leurs opinions ne comptent pas. Une fois qu'ils ont compris que leurs opinions comptent autant que celles des autres, ils tendent à devenir remarquablement compréhensifs. Pour faire court : les anarchistes pensent que c'est en grande partie le pouvoir lui-même et ses effets qui rendent les gens stupides et irresponsables.

\begin{quotation}
\textbf{Es-tu membre d'un club, d'une équipe sportive ou de toute autre association ou organisation volontaire où les décisions ne sont pas imposées par un dirigeant mais prises sur la base du consentement général ?}
\end{quotation}

Si tu as répondu « oui », alors tu appartiens à une organisation qui travaille selon les principes anarchistes ! Un autre principe anarchiste fondamental est l'association volontaire. Il s'agit simplement d'appliquer les principes démocratiques à la vie ordinaire. La seule différence est que les anarchistes pensent qu'il devrait être possible d'avoir une société dans laquelle tout pourrait être organisé selon ces principes, tous les groupes étant basés sur le libre consentement de leurs membres, et donc que tous les styles d'organisation militaires, comme les armées, les bureaucraties ou les grandes entreprises, basés sur des chaînes de commandement, n'auraient plus lieu d'être. Peut-être tu ne crois pas que cela soit possible. Peut-être que si. Mais chaque fois que tu parviens à un accord par consensus, plutôt que par la menace, chaque fois que tu conclus un arrangement volontaire avec une autre personne, que tu parviens à un accord ou à un compromis en tenant dûment compte de la situation ou des besoins particuliers de l'autre personne, tu es anarchiste - même si tu ne t’en rend pas compte.

L'anarchisme est simplement la façon dont les gens agissent lorsqu'ils sont libres de faire ce qu'ils veulent et lorsqu'ils traitent avec d'autres personnes également libres - et donc conscientes de la responsabilité envers les autres que cela implique. Cela nous amène à un autre point crucial : si les gens peuvent être raisonnables et prévenants lorsqu'ils traitent avec des égaux, la nature humaine est telle qu'on ne peut pas leur faire confiance lorsqu'on leur donne du pouvoir sur les autres. Si l'on donne un tel pouvoir à quelqu'un, il en abusera presque invariablement d'une manière ou d'une autre.

\begin{quotation}
\textbf{Crois-tu que la plupart des politiciens sont des porcs égoïstes qui ne se soucient pas vraiment de l'intérêt général ? Penses-tu que nous vivons dans un système économique stupide et injuste ?}
\end{quotation}

Si tu as répondu « oui », tu adhères à la critique anarchiste de la société actuelle, du moins dans ses grandes lignes. Les anarchistes pensent que le pouvoir corrompt et que ceux qui passent leur vie à chercher le pouvoir sont les dernières personnes qui devraient l'avoir. Les anarchistes pensent que notre système économique actuel est plus susceptible de récompenser les gens pour leur comportement égoïste et sans scrupules que pour leur qualité d'êtres humains décents et bienveillants. La plupart des gens sont de cet avis. La seule différence est que la plupart des gens ne pensent pas qu'il y ait quoi que ce soit à faire à ce sujet, ou en tout cas - et c'est ce que les fidèles serviteurs des puissants sont toujours les plus susceptibles d'insister - quoi que ce soit qui ne finisse pas par empirer les choses.

Et si ce n'était pas le cas ?

Et y a-t-il vraiment une raison de le croire ? Lorsqu'on peut les tester, la plupart des prédictions habituelles sur ce qui se passerait sans État ni capitalisme se révèlent totalement fausses. Pendant des milliers d'années, les gens ont vécu sans gouvernement. Aujourd'hui, dans de nombreuses régions du monde, les gens vivent en dehors du contrôle des gouvernements. Ils ne s'entretuent pas tous. La plupart du temps, ils mènent leur vie comme n'importe qui d'autre. Bien sûr, dans une société complexe, urbaine et technologique, tout cela serait plus compliqué : mais la technologie peut aussi rendre tous ces problèmes beaucoup plus faciles à résoudre. En fait, nous n'avons même pas commencé à réfléchir à ce que pourrait être notre vie si la technologie était réellement mise au service des besoins humains. Combien d'heures devrions-nous réellement travailler pour maintenir une société fonctionnelle - c'est-à-dire si nous nous débarrassions de toutes les professions inutiles ou destructrices comme les télévendeurs, les avocats, les gardiens de prison, les analystes financiers, les experts en relations publiques, les bureaucrates et les politiciens, et si nous détournions nos meilleurs esprits scientifiques de l'armement spatial ou des systèmes boursiers pour mécaniser les tâches dangereuses ou ennuyeuses comme l'extraction du charbon ou le nettoyage des toilettes, et si nous répartissions le travail restant entre tous de manière égale ? Cinq heures par jour ? Quatre heures ? Trois heures ? Deux heures ? Personne ne le sait, car personne ne se pose ce genre de question. Les anarchistes pensent que ce sont précisément ces questions que nous devrions poser.

\begin{quotation}
\textbf{Crois-tu vraiment aux choses que tu dis à tes enfants, ou que tes parents t’ont dit?}
\end{quotation}

« Peu importe qui a commencé. » « Une mauvaise action n'en excuse pas une autre » « Répare tes propres bêtises. » « Fais aux autres… » « Ne sois pas méchant avec les gens juste parce qu'ils sont différents. » Peut-être devrions-nous nous demander si nous mentons à nos enfants lorsque nous leur parlons de bien et de mal, ou si nous sommes prêts à prendre nos propres injonctions au sérieux. Car si l'on pousse ces principes moraux jusqu'à leur conclusion logique, on arrive à l'anarchisme.

Prenez le principe selon lequel une mauvaise action n'en excuse pas une autre. Si on le prenait vraiment au sérieux, cela suffirait à faire disparaître presque tout le fondement de la guerre et du système de justice pénale. Il en va de même pour le partage : nous répétons sans cesse aux enfants qu'ils doivent apprendre à partager, à tenir compte des besoins des autres, à s'entraider ; puis nous partons dans le monde réel où nous supposons que tout le monde est naturellement égoïste et compétitif. Mais un anarchiste ferait remarquer qu'en fait, ce que nous disons à nos enfants est juste. Pratiquement toutes les grandes réalisations de l'histoire de l'humanité, toutes les découvertes et tous les accomplissements qui ont amélioré nos vies, ont été fondés sur la coopération et l'entraide ; même aujourd'hui, la plupart d'entre nous dépensent plus d'argent pour leurs amis et leur famille que pour eux-mêmes ; même s'il est probable qu'il y aura toujours des gens compétitifs dans le monde, il n'y a aucune raison pour que la société soit fondée sur l'encouragement d'un tel comportement, et encore moins pour que les gens se fassent concurrence pour les besoins essentiels de la vie. Cela ne sert que les intérêts des gens au pouvoir, qui veulent que nous vivions dans la peur les uns des autres. C'est pourquoi les anarchistes appellent à une société fondée non seulement sur la libre association, mais aussi sur l'entraide. Le fait est que la plupart des enfants grandissent en croyant à la morale anarchiste, puis doivent progressivement se rendre compte que le monde des adultes ne fonctionne pas vraiment de cette manière. C'est pourquoi tant d'entre eux deviennent rebelles, aliénés, voire suicidaires à l'adolescence, et finalement résignés et amers à l'âge adulte ; leur seul réconfort étant souvent la possibilité d'élever leurs propres enfants et de leur faire croire que le monde est juste. Mais si nous pouvions vraiment commencer à construire un monde qui soit au moins fondé sur des principes de justice ? Ne serait-ce pas là le plus beau cadeau que l'on puisse faire à ses enfants ?

\begin{quotation}
\textbf{Crois-tu que les êtres humains sont fondamentalement corrompus et mauvais, ou que certaines catégories de personnes (les femmes, les personnes de couleur, les gens ordinaires qui ne sont pas riches ou très instruits) sont des spécimens inférieurs, destinés à être gouvernés par leurs supérieurs ?}
\end{quotation}

Si tu as répondu « oui », il semble que vous ne soyez pas anarchiste. Mais si tu as répondu « non », il y a de fortes chances que tu adhères déjà à 90 \% des principes anarchistes et que tu vives ta vie en grande partie en accord avec eux. Chaque fois que tu traites un autre être humain avec considération et respect, tu es anarchiste. Chaque fois que tu règles tes différends avec les autres en parvenant à un compromis raisonnable, en écoutant ce que chacun a à dire plutôt qu'en laissant une personne décider pour les autres, tu es anarchiste. Chaque fois que tu as la possibilité de forcer quelqu'un à faire quelque chose, mais que tu décides plutôt de faire appel à son sens de la raison ou de la justice, tu es anarchiste. Il en va de même chaque fois que tu partages quelque chose avec un ami, que tu décides qui va faire la vaisselle ou que tu fais quoi que ce soit dans un souci d'équité.

On pourrait objecter que tout cela est très bien pour permettre à de petits groupes de personnes de s'entendre, mais que la gestion d'une ville, ou d'un pays, est une tout autre affaire. Et bien sûr, il y a une raison à cela. Même si l'on décentralise la société et que l'on confie le plus de pouvoir possible aux petites communautés, il y aura toujours beaucoup de choses à coordonner, qu'il s'agisse de faire fonctionner les chemins de fer ou de décider des orientations de la recherche médicale. Mais ce n'est pas parce qu'une chose est compliquée qu'il n'y a pas moyen de la faire démocratiquement. Ce serait simplement compliqué. En fait, les anarchistes ont toutes sortes d'idées et de visions différentes sur la manière dont une société complexe pourrait se gérer elle-même. Les expliquer dépasserait largement le cadre d'un petit texte d'introduction comme celui-ci. Il suffit de dire, tout d'abord, que beaucoup de gens ont passé beaucoup de temps à élaborer des modèles de fonctionnement d'une société vraiment démocratique et saine ; mais ensuite, et c'est tout aussi important, aucun anarchiste ne prétend avoir un plan parfait. De toute façon, la dernière chose que nous voulons, c'est imposer des modèles préfabriqués à la société. La vérité est que nous ne pouvons probablement même pas imaginer la moitié des problèmes qui surgiront lorsque nous essaierons de créer une société démocratique ; néanmoins, nous sommes convaincus que, l'ingéniosité humaine étant ce qu'elle est, de tels problèmes peuvent toujours être résolus, tant que c'est dans l'esprit de nos principes de base - qui sont, en fin de compte, simplement les principes de la décence humaine fondamentale.


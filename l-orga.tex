\chapter{Généralités sur le problème de l'organisation}\hypertarget{gnralits-sur-le-problme-de-lorganisation}{}\label{gnralits-sur-le-problme-de-lorganisation}

Outre le problème de la conception stratégique d'un processus visant à révolutionner les conditions qui forment la situation politique actuelle, l'émergence de la conscience de classe pose aux militants de la lutte de classe, en interne, un autre problème : celui de l'organisation.

L'organisation du prolétariat est une exigence, une nécessité, une condition essentielle de son émancipation.

C'est une exigence parce que dans toute situation de lutte, les tâches des militants de la lutte de classe sont différenciées, spécialisées, tout en visant le même résultat tant au niveau des objectifs immédiats qu'à des niveaux plus généraux et plus globaux.

C'est une nécessité dans la mesure où pour la victoire d'une lutte il n'y a pas d'alternative réelle à toute forme d'organisation, en ce sens que toute forme de lutte qui ne se traduit pas en CONCEPTS et FORMES d'organisation ne s'exprime qu'à des niveaux de conceptions politiques nettement inférieurs aux besoins qui ont produit cette lutte. En effet, la réaction de la bourgeoisie, si elle ne parvient pas à vaincre militairement le prolétariat en raison de la ferveur avec laquelle il parvient à exprimer son besoin de libération, est certainement capable de récupérer, sur le plan idéologique et économique, les FORMES dans lesquelles la lutte s'est traduite.

Il s'agit d'une prémisse essentielle parce que, en supposant qu'il n'y ait pas encore eu de révolution sociale (comme cela a été démontré ailleurs), il est par conséquent vrai que des erreurs ont été commises. Ainsi, seule une organisation est capable de traiter les problèmes soulevés par plus d'un siècle de lutte de classe, parce que seul un groupe organisé de militants de la lutte de classe peut lier les besoins immédiats aux enseignements historiques, afin d'assurer que le projet historique et spontané du prolétariat puisse survivre à travers les années.

Le concept essentiel qui sous-entend le terme « organisation'', tel qu'il a été exprimé dans la praxis politique du prolétariat, est celui « d'autogestion'', en ce sens que toute forme de lutte organisée, pour nous anarchistes, ne doit être « gérée'' que par ceux qui la mènent et encore, pour être plus clair et ne pas laisser de place au malentendu, par TOUS ceux qui la mènent.

Ce concept est clairement discriminatoire à l'égard de ceux qui, en déformant la réalité, font de l'organisation un instrument de pouvoir.

Mais ce concept ne signifie nullement que les anarchistes conçoivent la lutte des classes comme un ensemble d'organismes de lutte produits par la réalité de classe qui, par nécessité supérieure, tendent vers les mêmes fins, ni qu'ils confondent les intérêts immédiats du prolétariat (qui sont la base de toute lutte) avec les intérêts réels du prolétariat qui sont la base de la lutte des classes.

Pour nous, il y a deux niveaux d'organisation, qui correspondent fidèlement à deux niveaux de conscience et de lutte : l'organisation « spécifique'' et l'organisation « de masse''.

L'organisation spécifique, aussi appelée le parti, rassemble les militants de la lutte de classe dont la conscience exige une vision complète et définie de l'ensemble de la problématique de la lutte de classe, c'est-à-dire une théorie précise et un dessein historique articulé et concret.

Les membres de cette organisation existent avant et, si l'on veut, même sans l'organisation. Leur union, la clarification et l'homogénéité de leurs thèses est le premier pas nécessaire à l'avancement de la lutte des classes.

L'organisation spécifique  est précisément anarchiste, en ce sens qu'elle n'est formée que d'anarchistes et qu'elle se distingue clairement des autres organisations spécifiques  par sa théorie, son organisation, sa conception historique et sa pratique caractéristiques.

L'organisation de masse, ou le syndicat, rassemble les différentes catégories de travailleurs sur la base immédiate de la survie et sur la base de la nécessité d'améliorer les conditions de vie.

Ce qui est demandé au syndicat, ce n'est pas une vision globale des problèmes plus généraux de la lutte des classes, mais une capacité pratique et une volonté précise de lutter contre le capital.

Dans la sphère syndicale, l'idéologie joue un rôle, mais seulement dans la mesure où les militants de la lutte des classes qui font partie des organisations spécifiques et qui entrent dans l'organisation de masse en tant que prolétaires y apportent le leur.

Les membres de l'organisation de masse sont tous ceux qui, au sein du prolétariat, ont compris que l'amélioration des conditions de vie ne peut être obtenue que par la force et non par la prière.

Par rapport à ce qui précède, l'organisation de masse n'est pas, ni ne peut être, anarchiste ; Mais en fait, les thèses politiques léninistes (qui voient dans le syndicat la courroie de transmission du parti) et la pratique réformiste (qui fait de la conquête salariale un objectif positif dans la construction du socialisme) se sont traduites, dans l'organisation syndicale, par des formes de gestion hiérarchique de la politique syndicale elle-même, de sorte qu'au niveau stratégique, pour les anarchistes, il y a clairement la nécessité de construire une organisation syndicale dont l'organisation interne est l'autogestion de la ligne politique par tous les prolétaires qui en sont membres.

\chapter{L'organisation spécifique : théorie et pratique}\hypertarget{lorganisation-spcifique--thorie-et-pratique}{}\label{lorganisation-spcifique--thorie-et-pratique}

Parler d'organisation, c'est aborder simultanément deux problèmes différents : le CONCEPT d'organisation et la PRATIQUE d'organisation.

Par concept, on entend l'identification consciente et claire par ceux qui s'organisent des relations qui doivent exister entre les éléments de cet ensemble qu'est l'organisation politique, le parti. Par pratique, on entend la tâche difficile de traduire techniquement et normativement dans la PRATIQUE QUOTIDIENNE d'une organisation politique ces concepts sans qu'ils aient la possibilité de se détériorer ou d'être subvertis et sans, d'autre part, qu'ils restent toujours identiques à eux-mêmes en se détachant des nécessités de l'organisation.

Nous traiterons ici du premier problème, non pas parce qu'il est plus important, ni parce qu'il est moins important, mais seulement parce qu'il est une condition indispensable pour aborder la clarté politique (qui signifie clarté historique) du second problème, à la résolution duquel les « BESOINS POLITIQUES'', qui sont généralement les « nécessités du moment'', entrent comme des conditions déterminantes.

Ce que nous avons dit sert à mettre en lumière les erreurs que l'on peut commettre en abordant le problème de l'organisation de manière globale :

\begin{enumerate}
\item{} manque de clarté dans l'identification du concept organisationnel
\item{} Absence de conséquence dans l'élaboration des règles organisationnelles
\item{} ne pas concevoir le problème de l'évolution de l'organisation dans son ensemble en termes corrects ;
\end{enumerate}

En effet, il faut toujours garder à l'esprit que toute forme d'organisation de la lutte des classes est toujours le produit d'un processus historique avec lequel il faut à chaque fois comparer une expérience ; le risque est en effet de comparer la réalité avec ce qu'elle est aujourd'hui sans se concevoir historiquement et sans évaluer chaque fait à la lumière de l'histoire de la lutte des classes ; il arrive souvent de voir des camarades qui conçoivent la lutte des classes comme née de leur militantisme politique.

\chapter{L'organisation spécifique : les origines historiques et politiques de nos thèses organisationnelles}\hypertarget{lorganisation-spcifique--les-origines-historiques-et-politiques-de-nos-thses-organisationnelles}{}\label{lorganisation-spcifique--les-origines-historiques-et-politiques-de-nos-thses-organisationnelles}

L'histoire de la lutte des classes, l'expérience du stalinisme d'une part et du spontanéisme d'autre part, nous ont montré sans équivoque que le problème de l'organisation est un terrain traître et glissant sur lequel se joue l'avenir de toute lutte et de toute révolution.

De nombreux militants héroïques de la lutte des classes se sont exprimés sur ce sujet et les événements historiques ont montré les nombreuses erreurs et induit la clarté sur cette question.

Cependant, de nombreux camarades refusent encore de juger les thèses organisationnelles de Lénine ou de Kropotkine à l'aune de leurs résultats et risquent, par leur foi irrationnelle dans les paroles des « grands'', de perpétuer des erreurs que le prolétariat a payées si durement et qu'il paie encore aujourd'hui.

L'histoire de la lutte des classes a produit trois concepts organisateurs :

\begin{enumerate}
\item{} l'organisation léniniste, qui conçoit l'organisation comme une structure politique remplaçant la classe ;
\item{} le concept « bordiguiste'' qui conçoit le parti comme l'organe de la classe ;
\item{} l'anarchiste qui conçoit le parti comme une partie de la classe, celui qui est conscient du rôle historique du prolétariat.
\end{enumerate}

Ces trois thèses organisationnelles ont été exprimées mille fois dans l'histoire de la lutte des classes, parfois correctement, parfois incorrectement ; quoi qu'il en soit, aujourd'hui, ceux qui militent dans les rangs des révolutionnaires disposent d'un matériel idéologique et historique suffisant pour prendre des positions conscientes sur le problème de l'organisation.

Nous avons choisi la troisième thèse.

Cependant, il a fait l'objet de nombreuses interprétations dans son articulation pratique et son élaboration générale ; fondamentalement, nous pouvons identifier deux lignes substantielles de dégénération.

Le premier type est celui exemplairement stigmatisé par la F.A.I.(Fédération Anarchiste Italienne) qui, après avoir rassemblé une grande partie du prolétariat sous ses drapeaux en 1945, est tombée dans la fange d'un interclassisme indigne pour n'avoir pas eu la capacité d'élaborer une théorie anarchiste du prolétariat que ses membres ont toujours porté en eux, et pour n'avoir pas eu le courage de se transformer en organisation, faute d'avoir su tirer les leçons historiques des expériences ratées de ses organisations sœurs à d'autres époques et dans d'autres nations.

Le deuxième type est celui stigmatisé par les G.A.A.P. (Groupes d'Action Prolétarienne Anarchiste) qui, après avoir critiqué à juste titre la F.A.I., n'ont pas su se définir positivement, et ont été incapables de prendre la responsabilité de concevoir historiquement leur propre rôle, mais ont cherché, en renouvelant les théories anarchistes, à assumer une position politique « confortable et donc plus acceptable - à leur avis - pour le prolétariat ; ils se sont ainsi perdus et ont perdu leur conscience dans la praxis politique, dans la tactique politique erronée qui les a détruits en peu de temps.

Nos thèses politiques et organisationnelles sont les filles de la « Plateforme des communistes anarchistes de 1926'' - qui représente le recueil le plus abouti des thèses organisationnelles anarchistes, bien qu'avec ses erreurs - du point de vue historique, et elles sont les filles de 1968 du point de vue politique ; c'est-à-dire du point de vue de l'origine des contradictions bourgeoises qui ont provoqué une puissante relance de la lutte des classes qui a réussi à produire un nouveau projet politico-révolutionnaire par rapport à la situation du capitalisme et à initier sous des formes incertaines et souvent contradictoires l'autogestion des luttes et de l'organisation.

\chapter{L'organisation spécifique : principes fondamentaux}\hypertarget{lorganisation-spcifique--principes-fondamentaux}{}\label{lorganisation-spcifique--principes-fondamentaux}

L'organisation spécifique des anarchistes est l'identification consciente des relations qui existent entre les éléments de cet ensemble que représentent les militants de la lutte des classes qui se réfèrent à la théorie libertaire.

\section{Militants}\hypertarget{militants}{}\label{militants}

Le front de la lutte des classes comprend une masse hétérogène de combattants. L'hétérogénéité concerne à la fois le bagage politique avec lequel les problèmes réels posés par la lutte des classes sont abordés et l'engagement politique avec lequel ces problèmes sont abordés.

Beaucoup de prolétaires qui devraient à juste titre constituer le front le plus avancé de cette lutte sont absents : c'est à nous de réussir à les impliquer dans cette lutte qui est la leur ; d'autres sont présents et ne s'impliquent que lorsque leurs intérêts immédiats sont touchés : ils doivent eux aussi s'impliquer plus globalement dans « leur'' lutte ; ceux-ci jouent en tout cas leur rôle dans les organisations de masse.

Certains camarades ont assumé la pleine responsabilité de leurs idées, assortissant cette acquisition d'un engagement politique tout à fait admirable, mettant en pratique leur conscience politique, même au prix de risques et de sacrifices à payer : ce sont les militants de la lutte des classes, ce sont ceux qui veulent l'organisation spécifique et ce sont ceux qui en font partie.

Les sympathisants sont ceux qui, dans les organisations de masse ou dans la vie publique en général, font explicitement référence à la conception politique générale - et pas nécessairement particulière - de l'organisation spécifique.

Ils se reconnaissent dans une idéologie, dans un projet politique général, mais soit ils n'y adhèrent pas complètement parce qu'ils ne sont pas pleinement convaincus, soit ils ne se sentent pas capables de s'engager jusqu'aux limites de leurs capacités dans l'activité politique.

Il est très important pour une organisation anarchiste-communiste d'avoir une distinction claire entre militants et sympathisants, précisément parce que la démocratie interne est absolue, c'est-à-dire que les décisions étant prises par tout le monde en même temps, il doit y avoir une identification précise des membres de l'organisation.

Tout cela est très réel dans le sens où cela représente ce qui se passe réellement dans la réalité quotidienne : ce que tous les combattants de la lutte des classes savent.

Mais quand on regarde autour de soi, on s'aperçoit tout de suite que les militants de la lutte des classes qui composent les organisations spécifiques qui existent aujourd'hui ne sont pas ceux que nous avons définis.

En effet, la conception que nous avons exprimée ne peut subsister en tant que telle et se traduire dans la pratique politique quotidienne que si elle est placée dans le cadre d'une idéologie précise - celle du communisme-anarchisme - qui permet sa réalisation EFFECTIVE ; en d'autres termes, ce n'est que dans le cadre d'une idéologie politique qui n'est qu'une autoconscience de la réalité de classe que le concept réel du militant dans la lutte de classe peut être préservé.

Voyons ce qui se passe dans une organisation spécifique (d'origine deuxième et troisième internationales) où la bureaucratie interne est acceptée et pratiquée.

Le militant de la lutte des classes devient un fonctionnaire du parti ; son militantisme politique devient de facto l'administration d'un pouvoir en partie octroyé par l'Etat (qui reconnaît le parti et ses fonctionnaires comme la « base de l'administration démocratique du pays'') et en partie octroyé par les sympathisants, c'est-à-dire par le consensus que la politique de ce parti (dont on se fiche de savoir si elle est bonne ou mauvaise) a cristallisé en termes de force numérique.

Cependant, la gestion réelle du pouvoir que le parti a ainsi « gagné'' n'est pas entre les mains de tous les « fonctionnaires'' (ex-militants de la lutte des classes), mais seulement de ceux qui, par mérite ou par ruse, ont atteint le statut de « membre du comité central''.

Voyons maintenant ce qui se passe dans ces organisations spécifiques (les divers partis marxiste-léninistes) où la bureaucratie est acceptée mais non pratiquée, étant donné le manque de « pouvoir'', mais où le concept de « politique d'abord'' (Mao) et que le parti est une structure qui remplace la classe, où l'activité politique vise à faire reconnaître les « masses populaires'' dans ce parti (Lénine) est accepté et pratiqué.

Le militant de la lutte des classes devient un militant de l'organisation : il est payé, entraîné, a un rôle et un pouvoir parce qu'il est, avant tout, celui qui est le plus à même de porter la ligne de ce parti aux « masses populaires''.

Son lieu de travail est l'organisation, il est devenu de facto extérieur à la classe en tant qu'unité productive : il ne peut pas faire grève parce qu'en allant contre son travail il irait contre le prolétariat, son temps - tout son temps - est consacré à la politique ; il a perdu son individualité, il s'est aliéné au moment où il est devenu militant de la lutte des classes en tant qu'unité productive, il est devenu militant d'une organisation spécifique qui n'a pas encore le pouvoir et qui n'a pas besoin d'unités productives, mais seulement de personnes disposant de beaucoup de temps pour que les thèses de l'organisation soient connues le plus largement possible.

Les militants de l'organisation spécifique communiste-anarchiste sont et restent avant tout des militants de la lutte des classes, leur travail dans l'organisation fait partie intégrante, mais non oppressive ou aliénante, de leur vie en tant qu'êtres humains et camarades.

Nous savons que tout est politique, de la manière dont nous luttons pour nos intérêts immédiats à la manière dont nous gérons notre vie privée et notre temps libre, en passant par la manière dont nous collaborons à la construction de notre organisation sans économies mais aussi sans privilèges autres que ceux que nous tirons de notre travail politique quotidien dans la lutte des classes.

\section{Responsabilité collective}\hypertarget{responsabilit-collective}{}\label{responsabilit-collective}

Le principe selon lequel chaque militant de la lutte des classes doit répondre de ses actes devant l'ensemble de la classe (et dans la mesure où cela est matériellement impossible, devant sa conscience, c'est-à-dire devant l'organisation politique), bien que conceptuellement valable, doit être rejeté dans une organisation anarchiste.

Si les militants de l'organisation considèrent la théorie libertaire comme correcte pour la lutte des classes et reconnaissent l'organisation dont ils sont membres comme le moyen le plus correct d'exprimer leurs idées politiques, ils doivent par conséquent concevoir l'organisation comme une unité : c'est-à-dire que les membres de l'organisation, dans la mesure où ils agissent collectivement dans la lutte des classes, sont un fait unitaire dans la mesure où ils se reconnaissent comme ayant des idées substantiellement similaires.

À ce stade, l'ensemble de l'organisation devient responsable de l'activité politique de chaque membre qui la représente en fait dans la lutte des classes, et de manière correspondante, chaque membre est responsable de l'activité politique de l'organisation dans son ensemble.

La responsabilité collective n'est cependant pas une loi qui, telle qu'elle est définie, devient existante simplement parce qu'elle est conçue.

La responsabilité collective signifie concrètement que si les militants prennent d'un commun accord une décision qui engage politiquement ceux qui l'ont prise, chaque militant est tenu pour responsable d'un éventuel manquement à sa tâche politique devant tous les autres.

En effet, les décisions prises d'un commun accord et qui concernent les militants de l'organisation constituent la ligne politique de l'organisation.

Nul n'a à répondre de problèmes ou de décisions sur lesquels il n'a pas été appelé à se prononcer ; d'autre part, l'assemblée militante ne doit pas se placer vis-à-vis de chaque militant en inquisiteur (examiner le pourquoi de faits ou de choses qui ne concernent pas l'assemblée) ni en juge des raisons invoquées par le militant pour expliquer son manquement à une tâche qu'il avait entreprise.

L'assemblée ne peut que constater si ce militant ou ce groupe est responsable ou non de la ligne politique et des engagements pris.

L'assemblée ne peut que dire : ce camarade est habilité ou le contraire et agir en conséquence.

\section{L'unité politique de l'organisation}\hypertarget{lunit-politique-de-lorganisation}{}\label{lunit-politique-de-lorganisation}

Les intérêts historiques du prolétariat sont identiques pour toutes les catégories qui le composent, ce qui contraste avec le fait que les intérêts immédiats de chaque catégorie diffèrent souvent considérablement.

Cela dépend de trois ordres de facteurs :

\begin{enumerate}
\item{} les pulsions corporatistes qu'engendre le besoin naturel des travailleurs d’améliorer leur situation;
\item{} la volonté du capital de diviser, de fractionner et d'opposer les luttes des différentes catégories de travailleurs afin de gérer mieux et plus longtemps son pouvoir ;
\item{} l'idéologie réformiste qui, voulant à tout prix partir des besoins immédiats des travailleurs, donne un alibi de gauche aux luttes sectorielles et corporatives qui, sinon, s'élargiraient et se généraliseraient, abordant ainsi le problème des intérêts historiques du prolétariat.
\end{enumerate}

La lutte des classes est donc, par nature, une lutte unitaire et, si elle ne l'est pas aujourd'hui, elle doit le devenir.

À notre époque, de nombreuses organisations politiques issues de la lutte des classes sont présentes dans la réalité politique et prétendent représenter le prolétariat avec leur ligne politique ; en fait, elles deviennent une raison supplémentaire de la désunion et de la fragmentation du prolétariat.

Ceci n'est cependant pas exécrable parce que le processus menant au renversement des relations politiques, économiques et sociales qui existent aujourd'hui n'est clair dans l'esprit de personne et que les différentes forces politiques qui s'opposent aujourd'hui font en réalité un travail intéressant de clarification et de débat sur les questions que la lutte des classes, et nous épargnent les erreurs qu'une seule organisation de type léniniste (s'il y en avait une) pourrait faire commettre au prolétariat.

Une organisation communiste-anarchiste doit exister et se présenter comme une alternative réelle et efficace aux autres forces politiques qui existent aujourd'hui.

Pour ce faire et pour pouvoir réaliser ses propositions politiques, nous ne pouvons la concevoir que comme une organisation unie.

A cela s'ajoute le fait que l'anarchisme-communisme est un projet politique bien défini, unique dans ses contours, de sorte que l'organisation qui le porte ne peut être qu'unitaire.

Mais s'il est vrai que l'unité politique, comme nous l'avons montré, est une nécessité, il est tout aussi vrai que.. :

\begin{enumerate}
\item{} Les décisions politiques, liées à l'analyse politique, peuvent souvent être différentes : a) parce qu'il est difficile de trouver des données et des sources qui soient définitivement fiables et scientifiques ; b) parce que les évaluations sont souvent le résultat d'expériences non généralisables et diverses ;
\item{} la conscience politique du prolétariat n'est pas unique mais diffère selon les époques, les régions, et il en va de même pour les décisions politiques ; en outre, il faut garder à l'esprit que les rapports de force variant, une ligne politique doit souvent tenir compte de cet autre facteur ;
\item{} L'unité politique signifie à la fois l'unité de la ligne politique et l'unité des forces politiques, de sorte que parfois une ligne politique unifiée peut signifier la scission en deux forces politiques, chacune ayant une ligne unifiée.
\end{enumerate}

Il découle de ce qui précède que l'unité politique de l'organisation est et restera toujours un objectif à atteindre, jamais un postulat de départ évident et donné a priori.

Le léninisme, avec la théorie du centralisme démocratique et du comité central, a apporté une réponse à ces contradictions dont l'histoire du prolétariat a montré qu'elles étaient absolument délétères et pleines de risques et qu'elles constituaient une condition préalable à de nombreuses erreurs et déviations.

Notre réponse à cette contradiction, comme toujours, consiste à observer la réalité et à la traduire en concepts politiques ; un fait fondamental est la nécessité de l'unité politique de l'organisation comme condition préalable à son fonctionnement ; un fait gênant est la différence d'opinion qui surgit souvent entre les camarades sur la ligne politique à adopter.

Nous devons donc préserver l'unité politique sans empêcher la diversité d'opinion, car il est bien connu que le pilier de l'évolution d'une ligne politique, comme d'autres choses, est liée à la possibilité de remettre en question la pensée communément acceptée comme juste, si bien sûr cela découle d'un besoin d'amélioration et se fonde sur des faits qui se sont produits et sur de nouvelles considérations jamais formulées auparavant ; en bref, la critique est juste non pas en elle-même, mais dans la mesure où le fait discuté et examiné est soit rejeté parce qu'il n'est RATIONNELLEMENT PAS JUSTE, soit accepté parce qu'il améliore l'attitude et le travail de l'organisation vis-à-vis de la lutte des classes.

Cette attitude présente deux risques :

1) être trop libéral et permettre à n'importe qui (mais toujours les militants de l'organisation) de tout remettre en question à tout moment ;

2) être trop restrictif et permettre une liberté d'expression maximale, mais exiger des militants de l'organisation qu'ils se conforment en tout état de cause aux décisions de l'organisation.

En d'autres termes, le problème reste toujours le même : s'il n'y a pas de leader charismatique, il PEUT y avoir sur certaines questions une majorité et une minorité (toutes deux ayant la même responsabilité et le même devoir envers l'organisation) : que faire ?

Nous avons dit que la minorité est essentielle parce que toutes les innovations naissent d'abord comme le bagage d'une personne ou d'un groupe minoritaire et ensuite, soit parce qu'elles sont démontrées par des mots, soit parce que les faits le prouvent, elles deviennent l'héritage de la majorité.

Nous avons cependant dit que nous ne pouvions pas autoriser une minorité sur tout, ni permettre à la minorité en tant que telle de s'exprimer uniquement au sein de l'organisation.

La conclusion à ce stade suit logiquement, bien qu'en vérité cette logique soit le résultat de plus de 100 ans d'erreurs d'une grande partie du prolétariat qui s'est reconnu dans un anarchisme-communisme non encore défini.

Selon notre vision des choses, il existe une théorie de l'anarchisme-communisme qui est le résultat de plus de 100 ans d'histoire : c'est-à-dire que l'abstraction en termes verbaux et conceptuels des expériences anarchistes, que nous faisons aujourd'hui, ne peut absolument pas être remise en question.

Ses lignes substantielles et essentielles sont précisément l'identification historique de notre « être politique'', ce qui signifie concrètement que le document théorique, bien qu'il ne prétende pas résumer l'anarchisme-communisme dans son intégralité, résume notre mémoire historique en évaluant ses erreurs historiquement et à travers l'expérience acceptée, et ne peut donc pas être remis en question.

En d'autres termes, ceux qui la remettent en question ne peuvent pas la concilier avec l'appartenance à notre organisation.

Ceux qui contredisent notre théorie en étant dans l'organisation, dès qu'ils expriment cette dissidence, ils n'en font plus partie.

Si elle fait partie d'un mouvement communiste anarchiste plus large et pas toujours pleinement conscient, nous devrions l'évaluer et considérer ses innovations comme un pas en avant et donc les adapter, ou un pas en arrière et donc les critiquer, mais toujours après que cette minorité qui a émergé de l'organisation ait survécu à la lutte des classes et vérifié ses thèses avec l'expérience et l'effort.

Mais notre théorie n'est pas très nette, ni très précise, elle est surtout discriminante, c'est-à-dire qu'elle sert à éliminer les erreurs déjà commises par les anarchistes-communistes et qu'elle sert à nous discriminer des autres composantes de la lutte des classes, donc elle est aussi précise qu'elle a besoin d'être, c'est une PLATEFORME sur laquelle ne peuvent se tenir que quelques personnes qui ne pensent pas forcément de manière identique sur tout ce qui n'est pas discriminant ou qui reste à vérifier.

Concrètement et concrètement : sur ce qui est écrit, pas de minorité ; sur ce qui n'est pas écrit, liberté d'interprétation.

La rigidité que nous avons exprimée à l'égard d'une éventuelle minorité sur la théorie découle du fait que nous la croyons absolument correcte parce que l'histoire, avec ses vérifications et ses faits, l'a prouvée et que nous croyons également pouvoir la prouver scientifiquement à tous ceux qui n'ont aucun intérêt à croire le contraire.

La Stratégie Fondamentale de notre organisation découle de l'analyse de la situation politique actuelle ; plus précisément, la Stratégie Fondamentale est la vision que les anarchistes-communistes ont du pouvoir et des forces politiques contre-révolutionnaires, leur évaluation en termes stratégiques afin de définir le rôle concret que les anarchistes-communistes doivent jouer s'ils veulent réaliser les conditions historiques subjectives et objectives, c'est-à-dire si la lutte des classes conduit à une « période de transition'', pour construire une société sans classes.

Ainsi, si la Théorie sert à définir les prémisses historiquement déduites qui permettent de se définir comme communiste-anarchiste, la Stratégie Fondamentale sert à analyser les conditions du pouvoir politico-économique et celles des forces de gauche contre-révolutionnaires afin de définir notre rôle historique aujourd'hui et en transition.

La Stratégie Fondamentale doit être unifiée car, en définissant le rôle historique dans le moment présent des anarchistes-communistes organisés dans notre organisation, elle est l'âme même, la motivation la plus profonde et la plus convaincue, la raison la plus profonde de toutes nos actions politiques.

L'absence d'homogénéité à ce niveau de l'élaboration des politiques conduirait inévitablement au chaos dans la prise de décision sur les questions les plus triviales de stratégie, de méthodologie et d'alliances.

Mais d'autre part, l'analyse sur laquelle repose la définition de notre rôle (c'est-à-dire l'organisation de la lutte de classe et la période de transition) repose non seulement sur ce qui découle de notre théorie, c'est-à-dire de notre qualité d'anarchistes-communistes, mais aussi sur l'analyse du capitalisme, du socialisme d'État et du réformisme. Ces analyses peuvent être fausses à des degrés divers ; il est très important qu'elles soient correctes, car c'est d'elles que découle la définition de notre rôle ; mais il n'en reste pas moins vrai qu'elles peuvent être fausses ou du moins, dans la moins mauvaise hypothèse, il est possible à deux anarchistes-communistes d'arriver à la définition de deux stratégies fondamentales, chacune conforme à l'anarchisme-communisme.

Le problème d'une minorité probable s'est donc posé.

D'une part, la division au sein de l'organisation sur ce bagage politique n'est pas possible, d'autre part, la division anarchistes-communistes est à rejeter comme tout aussi néfaste.

En premier lieu, plus l'analyse stratégique sous-jacente est scientifique, mieux c'est ; une minorité, conduisant ainsi à une analyse approfondie des questions, sera la bienvenue et utile, surtout si le fait de surmonter la divergence entre la majorité et la minorité signifie que la stratégie sous-jacente de l'organisation sera plus scientifique.

Si les dissensions s'aggravent et ne peuvent être résolues, une scission de l'organisation est inévitable, à moins que la minorité dissidente ne s'abstienne de communiquer son désaccord à l'extérieur afin de ne pas rompre l'unité de l'organisation et qu'elle soit en mesure d'apporter des critiques constructives qui n'entravent pas le débat interne, c'est-à-dire à moins que la dissension n'absorbe complètement l'ensemble de l'organisation.

En conclusion, en cas de désaccord sur la stratégie fondamentales :

\begin{enumerate}
\item{} faire remonter le désaccord en interne et tenter de le résoudre en rendant l'analyse plus scientifique (TOUJOURS CONFORME À LA THÉORIE) ;
\item{} la minorité quitte l’organisation si elle estime qu'elle doit exprimer son désaccord à l'extérieur ;
\item{} la minorité est expulsée si la majorité estime que la minorité, avec son désaccord exprimé en interne, empêche les autres activités de l'organisation de se dérouler.
\end{enumerate}

Si l'on résume le tout, on peut dire que s'il est possible que deux fractions opposées se créent sur des questions stratégiques fondamentales, il est improbable qu'une scission ne se produise à moins que des intérêts économiques ou de pouvoir ne soient en jeu.

Nous concluons en disant que : dans la mesure où au sein des militants de notre organisation il n'y aura pas de positions de pouvoir ou de situations de prestige, dans la même mesure il n'y aura pas de divisions sur des questions stratégiques fondamentales, mais toute dissidence rationnellement et scientifiquement valable servira de toute façon à une définition plus précise et plus juste de notre rôle dans la lutte des classes.

La stratégie politique est la ligne générale d'intervention dans la classe en fonction de la réalité objective, des capacités subjectives et, bien sûr, de la théorie et de la stratégie sous-jacentes.

L'unité consciente et libre de tous les militants de l'organisation sur la question de la stratégie politique est évidemment une condition favorable très importante pour la réalisation des objectifs définis dans la stratégie politique.

Voyons maintenant comment et pourquoi des minorités peuvent être créées et ce qu'il doit advenir de ces minorités.

Des divergences sur la partie constructive de la stratégie politique peuvent apparaître :

\begin{enumerate}
\item{} par la non-conformité d'une stratégie politique proposée avec la Théorie et la Stratégie Fondamentale ;
\item{} par des différences d'analyse politique ;
\item{} par des différences dans l'évaluation de l'analyse politique ;
\item{} par des différences dans l'évaluation de la situation subjective.
\end{enumerate}

Si les divergences proviennent de contradictions dans la stratégie politique par rapport à la stratégie et à la théorie sous-jacentes, on se comporte comme nous l'avons vu précédemment.

Si les divergences sont liées à des analyses différentes de la réalité, le problème est d'abord d'évaluer s'il est possible surmonter ces divergences en clarifiant et en analysant mieux, ce qui conduit, si c'est bien fait, à des conclusions plus exactes et plus complètes ; si cela ne conduit pas à des conclusions unanimes, l'organisation aura une stratégie politique officielle qui sera celle de la majorité et une ou plusieurs hypothèses de stratégie politique d'une ou plusieurs minorités qui pourront exprimer leur thèse à l'extérieur, mais en tout cas dans l'élaboration de la tactique et donc dans la pratique politique, elles ne devront pas contredire la thèse de la majorité.

Si chaque divergence analytique aboutit à une vision du rôle politique de l'organisation qui contredit celle de la majorité, il arrive que la thèse minoritaire, par le libre choix de ceux qui la soutiennent, reste « verbale'', c'est-à-dire qu'elle ne s'exprime (aussi a l’exterieur de l’organisation) que par des mots et, dans ce cas, l'unité politique demeure, ou qu'elle s'exprime par une tactique qui contredit celle de la majorité et, dans ce cas, la minorité, rompant consciemment l'unité politique de l'organisation, doit être expulsée.

Il peut arriver - et ici il faut être très prudent - que des désaccords surgissent sur des questions d'évaluation de l'analyse objective et des conditions subjectives : cela doit être lié à l’optimisme ou au pessimisme découlant des conditions différentes dans lesquelles se trouvent les compagnons.

Nous devons donc veiller à ne pas généraliser à l'ensemble de l'organisation les évaluations positives qui découlent d'une situation politique positive, ni généraliser les évaluations négatives qui découlent d'une situation politique negative ; à ne pas créer de faux problèmes ou de failles à ce sujet et à rechercher un accord et une médiation équitables sans tomber dans l'erreur « maximaliste'' qui consiste à rompre pour ne pas avoir à trouver d’accord. Les médiations ne doivent pas être considérées comme des compromis, mais seulement comme un juste équilibre.

En conclusion, nous pouvons dire qu'une organisation dotée d'une théorie et d'une stratégie unitaires ne devrait pas éprouver de difficultés à rendre sa stratégie politique unitaire. Il est ainsi tant que la minorité n'entre pas dans une activité politique concrète en contradiction avec la majorité, et que le besoin d'unité ne succombe pas au maximalisme dogmatique de ceux qui ne savent pas s'adapter à leurs responsabilités collectives.

Mais que la majorité ait toujours à l'esprit le concept FONDAMENTAL qu'une minorité de militants anarchistes-communistes ne surgit pas au hasard, mais provient soit d'une expérience négative que la majorité n'a pas encore vécue, soit d'une inexpérience que cette minorité surmontera avec le temps ; que la majorité ait donc toujours la possibilité d'officialiser une thèse minoritaire face à des faits qui l'indiquent comme étant plus juste.

La tactique de l'organisation représente l'hypothèse de travail de l'organisation valable d'un congrès à l'autre, basée sur l'analyse de la situation du moment et de ses lignes d'évolution envisageables ; sur les échéances politiques à venir, sur les possibilités subjectives, c'est-à-dire sur la force de l'organisation et les possibilités d'alliances possibles.

Les hypothèses de travail doivent évidemment être strictement liées et conséquentes à la théorie : il faut d'abord rappeler que la thèse de ceux qui disent qu'une tactique non conforme à la théorie pourrait « en fait'', en utilisant les contradictions objectives, permettre, par un renforcement de l'organisation, une plus grande « force'' dans l'exécution d'une tactique conforme à la théorie par la suite, doit être absolument rejetée.

Ce n'est pas vrai, d'abord parce que c'est historiquement vérifiable et qu'il est inutile de se faire des illusions sur le moyen tactique qui, s'il est contradictoire avec la théorie, ne peut en faciliter la mise en œuvre ; ensuite parce que les consensus obtenus sur la base d'une tactique contradictoire avec la théorie seront des consensus d'une théorie « différente'' et donc des consensus qui ne pourront que conduire à des divisions internes au sein de l'organisation, ce qui est en contradiction avec le fait qu'il doit y avoir une unité politique en son sein.

Dans l'histoire, certains ont soutenu qu'il était possible d'utiliser tactiquement les structures de pouvoir de manière révolutionnaire ; il convient ici de rappeler les conséquences délétères que cette façon de penser a entraînées.

La tactique ne doit pas être en contradiction avec la stratégie fondamentale  non-plus, ce qui dépend essentiellement du fait que, l'analyse politique tactique étant une clarification de l'analyse plus générale et globale déjà décidée pour la stratégie fondamentale, les initiatives et les hypothèses du travail politique ne peuvent pas être différentes ; tout au plus peut-on supposer qu'une analyse pour la tactique peut servir à reconnaître comme erronées certaines évaluations faites pour la stratégie fondamentale.

À ce stade, cependant, il est nécessaire de procéder, comme nous l'avons déjà vu, à un audit de la stratégie fondamentale.

Enfin, une tactique contradictoire avec la Stratégie politique serait un non-sens, puisque la Stratégie politique est précisément le concept unificateur des tactiques qui se succèdent dans le temps, et que l'unité politique de l'organisation signifie aussi l'unité historique, au fil des années, de l'activité politique de l'organisation, il est absurde de proposer des tactiques qui contredisent la Stratégie politique.

Mais - il est bon de le rappeler - une analyse tactique peut ou doit, si nécessaire, être le point de départ d'une révision constante de la stratégie politique à la lumière à la fois de l'évolution des formes d'oppression politique et de l'état subjectif de l'organisation.

Dans ces conditions, il est possible, voire probable, qu'à chaque congrès, il y ait toujours une confrontation entre deux ou plusieurs tactiques, toutes issues de la Théorie, de la Stratégie fondamentale et de la Stratégie politique.

Le congrès a pour mission de clarifier les analyses, d'éliminer les malentendus, de corriger les appréciations, etc.

Mais plusieurs tactiques différentes subsistent ; en effet, il n'est pas toujours possible de prouver scientifiquement qu'une tactique politique est sûre.

Une question se pose clairement à ce stade : est-il possible pour l'organisation de s'exprimer à l'extérieur avec deux ou plusieurs tactiques, différentes à la fois dans les mots et dans la pratique ?

En d'autres termes : dans quelle mesure l'unité politique signifie-t-elle l'unité tactique ?

L'homogénéité, en tant qu'anarchistes-communistes, est fondée sur l'unité théorique ; le rôle historique des anarchistes-communistes, qui est unitaire, est prouvé par l'unité stratégique. Les objectifs politiques à long terme, le projet politique qui porte l'organisation est unitaire parce que la stratégie politique est unitaire.

Mais si l'unité interne de l'organisation et la possibilité que chaque débat et chaque minorité soient positifs en découlent, l'unité pour les personnes extérieures, c'est-à-dire la fiabilité de  l'organisation - un concept qui pour beaucoup s'identifie à celui de sérieux politique - est représentée par l'unité tactique.

En outre, la politique d'alliances est d'une grande utilité dans un projet tactique, elle a ses conditions préalables dans l'unité tactique, et la politique de « contraste'' avec les ennemis de classe et les opposants politiques est d'autant plus efficace que l'organisation est unifiée.

Il ne fait aucun doute, en revanche, qu'un désaccord entre camarades ne doit pas provoquer la moindre fracture et, surtout, ne doit pas créer des factions en opposition les unes aux autres, ce qui est la chose la plus dangereuse qui puisse exister ; c'est celle qui sape le plus les fondements de l'unité politique de l'organisation, qui, après tout, réside avant tout dans la justesse des rapports politiques.

Une tactique nationale unifiée, si elle est juste, est très productive pour l'organisation ; une minorité et une majorité peuvent contribuer à rendre la tactique plus juste si les différences sont clarifiées et surmontées, mais sinon cela peut aussi diviser l'organisation.

En l'espace d'un congrès, c'est-à-dire de quelques jours, une organisation doit trouver une tactique claire et unifiée : la tâche est difficile et il faudra compter sur la maturité politique des camarades et leur sérénité pour y parvenir.

La démagogie, le culte de la personnalité, la présomption et la mauvaise foi qui se traduisent par des FAITS POLITIQUES sont les plus grands ennemis de l'unité.

A ce stade, pour conclure, disons :

\begin{enumerate}
\item{} l'organisation en tant que telle s'exprime officiellement avec une seule ligne tactique, mais elle doit en tout état de cause laisser aux lignes tactiques minoritaires la possibilité de s'exprimer verbalement à l'extérieur ;
\item{} Dans la pratique, les sections de l'organisation poursuivront les tactiques qu'elles jugent justes, pour autant que ces tactiques ne soient pas contradictoires de la tactique officielle et majoritaire, étant entendu qu'il serait souhaitable que les minorités, par leur propre décision, s'en tiennent à la tactique officielle dans les faits.
\end{enumerate}

En effet, nous permettrons aux militants de se tromper de tactique, en suivant leur propre raison (et ce peut être la minorité comme la majorité qui se trompe), sans qu'ils quittent l'organisation, car nous savons que parfois on apprend surtout de ses erreurs, à moins qu'une tactique minoritaire ne soit préjudiciable à l'organisation. Dans ce cas :

\begin{enumerate}
\item{} ou que les camarades qui la soutiennent jugent bon de s'abstenir de la pratiquer ;
\item{} soit ils doivent être expulsés, car il est absurde qu'une organisation mène deux tactiques, l'une au détriment de l'autre.
\end{enumerate}

Cependant, s'il existe des tactiques différentes et compatibles, cela signifiera que l'organisation dans son ensemble, ou la majorité de celle-ci, ou les minorités, seront toujours prêtes à réviser leur ligne tactique promptement et rapidement, dans la mesure où la présence de tactiques différentes imposera une vérification et une comparaison constantes de celles-ci.

\section{Le fédéralisme des cadres}\hypertarget{le-fdralisme-des-cadres}{}\label{le-fdralisme-des-cadres}

Dans l'organisation anarchiste-communiste, il n'y a pas d'autorité qui se charge de diriger l'activité politique, ni de « groupe de base » qui mettent en pratique les directives du comité central.

Dans notre organisation, il existe une identité absolue entre ceux qui décident et ceux qui agissent ; les décisions opérationnelles, qui sont le résultat d'une adhésion consciente à la ligne politique de l'organisation que TOUS ont contribué à définir, ne sont prises que par ceux qui, après les avoir prises, les mettent en œuvre.

Dire cela, c'est-à-dire opposer ces deux modes d'organisation, c'est rejeter le centralisme dit démocratique et mettre en œuvre le fédéralisme, fondamental pour qu'une organisation communiste-anarchiste reste identique à elle-même dans le temps.

Il est important de souligner que dans un système fédéraliste, la responsabilité du militant individuel n'est pas envers ses « leaders'', mais envers tous ceux qui travaillent avec lui dans la pratique quotidienne (responsabilité collective).

N'oublions pas non plus que chaque décision doit être fixée et doit être prise en fonction des thèses de l'organisation (unité politique).

Ce concept organisationnel est valable pour tout ensemble de camarades qui veulent pratiquer l'anarchisme-communisme ; c'est-à-dire que si être anarchiste-communiste signifie avoir de la clarté politique, se comporter en pratique comme tel signifie respecter le principe fédéraliste.

Ceci est correct non seulement parce que cette façon de s'organiser appartient à l'essence politique de l'anarchisme-communisme, mais aussi parce qu'elle est utile et productive pour la croissance de la conscience de classe.

Nous pouvons en effet dire plus, c'est-à-dire que nous pouvons identifier dans le fédéralisme l'évolution conséquente des formes d'organisation autonome et autogérée que la lutte des classes produit souvent, et nous pouvons également dire que la méthode d'autonomie et d'autogestion d'une structure organisationnelle du prolétariat conduit à la fois à une maturation dans un sens communiste-anarchiste de la conscience politique, et à la conception du fédéralisme comme une forme hiérarchiquement étendue de l'autonomie et de l'autogestion.

Il semble que nous ayons fait mouche à ce stade.

Le fédéralisme a pour mission, en tant que structure organisationnelle, de permettre à des structures autonomes et autogérées au niveau local de constituer une force unifiée au niveau national, tout en permettant l'autonomie et l'autogestion locales et en permettant ensuite à l'organisation nationale d'être autogérée.

Pour exister, une structure fédéraliste doit d'abord être claire sur les unités élémentaires qui la composent : c'est-à-dire les militants, les camarades capables d'être réellement des éléments capables d'aborder et de résoudre les problèmes qui se posent au niveau de l'organisation.

Cette considération impose une distinction importante entre les militants de l'organisation, qui ont un niveau de conscience et de connaissance suffisant, et les sympathisants, qui ne sont pas à ce niveau et ne font donc pas partie des structures de l'organisation, ou plutôt, qui sympathisent avec elle mais n'en font pas partie.

L'organisation repose sur ses congrès, qui sont périodiques, et ce sont les décisions de ceux-ci qui représentent l'unité de l'organisation, et chaque congrès, NOUS LE RÉITÉRONS, est dirigé par TOUS LES MILITANTS et n'implique QUE LES MILITANTS. Même si des décisions peuvent être intéressantes pour des alliés ou autres personnes non-membres, seuls les Militants, cependant, sont liés par le concept de responsabilité collective.

Maintenant, nous pouvons nous rendre compte qu'il est possible pour une structure organisationnelle, celle des communistes-anarchistes, d'être composée de nombreuses unités, chacune autonome, et en même temps de rester unitaire dans son ensemble, permettant à ses sympathisants de grandir en son sein en autogérant leur activité politique.

Rappelons toutefois que si la structure fédéraliste fait de la section locale la structure opérationnelle de la ligne politique, il s'ensuit nécessairement que sa fonctionnalité sera liée à l'efficacité opérationnelle de chacune de ses sections.

N'oublions pas non plus que la lutte pour l'anarchisme-communisme gagnera ou perdra si les camarades anarchistes-communistes dans leurs villes ou villages gagnent ou perdent, et que par conséquent l'activité dans nos lieux de vie est le point final vers lequel tous nos efforts doivent converger ; nous voulons dire par là que le but ultime d'une structure organisationnelle nationale est toujours et uniquement de rendre le travail politique de chaque section plus productif (et non pas PLUS FACILE !), plus incisif et crédible le travail politique de chaque section et rappelons-nous qu'en fin de compte l'existence d'une section doit être fonctionnelle et utile au travail politique que \emph{chaque militant} effectue dans le prolétariat.

À ce stade, un problème de taille se pose : celui de l'efficacité maximale du système fédéraliste. Pourquoi s'agit-il d'un problème majeur ?

Il faut veiller à ce que, sous couvert d'efficacité, des propositions d'organisation ne soient pas présentées qui contredisent et annulent les avantages politiques (autonomie et intégrité de l'anarchisme-communisme) que le système fédéraliste nous a permis d'obtenir.

Avant tout, nous voulons que chaque militant n'agisse que s'il est pleinement convaincu de ce qu'il fait, c'est pourquoi nous acceptons les minorités en notre sein ; en effet, nous sommes certains que dans la pratique politique, on ne mûrit et on ne construit quelque chose de bien que si ceux qui agissent comprennent pleinement la vraie valeur des décisions prises (parce qu'ils y ont participé). C'est la seule garantie réelle qui nous permet d'être certains que le sens véritable des décisions prises sera le motif dominant des actions pratiques (différentes les unes des autres pour des raisons objectives) que chaque militant entreprendra sur son lieu de lutte et de travail.

Il nous appartient maintenant de voir comment nous pouvons être efficaces sans créer de comité central.

Pour répondre aux exigences de précision, de qualification et de centralisation qu'impose la lutte des classes, il est nécessaire de REPARTIR CERTAINES TÂCHES à des commissions spécialisées.

Le plus grand danger est que ces commissions deviennent des centres de pouvoir ; pour éviter cela, les commissions ne doivent avoir un pouvoir exécutif que dans le cadre d'une ligne politique tactique décidée lors des congrès et à laquelle les commissions doivent scrupuleusement se conformer.

Il y aura donc deux types de commissions :

\begin{enumerate}
\item{} les commissions « de service'', c'est-à-dire la commission des relations, la commission des finances, la commission de la presse, la commission des sympathisants, et autant d'autres qu'il sera jugé nécessaire lors des congrès ;
\item{} les commissions « de travail'', c'est-à-dire la commission syndicale, la commission scolaire, la commission des relations avec les autres organisations politiques et autant d'autres que les congrès jugeront nécessaires.
\end{enumerate}

La Commission des Relations Internationales mérite un discours séparé ; un discours auquel nous reviendrons lorsque le problème de l'Internationale Anarchiste sera abordé dans son ensemble.

Les commissions seront formées par un groupe de militants d'une même province ; le contrôle du travail des commissions sera effectué par tous les militants, et tous les militants participeront de manière active et productive au travail des commissions.

Ce qui ne doit jamais être créé, c'est la « commission politique'', c'est-à-dire une commission ayant pour tâche spécifique de prendre des décisions politiques particulières en plus ou en remplacement des tactiques décidées lors du congrès.

Sous quelque forme que ce soit, cette proposition doit être absolument rejetée parce qu'elle est une négation du sens même de l'anarchisme-communisme. Cependant, il n'est pas difficile d'émettre l'hypothèse que dans des situations particulièrement « chaudes'' de la vie politique, des décisions rapides et efficaces doivent être prises pour rendre possible l'unité d'action et donc une plus grande incisivité de notre organisation dans la lutte de classe.

Une fois qu'il est établi que cette tâche ne peut être déléguée à aucun comité, car il est absurde qu'une partie de l'organisation prenne des décisions qui s'appliquent à tous, l'organisation crée le CONSEIL NATIONAL.

Il est composé d'un nombre limité de personnes choisies par le congrès en fonction de leurs capacités politiques démontrées et prouvées ; il a pour mission de se réunir dans des situations politiques particulièrement graves et d'émettre un communiqué politique évidemment en stricte conformité avec la théorie, la stratégie fondamentale, la stratégie politique et la tactique qui servira de recommandation à tous les militants qui choisiront LIBREMENT ET SANS AUCUNE IMPOSITION de l'accepter.

Plus les membres du conseil national seront des camarades réellement qualifiés politiquement, plus leur déclaration sera claire, explicite et motivée ; plus les militants seront mûrs, plus ce système fonctionnera.

Les congrès suivants serviront à évaluer le travail du Conseil national et à définir ses limites et ses possibilités.

\chapter{L'organisation spécifique et des déviations communes}\hypertarget{lorganisation-spcifique-et-des-dviations-communes}{}\label{lorganisation-spcifique-et-des-dviations-communes}

Cette partie de la stratégie fondamentale représentera la mémoire historique de notre organisation dans la mesure où chaque fois qu'il y aura des déviations, nous noterons ici les déviations que nous ne connaissons pas aujourd'hui sur la base de nos expériences, alors qu'il peut y avoir des possibilités de déviations dans une direction NON COMMUNISTE-ANARCHIQUE, même avec toutes les clarifications de cette plate-forme. Les camarades qui rejoindront notre organisation doivent également garder à l'esprit qu'au sein de toute structure politique, tant que le capitalisme existe, des lignes politiques subjectivement ou objectivement provocatrices peuvent apparaître, qu'elles soient réformistes ou aventuristes.

Rappelons que si devenir communiste-anarchiste dans cette société peut être relativement facile, continuer à l'être est beaucoup plus difficile et c'est peut-être la tâche la plus difficile qui nous incombe, car nous devrons lutter contre le fascisme, la répression bourgeoise, les pièges du réformisme et surtout contre l'éducation et l'idéologie autoritaire que cette société nous impose.

\emph{(document crée lors du 1er Congrès de la FdCA en 1985)}

